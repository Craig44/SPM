\section{Troubleshooting\label{sec:trouble-shooting}}

\subsection{Introduction}

\SPM\ is a complex system, providing many opportunities for error \textemdash either because your parameter files do not correctly specify your model, or because the model you tried to specify does not work as you expect. When in doubt, ask an experienced user. Debugging versions of \SPM\ can also be compiled that help to track down cryptic errors.

When \SPM\ generates an error and the error message makes no sense, please let the \SPM\ \authors\ know. Even if you manage to fix the problem yourself, we may be able to implement a more helpful error message and make life easier for the next person to encounter the problem. Guidelines for reporting an error are given in Section \ref{sec:error-guidelines}.

Some parameter values of functions or selectivities can result in either very large or very small numbers. These can, on occasion, generate internal numeric overflow errors within \SPM. This is the most common cause of an overflow error, and can result in parameter estimates of \texttt{NaN}. The work-around to this type of error is to impose bounds on parameters that exclude the possibility of an overflow error.

\subsection{Reporting errors\label{sec:reporting-errors}}

If you wish to report a bug or problem with \SPM, then please send a bug report to the \authors. Use the text \texttt{SPM:} as the start of the subject line in the email. Note that following these guidelines will assist the \SPM\ \authors\ identify, reproduce, and hopefully solve any reported bugs. 

Note that \SPM\ is distributed as unsupported software. We will not, as a rule, provide help for users of \SPM\ outside of \href{http://www.niwa.co.nz}{\Organisation} \textemdash although we will usually endeavour to try. And, while we would appreciate being notified of any problems or errors in \SPM, please note that we may not be able to provide timely solutions.

\subsection{Guidelines for reporting a bug in \SPM\label{sec:error-guidelines}}

\begin{enumerate}
\item Detail the version of \SPM\ are you using? e.g., ``\SPM\ \VER Microsoft Windows executable''

\item What operating system or environment are you using? e.g., ``IBM-PC Intel CPU running Microsoft Windows XP Service Pack 3''.

\item Give a brief one-line description of the problem, e.g., ``a segmentation fault was reported''.

\item If the problem is reproducible, please list the exact steps required to cause it, remembering to include the relevant \SPM\ configuration file, other input files, and any out generated. Specify the \emph{exact} command line arguments that were used, e.g., ``Using the command \texttt{spm -e config.spm -q > logfile.out} reports a segmentation fault. The \config s are attached.''

\item If the problem is not reproducible (only happened once, or occasionally for no apparent reason), please describe the circumstances in which it occurred and the symptoms observed (but note it is much harder to reproduce and hence fix non-reproducible bugs, but if several reports are made over time that relate to the same thing, then this may help to track down the problem), e.g., ``SPM crashed, but I cannot reproduce how I did it. It seemed to be related to a local network crash but I cannot be sure.''

\item If the problem causes any error messages to appear, please give the \emph{exact} text displayed, e.g., \texttt{segmentation fault (core dumped)}.

\item Remember to attach all relevant input and output files so that the problem can be reproduced (it can helpful to compress these into a single file). Without these, it may not be possible to determine the cause of the problem. Note that it is helpful to be as specific as possible when describing the problem.

\end{enumerate}
