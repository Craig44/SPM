\section{Quick reference\label{sec:quick-reference}}
\subsection{General commands and subcommands}
\defComArg{IncludeFile}{file\_name}{Include an external file}
\subsection{Population command and subcommand syntax}
\defCom{Model}{Define the spatial structure, population structure, annual cycle, and model years}\par \par
\defSub{CellLength}{The length (distance) of one side of a cell}
\defSub{Nrows}{The number of rows $n_{rows}$ in the spatial structure}
\defSub{Ncols}{The number of columns $n_{cols}$ in the spatial structure}
\defSub{Layer}{The label for the base layer}
\defSub{Categories} {Labels of the categories (rows) of the population component of the partition}
\defSub{MinAge}{Minimum age of the population}
\defSub{MaxAge}{Maximum age of the population}
\defSub{PlusGroup}{Define the largest age as a plus group}
\defSub{InitialisationPhases}{Define the labels of the phases of the initialisation}
\defSub{InitialYear}{Define the first year of the model, immediately following initialisation}
\defSub{CurrentYear}{Define the current year of the model}
\defSub{FinalYear}{Define the final year of the model in projections}
\defSub{TimeSteps} {Define the \command{TimeStep} labels (in order that they are applied) to form the annual cycle}
\par \defComLab{InitialisationPhase}{Define the processes and years of the initialisation phase with label}\par \par
\defSub{Years} {Define the number of years to run}
\defSub{Processes} {Define the processes (in order of occurrence) to run in each year of the initialisation}
\par \defComLab{TimeStep} {Define a time step with label}\par \par
\defSub{Process} {Define the process labels, in the order that they are applied, for the time step}
\par \defComLab{Process} {Define a process with label}\par \par
\defSub{Type} {Define the type of process}
\defSub{R0} {Define the total amount of recruitment at equilibrium abundance levels}
\defSub{Categories} {Define the categories into which recruitment occurs}
\defSub{Proportions} {Define the proportion of recruitment that occurs into each category}
\defSub{Ages} {Define the ages within each category that receive recruitment}
\defSub{Layer} {Name of the layer used to determine where recruitment occurs}
\defSub{R0} {Define the total amount of recruitment at equilibrium abundance levels}
\defSub{Categories} {Define the categories into which recruitment occurs}
\defSub{Proportions} {Define the proportion of recruitment that occurs into each category}
\defSub{Ages} {Define the age within each category that receive recruitment}
\defSub{Steepness} {Define the Beverton-Holt stock recruitment relationship steepness ($h$) parameter}
\defSub{SigmaR} {Define the recruitment variability $\sigma_R$ in the stock-recruitment relationship for projections}
\defSub{Rho} {Define the autocorrelation $\rho$ in the recruitment variability in the stock-recruitment relationship for projections}
\defSub{SSB} {Define the label of the \command{DerivedParameter} that defines the SSB}
\defSub{YCS-Values} {YCS values}
\defSub{YCS-Years} {Years for YCS values}
\defSub{Layer} {Name of the layer used to determine where recruitment occurs}
\defSub{Categories} {Define the categories that ageing is applied to}
\defSub{M} {Define the constant mortality rate to be applied}
\defSub{Categories} {Define the categories that mortality is applied to}
\defSub{Selectivities} {Define the selectivities applied to each category}
\defSub{Years} {Define the years when the mortality rates are applied}
\defSub{M} {Define the mortality rate to be applied for each year}
\defSub{Categories} {Define the categories that mortality is applied to}
\defSub{Selectivities} {Define the selectivities applied to each category}
\defSub{Categories} {Define the categories that the event mortality is applied to}
\defSub{Years} {Define the years where the mortality even is applied}
\defSub{Layers} {Define the layers that specify the event mortality (as the abundance) in each year}
\defSub{Umax}{Define the maximum exploitation rate}
\defSub{Selectivities} {Define the selectivities applied to each category}
\defSub{Penalty} {Define the event mortality penalty label}
\defSub{Categories}{Define the categories that the event mortality is applied to}
\defSub{Years}{Define the years where the mortality even is applied}
\defSub{Layers}{Define the layers that specify the event mortality (as a biomass) in each year}
\defSub{Umax} {Define the maximum exploitation rate}
\defSub{Selectivities}{Define the selectivities applied to each category}
\defSub{Penalty} {Define the event mortality penalty label}
\defSub{From} {Define the category that is the source of the transition process}
\defSub{To} {Define the category that is the sink of the transition process}
\defSub{N} {Define the number of individuals to move}
\defSub{Selectivity} {Define the selectivity applied to the source category}
\defSub{From} {Define the category that is the source of the transition process}
\defSub{To} {Define the category that is the sink of the transition process}
\defSub{Proportion} {Define the proportion of individuals to move}
\defSub{Selectivity} {Define the selectivity applied to the source category}
\defSub{Categories} {Define the categories that the preference function movement is applied to}
\defSub{PreferenceFunctions} {Define the labels of the individual  preference functions that make up the total preference function}
\par \defComLab{PreferenceFunction} {Define a preference function with label}\par \par
\defSub{Type} {Define the type of preference function}
\defSub{Layer} {Defines the layer which supplies the preference function independent variable}
\defSub{Alpha} {Defines the multiplicative constant $\alpha$}
\defSub{Layer} {Defines the layer which supplies the preference function independent variable}
\defSub{Alpha} {Defines the multiplicative constant $\alpha$}
\defSub{Mu} {Defines the $\mu$ parameter of the normal preference function}
\defSub{Sigma} {Defines the $\sigma$ parameter of the normal preference function}
\defSub{Layer} {Defines the layer which supplies the preference function independent variable}
\defSub{Alpha} {Defines the multiplicative constant $\alpha$}
\defSub{Mu} {Defines the $\mu$ parameter of the double-normal preference function}
\defSub{SigmaL} {Defines the $\sigma_L$ parameter of the double-normal preference function}
\defSub{SigmaR} {Defines the $\sigma_R$ parameter of the double-normal preference function}
\defSub{Layer} {Defines the layer which supplies the preference function independent variable}
\defSub{Alpha} {Defines the multiplicative constant $\alpha$}
\defSub{a50} {Defines the $a_{50}$ parameter of the logistic preference function}
\defSub{ato95} {Defines the $a_{to95}$ parameter of the logistic preference function}
\defSub{Layer} {Defines the layer which supplies the preference function independent variable}
\defSub{Alpha} {Defines the multiplicative constant $\alpha$}
\defSub{a50} {Defines the $a_{50}$ parameter of the inverse-logistic preference function}
\defSub{ato95} {Defines the $a_{to95}$ parameter of the inverse-logistic preference function}
\defSub{Layer} {Defines the layer which supplies the preference function independent variable}
\defSub{Alpha} {Defines the multiplicative constant $\alpha$}
\defSub{Lambda} {Defines the $\lambda$ parameter of the exponential-decay preference function}
\defSub{Layer} {Defines the layer which supplies the preference function independent variable}
\defSub{Alpha} {Defines the multiplicative constant $\alpha$}
\defSub{N} {Defines the $N$ parameter of the threshold preference function}
\defSub{Lambda} {Defines the $\lambda$ parameter of the threshold preference function}
\defSub{Layer} {Defines the layer which supplies the preference function independent variable}
\defSub{Alpha} {Defines the multiplicative constant $\alpha$}
\defSub{Biomass} {Defines the $B$ biomass parameter of the threshold biomass preference function}
\defSub{Lambda} {Defines the $\lambda$ parameter of the threshold biomass preference function}
\par \defComLab{Layer} {Define a layer function with label}\par \par
\defSub{Type} {Define the type of layer}
\defSub{Row$x$} {Define the values of the layer for row $x$}
\defSub{Row$x$} {Define the values of the layer for row $x$}
\defSub{Rescale} {Rescale values of the layer}
\defSub{Year} {Define the years}
\defSub{Layers} {Define the layer labels for each of the years}
\par \defComLab{DerivedQuantity} {Define a derived quantity with label}\par \par
\defSub{Type} {Define the type of derived quantity}
\defSub{Categories} {Define the categories are used to calculate the derived quantity}
\defSub{Selectivity} {Define the selectivities}
\defSub{TimeStep} {Define the time step at the end of which, the derived quantity is calculated}
\defSub{Categories} {Define the categories are used to calculate the derived quantity}
\defSub{Selectivity} {Define the selectivities}
\defSub{TimeStep} {Define the time step at the end of which, the derived quantity is calculated}
\par \defComLab{SizeAtAge} {Define a size-at-age relationship with label}\par \par
\defSub{Type} {Define the type of relationship}
\defSub{Linf} {Define the $L_\infty$ parameter of the von Bertalanffy relationship}
\defSub{k} {Define the $k$ parameter of the von Bertalanffy relationship}
\defSub{t0} {Define the $t_0$ parameter of the von Bertalanffy relationship}
\defSub{GrowthProps} {Define the proportion of the year for each time step for evaluating size}
\subsection{Estimation command and subcommand syntax}
\subsection{Output command and subcommand syntax}
