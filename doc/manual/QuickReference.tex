\section{Quick reference\label{sec:quick-reference}}
\subsection{General commands and subcommands}
\defComArg{IncludeFile}{file\_name}{Include an external file}
\subsection{Population command and subcommand syntax}
\defCom{Model}{Define the spatial structure, population structure, annual cycle, and model years}\par
\defSub{CellLength}{The length (distance) of one side of a cell}
\defSub{Nrows}{The number of rows $n_{rows}$ in the spatial structure}
\defSub{Ncols}{The number of columns $n_{cols}$ in the spatial structure}
\defSub{Layer}{The label for the base layer}
\defSub{Categories} {Labels of the categories (rows) of the population component of the partition}
\defSub{SizeAtAge} {Define the label of the associated size-at-age relationship for each category}
\defSub{MinAge}{Minimum age of the population}
\defSub{MaxAge}{Maximum age of the population}
\defSub{PlusGroup}{Define the largest age as a plus group}
\defSub{InitialisationPhases}{Define the labels of the phases of the initialisation}
\defSub{InitialYear}{Define the first year of the model, immediately following initialisation}
\defSub{CurrentYear}{Define the current year of the model}
\defSub{FinalYear}{Define the final year of the model in projections}
\defSub{TimeSteps} {Define the \command{TimeStep} labels (in order that they are applied) to form the annual cycle}
\par \defComLab{InitialisationPhase}{Define the processes and years of the initialisation phase with label}\par
\defSub{Years} {Define the number of years to run}
\defSub{Processes} {Define the processes (in order of occurrence) to run in each year of the initialisation}
\par \defComLab{TimeStep} {Define a time step with label}\par
\defSub{Processes} {Define the process labels, in the order that they are applied, for the time step}
\par \defComLab{Process} {Define a process with label}\par
\defSub{Type} {Define the type of process}
\par\textbf{\commandlabsubarg{Process}{Type}{ConstantRecruitment}}\par
\defSub{R0} {Define the total amount of recruitment at equilibrium abundance levels}
\defSub{Categories} {Define the categories into which recruitment occurs}
\defSub{Proportions} {Define the proportion of recruitment that occurs into each category}
\defSub{Ages} {Define the ages within each category that receive recruitment}
\defSub{LayerName} {Name of the layer used to determine where recruitment occurs}
\par\textbf{\commandlabsubarg{Process}{Type}{BHRecruitment}}\par
\defSub{R0} {Define the total amount of recruitment at equilibrium abundance levels}
\defSub{Categories} {Define the categories into which recruitment occurs}
\defSub{Proportions} {Define the proportion of recruitment that occurs into each category}
\defSub{Ages} {Define the age within each category that receive recruitment}
\defSub{Steepness} {Define the Beverton-Holt stock recruitment relationship steepness ($h$) parameter}
\defSub{SigmaR} {Define the recruitment variability $\sigma_R$ in the stock-recruitment relationship for projections}
\defSub{Rho} {Define the autocorrelation $\rho$ in the recruitment variability in the stock-recruitment relationship for projections}
\defSub{SSB} {Define the label of the \command{DerivedParameter} that defines the SSB}
\defSub{YCSValues} {YCS values}
\defSub{YCSYears} {Years for year class strength values}
\defSub{LayerName} {Name of the layer used to determine where recruitment occurs}
\par\textbf{\commandlabsubarg{Process}{Type}{Ageing}}\par
\defSub{Categories} {Define the categories that ageing is applied to}
\par\textbf{\commandlabsubarg{Process}{Type}{ConstantMortalityRate}}\par
\defSub{M} {Define the constant mortality rate to be applied}
\defSub{Categories} {Define the categories that mortality is applied to}
\defSub{Selectivities} {Define the selectivities applied to each category}
\par\textbf{\commandlabsubarg{Process}{Type}{AnnualMortalityRate}}\par
\defSub{Years} {Define the years when the mortality rates are applied}
\defSub{M} {Define the mortality rate to be applied for each year}
\defSub{Categories} {Define the categories that mortality is applied to}
\defSub{Selectivities} {Define the selectivities applied to each category}
\par\textbf{\commandlabsubarg{Process}{Type}{EventMortality}}\par
\defSub{Categories} {Define the categories that the event mortality is applied to}
\defSub{Years} {Define the years where the mortality even is applied}
\defSub{LayerNames} {Define the layers that specify the event mortality (as the abundance) in each year}
\defSub{Umax}{Define the maximum exploitation rate}
\defSub{Selectivities} {Define the selectivities applied to each category}
\defSub{Penalty} {Define the event mortality penalty label}
\par\textbf{\commandlabsubarg{Process}{Type}{BiomassEventMortality}}\par
\defSub{Categories}{Define the categories that the event mortality is applied to}
\defSub{SizeAtAge}{Define the age-weight relationships for each of the categories that the event mortality is applied to}
\defSub{Years}{Define the years where the mortality even is applied}
\defSub{LayerNames}{Define the layers that specify the event mortality (as a biomass) in each year}
\defSub{Umax} {Define the maximum exploitation rate}
\defSub{Selectivities}{Define the selectivities applied to each category}
\defSub{Penalty} {Define the event mortality penalty label}
\par\textbf{\commandlabsubarg{Process}{Type}{CategoryTransition}}\par
\defSub{From} {Define the category that is the source of the transition process}
\defSub{To} {Define the category that is the sink of the transition process}
\defSub{N} {Define the number of individuals to move}
\defSub{Selectivity} {Define the selectivity applied to the source category}
\par\textbf{\commandlabsubarg{Process}{Type}{CategoryTransitionRate}}\par
\defSub{From} {Define the category that is the source of the transition process}
\defSub{To} {Define the category that is the sink of the transition process}
\defSub{Proportion} {Define the proportion of individuals to move}
\defSub{Selectivity} {Define the selectivity applied to the source category}
\par\textbf{\commandlabsubarg{Process}{Type}{MigrationMovement}}\par
\defSub{Categories} {Define the categories that the migration movement event is applied to}
\defSub{SourceLayerName} {Define the label of a layer that defines the source cells of the migration movement event}
\defSub{SinkLayerName} {Define the label of a layer that defines the sink cells of the migration movement event}
\defSub{Proportions} {Define the constant multiplier for the proportions that migrate}
\defSub{Selectivities} {Define the selectivities applied to each category}
\par\textbf{\commandlabsubarg{Process}{Type}{AdjacentCellMovement}}\par
\par\textbf{\commandlabsubarg{Process}{Type}{PreferenceMovement}}\par
\defSub{Categories} {Define the categories that the preference function movement is applied to}
\defSub{PreferenceFunctions} {Define the labels of the individual  preference functions that make up the total preference function}
\par \defComLab{PreferenceFunction} {Define a preference function with label}\par
\defSub{Type} {Define the type of preference function}
\par\textbf{\commandlabsubarg{PreferenceFunction}{Type}{Constant}}\par
\defSub{LayerName} {Defines the layer which supplies the preference function independent variable}
\defSub{Alpha} {Defines the multiplicative constant $\alpha$}
\par\textbf{\commandlabsubarg{PreferenceFunction}{Type}{Normal}}\par
\defSub{LayerName} {Defines the layer which supplies the preference function independent variable}
\defSub{Alpha} {Defines the multiplicative constant $\alpha$}
\defSub{Mu} {Defines the $\mu$ parameter of the normal preference function}
\defSub{Sigma} {Defines the $\sigma$ parameter of the normal preference function}
\par\textbf{\commandlabsubarg{PreferenceFunction}{Type}{DoubleNormal}}\par
\defSub{LayerName} {Defines the layer which supplies the preference function independent variable}
\defSub{Alpha} {Defines the multiplicative constant $\alpha$}
\defSub{Mu} {Defines the $\mu$ parameter of the double-normal preference function}
\defSub{SigmaL} {Defines the $\sigma_L$ parameter of the double-normal preference function}
\defSub{SigmaR} {Defines the $\sigma_R$ parameter of the double-normal preference function}
\par\textbf{\commandlabsubarg{PreferenceFunction}{Type}{Logistic}}\par
\defSub{LayerName} {Defines the layer which supplies the preference function independent variable}
\defSub{Alpha} {Defines the multiplicative constant $\alpha$}
\defSub{a50} {Defines the $a_{50}$ parameter of the logistic preference function}
\defSub{ato95} {Defines the $a_{to95}$ parameter of the logistic preference function}
\par\textbf{\commandlabsubarg{PreferenceFunction}{Type}{InverseLogistic}}\par
\defSub{LayerName} {Defines the layer which supplies the preference function independent variable}
\defSub{Alpha} {Defines the multiplicative constant $\alpha$}
\defSub{a50} {Defines the $a_{50}$ parameter of the inverse-logistic preference function}
\defSub{ato95} {Defines the $a_{to95}$ parameter of the inverse-logistic preference function}
\par\textbf{\commandlabsubarg{PreferenceFunction}{Type}{ExponentialDecay}}\par
\defSub{LayerName} {Defines the layer which supplies the preference function independent variable}
\defSub{Alpha} {Defines the multiplicative constant $\alpha$}
\defSub{Lambda} {Defines the $\lambda$ parameter of the exponential-decay preference function}
\par\textbf{\commandlabsubarg{PreferenceFunction}{Type}{Threshold}}\par
\defSub{LayerName} {Defines the layer which supplies the preference function independent variable}
\defSub{Categories} {Define the categories are used to calculate the abundance}
\defSub{Selectivity} {Define the selectivities applied to each category}
\defSub{Alpha} {Defines the multiplicative constant $\alpha$}
\defSub{N} {Defines the $N$ parameter of the threshold preference function}
\defSub{Lambda} {Defines the $\lambda$ parameter of the threshold preference function}
\par\textbf{\commandlabsubarg{PreferenceFunction}{Type}{ThresholdBiomass}}\par
\defSub{LayerName} {Defines the layer which supplies the preference function independent variable}
\defSub{Categories} {Define the categories are used to calculate the biomass}
\defSub{Selectivity} {Define the selectivities applied to each category}
\defSub{SizeAtAge}{Define the age-weight relationships for each of the categories that the biomass is calculated from}
\defSub{Alpha} {Defines the multiplicative constant $\alpha$}
\defSub{Biomass} {Defines the $B$ biomass parameter of the threshold biomass preference function}
\defSub{Lambda} {Defines the $\lambda$ parameter of the threshold biomass preference function}
\par \defComLab{Layer} {Define a layer function with label}\par
\defSub{Type} {Define the type of layer}
\par\textbf{\commandlabsubarg{Layer}{Type}{Numeric}}\par
\defSub{Row$x$} {Define the values of the layer for row $x$}
\defSub{Rescale} {Rescale values of the layer}
\par\textbf{\commandlabsubarg{Layer}{Type}{Categorical}}\par
\defSub{Row$x$} {Define the values of the layer for row $x$}
\par\textbf{\commandlabsubarg{Layer}{Type}{Distance}}\par
\par\textbf{\commandlabsubarg{Layer}{Type}{Abundance}}\par
\defSub{Categories} {Define the categories are used to calculate the abundance}
\defSub{Selectivity} {Define the selectivities applied to each category}
\par\textbf{\commandlabsubarg{Layer}{Type}{Biomass}}\par
\defSub{Categories} {Define the categories are used to calculate the biomass}
\defSub{Selectivity} {Define the selectivities applied to each category}
\defSub{SizeAtAge}{Define the age-weight relationships for each of the categories that the biomass is calculated from}
\par\textbf{\commandlabsubarg{Layer}{Type}{AbundanceDensity}}\par
\defSub{Categories} {Define the categories are used to calculate the abundance}
\defSub{Selectivity} {Define the selectivities applied to each category}
\par\textbf{\commandlabsubarg{Layer}{Type}{BiomassDensity}}\par
\defSub{Categories} {Define the categories are used to calculate the biomass}
\defSub{Selectivity} {Define the selectivities applied to each category}
\defSub{SizeAtAge}{Define the age-weight relationships for each of the categories that the biomass is calculated from}
\par\textbf{\commandlabsubarg{Layer}{Type}{MetaLayer}}\par
\defSub{Year} {Define the years}
\defSub{LayerNames} {Define the layer labels for each of the years}
\par \defComLab{DerivedQuantity} {Define a derived quantity with label}\par
\defSub{Type} {Define the type of derived quantity}
\par\textbf{\commandlabsubarg{DerivedQuantity}{Type}{Abundance}}\par
\defSub{Categories} {Define the categories are used to calculate the derived quantity}
\defSub{Selectivity} {Define the selectivities}
\defSub{TimeStep} {Define the time step at the end of which, the derived quantity is calculated}
\par\textbf{\commandlabsubarg{DerivedQuantity}{Type}{Biomass}}\par
\defSub{Categories} {Define the categories are used to calculate the derived quantity}
\defSub{Selectivity} {Define the selectivities}
\defSub{TimeStep} {Define the time step at the end of which, the derived quantity is calculated}
\par \defComLab{SizeAtAge} {Define a size-at-age relationship with label}\par
\defSub{Type} {Define the type of relationship}
\par\textbf{\commandlabsubarg{SizeAtAge}{Type}{vonBert}}\par
\defSub{Linf} {Define the $L_\infty$ parameter of the von Bertalanffy relationship}
\defSub{k} {Define the $k$ parameter of the von Bertalanffy relationship}
\defSub{t0} {Define the $t_0$ parameter of the von Bertalanffy relationship}
\defSub{Distribution} {Define the distribution of sizes-at-age around the mean}
\defSub{ByLength} {Specifies if the linear interpolation of c.v.s is a linear function of mean size or of age}
\defSub{cv} {Define the c.v. of the distribution of sizes-at-age around the mean}
\defSub{GrowthProps} {Define the proportion of the year for each time step for evaluating size}
\defSub{SizeWeight} {Define the label of the associated size-weight relationship}
\par\textbf{\commandlabsubarg{SizeAtAge}{Type}{Schnute}}\par
\defSub{y1} {Define the $y_1$ parameter of the Schnute relationship}
\defSub{y2} {Define the $y_2$ parameter of the Schnute relationship}
\defSub{tau1} {Define the $\tau_1$ parameter of the Schnute relationship}
\defSub{tau2} {Define the $\tau_2$ parameter of the Schnute relationship}
\defSub{a} {Define the $a$ parameter of the Schnute relationship}
\defSub{b} {Define the $b$ parameter of the Schnute relationship}
\defSub{ByLength} {Specifies if the linear interpolation of c.v.s is a linear function of mean size or of age}
\defSub{cv} {Define the c.v. of the distribution of sizes-at-age around the mean}
\defSub{GrowthProps} {Define the proportion of the year for each time step for evaluating size}
\defSub{SizeWeight} {Define the label of the associated size-weight relationship}
\par \defComLab{SizeWeight} {Define a size-weight relationship with label}\par
\defSub{Type} {Define the type of relationship}
\par\textbf{\commandlabsubarg{SizeWeight}{Type}{None}}\par
\par\textbf{\commandlabsubarg{SizeWeight}{Type}{Basic}}\par
\defSub{SizeAtAge} {Define the label of the associated size-at-age relationship relationship}
\defSub{a} {Define the $a$ parameter of the basic relationship}
\defSub{b} {Define the $b$ parameter of the basic relationship}
\par \defComLab{Selectivity} {Define a selectivity function with label}\par
\defSub{Type} {Define the type of selectivity function}
\par\textbf{\commandlabsubarg{Selectivity}{Type}{Constant}}\par
\defSub{C} {Defines the $C$ parameter of the selectivity function}
\par\textbf{\commandlabsubarg{Selectivity}{Type}{KnifeEdge}}\par
\defSub{E} {Defines the $E$ parameter of the selectivity function}
\par\textbf{\commandlabsubarg{Selectivity}{Type}{AllValues}}\par
\defSub{V} {Defines the $V$ parameters (one for each age class) of the selectivity function}
\par\textbf{\commandlabsubarg{Selectivity}{Type}{AllValuesBounded}}\par
\defSub{L} {Defines the $L$ parameter of the selectivity function}
\defSub{H} {Defines the $H$ parameter of the selectivity function}
\defSub{V} {Defines the $V$ parameters (one for each age class from $L$ to $H$) of the selectivity function}
\par\textbf{\commandlabsubarg{Selectivity}{Type}{Increasing}}\par
\defSub{L} {Defines the $L$ parameter of the selectivity function}
\defSub{H} {Defines the $H$ parameter of the selectivity function}
\defSub{V} {Defines the $V$ parameters (one for each age class from $L$ to $H$) of the selectivity function}
\par\textbf{\commandlabsubarg{Selectivity}{Type}{Logistic}}\par
\defSub{a50} {Defines the $a_{50}$ parameter of the selectivity function}
\defSub{ato95} {Defines the $a_{to95}$ parameter of the selectivity function}
\par\textbf{\commandlabsubarg{Selectivity}{Type}{LogisticProducing}}\par
\defSub{L} {Defines the $L$ parameter of the selectivity function}
\defSub{H} {Defines the $H$ parameter of the selectivity function}
\defSub{a50} {Defines the $a_{50}$ parameter of the selectivity function}
\defSub{ato95} {Defines the $a_{to95}$ parameter of the selectivity function}
\par\textbf{\commandlabsubarg{Selectivity}{Type}{DoubleNormal}}\par
\defSub{Mu} {Defines the $\mu$ parameter of the selectivity function}
\defSub{SigmaL} {Defines the $\sigma_L$ parameter of the selectivity function}
\defSub{SigmaR} {Defines the $\sigma_R$ parameter of the selectivity function}
\par\textbf{\commandlabsubarg{Selectivity}{Type}{DoubleExponential}}\par
\defSub{x1} {Defines the $x_1$ parameter of the selectivity function}
\defSub{x2} {Defines the $x_2$ parameter of the selectivity function}
\defSub{x0} {Defines the $x_0$ parameter of the selectivity function}
\defSub{y0} {Defines the $y_0$ parameter of the selectivity function}
\defSub{y1} {Defines the $y_1$ parameter of the selectivity function}
\defSub{y2} {Defines the $y_2$ parameter of the selectivity function}
\subsection{Estimation command and subcommand syntax}
\defComArg{Seed}{value}{Defines the random number generator seed}\par
\par \defComLab{MPD}{Define the an estimator with label}\par
\defSub{Type} {Define the type of minimiser}
\par\textbf{\commandlabsubarg{MPD}{Type}{NumericalDifferences}}\par
\defSub{MaxIterations} {Define the maximum number of iterations for the Numerical Differences minimiser}
\defSub{MaxEvaluations} {Define the maximum number of evaluations for the Numerical Differences minimiser}
\defSub{StepSize} {Define the step-size for the Numerical Differences minimiser}
\defSub{GradientTolerance} {Define the gradient tolerance for the Numerical Differences minimiser}
\par\textbf{\commandlabsubarg{MPD}{Type}{DESolver}}\par
\defSub{MaxIterations} {Define the maximum number of iterations for the Differential Evolution minimiser}
\defSub{MaxEvaluations} {Define the maximum number of evaluations for the Differential Evolution minimiser}
\defSub{StepSize} {Define the step-size for the Differential Evolution minimiser}
\defSub{GradientTolerance} {Define the gradient tolerance for the Differential Evolution minimiser}
\par \defCom{MCMC}{Define the MCMC estimation arguments}\par
\defSub{Type} {Define the method of MCMC}
\par\textbf{\commandsubarg{MCMC}{Type}{MetropolisHastings}}\par
\defSub{Start} {Covariance multiplier for the starting point of the Markov chain}
\defSub{Length} {Length of the Markov chain}
\defSub{keep} {Spacing between recorded values in the chain}
\defSub{MaxCorrelation} {Maximum absolute correlation in the covariance matrix of the proposal distribution}
\defSub{CorrelationAdjustmentMethod} {Method for adjusting small variances in the covariance proposal matrix}
\defSub{CorrelationAdjustmentDiff} {Minimum non-zero variance times the range of the bounds in the covariance matrix of the proposal distribution}
\defSub{StepSize} {Initial step-size (as a multiplier of the approximate covariance matrix)}
\defSub{ProposalDistribution} {The shape of the proposal distribution (either \textit{t} or normal)}
\defSub{df} {Degrees of freedom of the multivariate t proposal distribution}
\defSub{Parameter} {Name of the parameter to be profiled}
\defSub{N} {Number of values at which to profile the parameter}
\defSub{Lower} {lower bound on parameter}
\defSub{Upper} {Upper bound on parameter}
\par \defComArg{Estimate}{parameter\_name}{Estimate a free parameter}\par
\defSub{Same}{Names of the other parameters which are constrained to have the same value}
\defSub{Phase}{Phase at which this parameter should be estimated, in point estimation}
\defSub{LowerBound}{Lower bound on this parameter}
\defSub{UpperBound}{Upper bound on this parameter}
\defSub{MCMCFixed}{Should this parameter be fixed during MCMC?}
\defSub{Prior}{Defines the prior for this parameter}
\par\textbf{\commandlabsubarg{Estimate}{Prior}{Uniform}}\par
\par\textbf{\commandlabsubarg{Estimate}{Prior}{UniformLog}}\par
\par\textbf{\commandlabsubarg{Estimate}{Prior}{Normal}}\par
\defSub{Mu}{Defines the mean $\mu$ of the normal prior}
\defSub{cv}{Defines the c.v. $c$ of the normal prior}
\par\textbf{\commandlabsubarg{Estimate}{Prior}{NormalByStdev}}\par
\defSub{Mu}{Defines the mean $\mu$ of the normal by standard deviation prior}
\defSub{stdev}{Defines the standard deviation $\sigma$ of the normal by standard deviation prior}
\par\textbf{\commandlabsubarg{Estimate}{Prior}{Lognormal}}\par
\defSub{Mu}{Defines the mean $\mu$ of the lognormal prior}
\defSub{cv}{Defines the c.v. $c$ of the lognormal prior}
\par\textbf{\commandlabsubarg{Estimate}{Prior}{Beta}}\par
\defSub{A}{The lower value of the range parameter $A$ of the Beta prior}
\defSub{B}{The upper value of the range parameter $B$ of the Beta prior}
\defSub{Mu}{Defines the mean $\mu$ of the Beta prior}
\defSub{stdev}{Defines the standard deviation $\sigma$ of the Beta prior}
\par \defComLab{AgeingError}{Define ageing error with \argument{label}}\par
\defSub{Type} {The type of ageing error}
\par\textbf{\commandlabsubarg{AgeingError}{Type}{None}}\par
\par\textbf{\commandlabsubarg{AgeingError}{Type}{Normal}}\par
\defSub{c} {Parameter of the normal ageing error model}
\par\textbf{\commandlabsubarg{AgeingError}{Type}{OffByOne}}\par
\defSub{k} {The $k$ parameter of the off-by-one ageing error model}
\defSub{p1} {The $p_1$ parameter of the off-by-one ageing error model}
\defSub{p2} {The $p_2$ parameter of the off-by-one ageing error model}
\par \defComLab{q}{Define a catchability constant with \argument{label}}\par
\defSub{q} {Value of the q parameter}
\subsection{Observation command and subcommand syntax}
\defComLab{EventMortalityAtAge}{Define an observation of an event mortality using proportions-at-age data}\par
\defSub{Year} {Define the year that the observation applies to}
\defSub{ProcessLabel} {Define the label of the event mortality process}
\defSub{ProportionTimestep} {Define the interpolated proportion of the time-step passes that the observation applies to}
\defSub{MinAge} {Define the minimum age for the observation}
\defSub{MaxAge} {Define the maximum age for the observation}
\defSub{PlusGroup} {Define is the the maximum age for the observation is a plus group}
\defSub{LayerName} {Name of the categorical layer used to group the spacial cells for the observation}
\defSub{Obs [label]}{Define the following data as observations for the categorical layer with value \texttt{[label]}}
\defSub{Tolerance}{Define the tolerance on the sum-to-one error check in \SPM}
\defSub{N [label]}{Define the following data as error values ($N$) for the categorical layer with value \texttt{[label]}}
\defSub{Distribution}{Define the likelihood distribution}
\defSub{ProcessError}{Define the process error term}
\defSub{Delta}{Define the delta robustifying constant for the distribution}
\defSub{Simulate}{Defines if this observation should be simulated when doing simulations}
\par \defComLab{ProportionsAtAge}{Define an observation using proportions-at-age data}\par
\defSub{Year} {Define the year that the observation applies to}
\defSub{Timestep} {Define the time-step that the observation applies to}
\defSub{ProportionTimestep} {Define the interpolated proportion of the time-step passes that the observation applies to}
\defSub{Categories} {Define the categories}
\defSub{Selectivities} {Define the selectivities applied to each category}
\defSub{MinAge} {Define the minimum age for the observation}
\defSub{MaxAge} {Define the maximum age for the observation}
\defSub{PlusGroup} {Define is the the maximum age for the observation is a plus group}
\defSub{LayerName} {Name of the categorical layer used to group the spacial cells for the observation}
\defSub{Obs [label]}{Define the following data as observations for the categorical layer with value \texttt{[label]}}
\defSub{Tolerance}{Define the tolerance on the sum-to-one error check in \SPM}
\defSub{N [label]}{Define the following data as error values ($N$) for the categorical layer with value \texttt{[label]}}
\defSub{Distribution}{Define the likelihood distribution}
\defSub{ProcessError}{Define the process error term}
\defSub{Delta}{Define the delta robustifying constant for the distribution}
\defSub{Simulate}{Defines if this observation should be simulated when doing simulations}
\par \defComLab{ProportionsByAge}{Define an observation using proportions-by-age data}\par
\defSub{Year} {Define the year that the observation applies to}
\defSub{Timestep} {Define the time-step that the observation applies to}
\defSub{ProportionTimestep} {Define the interpolated proportion of the time-step passes that the observation applies to}
\defSub{Categories} {Define the categories}
\defSub{Categories2} {Define the categories}
\defSub{Selectivities} {Define the selectivities applied to each category}
\defSub{Selectivities2} {Define the selectivities applied to each category}
\defSub{MinAge} {Define the minimum age for the observation}
\defSub{MaxAge} {Define the maximum age for the observation}
\defSub{PlusGroup} {Define is the the maximum age for the observation is a plus group}
\defSub{LayerName} {Name of the categorical layer used to group the spacial cells for the observation}
\defSub{Obs [label]}{Define the following data as observations for the categorical layer with value \argument{[label]}}
\defSub{N [label]}{Define the following data as error values ($N$) for the categorical layer with value \argument{[label]}}
\defSub{Distribution}{Define the likelihood distribution}
\defSub{ProcessError}{Define the process error term}
\defSub{Delta}{Define the delta robustifying constant for the distribution}
\defSub{Simulate}{Defines if this observation should be simulated when doing simulations}
\par \defComLab{Abundance}{Define an observation using abundance data}\par
\defSub{Biomass} {Defines if the abundance is a measure of biomass or number}
\defSub{Year} {Define the year that the observation applies to}
\defSub{Timestep} {Define the time-step that the observation applies to}
\defSub{ProportionTimestep} {Define the interpolated proportion of the time-step passes that the observation applies to}
\defSub{q} {Define the catchability constant for the observation}
\defSub{Categories} {Define the categories into which recruitment occurs}
\defSub{Selectivities} {Define the selectivities applied to each category}
\defSub{LayerName} {Name of the categorical layer used to group the spacial cells for the observation}
\defSub{Obs [label]}{Define the following data as observations for the categorical layer with value \argument{[label]}}
\defSub{cv [label]}{Define the following data as error values ($cv$) for the categorical layer with value \argument{[label]}}
\defSub{Distribution}{Define the likelihood distribution}
\defSub{ProcessError}{Define the process error term}
\defSub{Delta}{Define the delta robustifying constant for the distribution}
\defSub{Simulate}{Defines if this observation should be simulated when doing simulations}
\subsection{Output command and subcommand syntax}
