\section{Quick reference\label{sec:quick-reference}}
\subsection{Population command and subcommand syntax}\par
\defCom{model}{Define the spatial structure, population structure, annual cycle, and model years}\par
\defSub{nrows}{The number of rows $n_{rows}$ in the spatial structure}
\defSub{ncols}{The number of columns $n_{cols}$ in the spatial structure}
\defSub{layer}{The label for the base layer}
\defSub{categories} {Labels of the categories (rows) of the population component of the partition}
\defSub{min\_age}{Minimum age of the population}
\defSub{max\_age}{Maximum age of the population}
\defSub{age\_plus\_group}{Define the largest age as a plus group}
\defSub{cell\_length}{The length (distance) of one side of a cell}
\defSub{size\_at\_age}{Define the label of the associated size-at-age relationship for each category}
\defSub{initialisation\_phases}{Define the labels of the phases of the initialisation}
\defSub{initial\_year}{Define the first year of the model, immediately following initialisation}
\defSub{current\_year}{Define the current year of the model}
\defSub{final\_year}{Define the final year of the model in projections}
\defSub{time\_steps} {Define the \command{time\_step} labels (in order that they are applied) to form the annual cycle}
\par\defComLab{initialisation\_phase}{Define the processes and years of the initialisation phase with label}\par
\defSub{type} {Define the type of initialisation}
\par\textbf{\commandlabsubarg{initialisation\_phase}{type}{iterative}}\par
\defSub{years} {Define the number of years to run}
\defSub{processes} {Define the processes (in order of occurrence) to run in each year of the initialisation}
\par\defComLab{time\_step} {Define a time step with label}\par
\defSub{processes} {Define the process labels, in the order that they are applied, for the time step}
\par\defComLab{process} {Define a process with label}\par
\defSub{type} {Define the type of process}
\par\textbf{\commandlabsubarg{process}{type}{constant\_recruitment}}\par
\defSub{r0} {Define the total amount of recruitment at equilibrium abundance levels}
\defSub{categories} {Define the categories into which recruitment occurs}
\defSub{proportions} {Define the proportion of recruitment that occurs into each category}
\defSub{ages} {Define the ages within each category that receive recruitment}
\defSub{layer} {Name of the layer used to determine where recruitment occurs}
\par\textbf{\commandlabsubarg{process}{type}{BH\_recruitment}}\par
\defSub{r0} {Define the total amount of recruitment at equilibrium abundance levels}
\defSub{categories} {Define the categories into which recruitment occurs}
\defSub{proportions} {Define the proportion of recruitment that occurs into each category}
\defSub{ages} {Define the age within each category that receive recruitment}
\defSub{steepness} {Define the Beverton-Holt stock recruitment relationship steepness ($h$) parameter}
\defSub{sigma\_r} {Define the recruitment variability $\sigma_R$ in the stock-recruitment relationship for projections}
\defSub{rho} {Define the autocorrelation $\rho$ in the recruitment variability in the stock-recruitment relationship for projections}
\defSub{SSB} {Define the label of the \command{derived\_quantity} that defines the SSB}
\defSub{offset} {Define the offset (in years) for the year of the derived quantity that is to be applied in the stock-recruit relationp}
\defSub{YCS\_values} {YCS values}
\defSub{YCS\_years} {Years for year class strength values}
\defSub{layer} {Name of the layer used to determine where recruitment occurs}
\par\textbf{\commandlabsubarg{process}{type}{ageing}}\par
\defSub{categories} {Define the categories that ageing is applied to}
\par\textbf{\commandlabsubarg{process}{type}{constant\_mortality\_rate}}\par
\defSub{m} {Define the constant mortality rate to be applied}
\defSub{categories} {Define the categories that mortality is applied to}
\defSub{selectivities} {Define the selectivities applied to each category}
\defSub{layer} {Name of the layer}
\par\textbf{\commandlabsubarg{process}{type}{annual\_mortality\_rate}}\par
\defSub{years} {Define the years when the mortality rates are applied}
\defSub{m} {Define the mortality rate to be applied for each year}
\defSub{categories} {Define the categories that mortality is applied to}
\defSub{selectivities} {Define the selectivities applied to each category}
\defSub{layer} {Name of the layer}
\par\textbf{\commandlabsubarg{process}{type}{event\_mortality}}\par
\defSub{categories} {Define the categories that the event mortality is applied to}
\defSub{years} {Define the years where the mortality even is applied}
\defSub{layers} {Define the layers that specify the event mortality (as the abundance) in each year}
\defSub{U\_max}{Define the maximum exploitation rate}
\defSub{selectivities} {Define the selectivities applied to each category}
\defSub{penalty} {Define the event mortality penalty label}
\par\textbf{\commandlabsubarg{process}{type}{biomass\_event\_mortality}}\par
\defSub{categories}{Define the categories that the event mortality is applied to}
\defSub{size\_at\_age}{Define the age-weight relationships for each of the categories that the event mortality is applied to}
\defSub{years}{Define the years where the mortality event is applied}
\defSub{layers}{Define the layers that specify the event mortality (as a biomass) in each year}
\defSub{U\_max} {Define the maximum exploitation rate}
\defSub{selectivities}{Define the selectivities applied to each category}
\defSub{penalty} {Define the event mortality penalty label}
\par\textbf{\commandlabsubarg{process}{type}{category\_transition}}\par
\defSub{from} {Define the categories that are the source of the transition process}
\defSub{selectivities} {Define the selectivities applied to the source categories}
\defSub{to} {Define the categories that are the sink of the transition process}
\defSub{years} {Define the years where the category transition is applied}
\defSub{layers} {Define the layers that specify the event mortality (as N for each cell) in each year}
\defSub{penalty} {Define the penalty to encourage models parameter values away from those which result in not enough individuals to move}
\par\textbf{\commandlabsubarg{process}{type}{category\_transition\_rate}}\par
\defSub{from} {Define the category that is the source of the transition process}
\defSub{selectivities} {Define the selectivities applied to the source categories}
\defSub{to} {Define the category that is the sink of the transition process}
\defSub{proportion} {Define the proportion of individuals to move}
\defSub{layer} {Name of the layer}
\par\textbf{\commandlabsubarg{process}{type}{migration\_movement}}\par
\defSub{categories} {Define the categories that the migration movement event is applied to}
\defSub{source\_layer} {Define the label of a layer that defines the source cells of the migration movement event}
\defSub{sink\_layer} {Define the label of a layer that defines the sink cells of the migration movement event}
\defSub{proportions} {Define the constant multiplier for the proportions that migrate}
\defSub{layer} {Name of the layer}
\defSub{selectivities} {Define the selectivities applied to each category}
\par\textbf{\commandlabsubarg{process}{type}{adjacent\_cell\_movement}}\par
\par\textbf{\commandlabsubarg{process}{type}{preference\_movement}}\par
\defSub{categories} {Define the categories that the preference function movement is applied to}
\defSub{preference\_functions} {Define the labels of the individual  preference functions that make up the total preference function}
\par\defComLab{preference\_function} {Define a preference function with label}\par
\defSub{type} {Define the type of preference function}
\par\textbf{\commandlabsubarg{preference\_function}{type}{constant}}\par
\defSub{layer} {Defines the layer which supplies the preference function independent variable}
\defSub{alpha} {Defines the multiplicative constant $\alpha$}
\par\textbf{\commandlabsubarg{preference\_function}{type}{normal}}\par
\defSub{layer} {Defines the layer which supplies the preference function independent variable}
\defSub{alpha} {Defines the multiplicative constant $\alpha$}
\defSub{mu} {Defines the $\mu$ parameter of the normal preference function}
\defSub{sigma} {Defines the $\sigma$ parameter of the normal preference function}
\par\textbf{\commandlabsubarg{preference\_function}{type}{double\_normal}}\par
\defSub{layer} {Defines the layer which supplies the preference function independent variable}
\defSub{alpha} {Defines the multiplicative constant $\alpha$}
\defSub{mu} {Defines the $\mu$ parameter of the double-normal preference function}
\defSub{sigma\_l} {Defines the $\sigma_L$ parameter of the double-normal preference function}
\defSub{sigma\_r} {Defines the $\sigma_R$ parameter of the double-normal preference function}
\par\textbf{\commandlabsubarg{preference\_function}{type}{logistic}}\par
\defSub{layer} {Defines the layer which supplies the preference function independent variable}
\defSub{alpha} {Defines the multiplicative constant $\alpha$}
\defSub{a50} {Defines the $a_{50}$ parameter of the logistic preference function}
\defSub{ato95} {Defines the $a_{to95}$ parameter of the logistic preference function}
\par\textbf{\commandlabsubarg{preference\_function}{type}{inverse\_logistic}}\par
\defSub{layer} {Defines the layer which supplies the preference function independent variable}
\defSub{alpha} {Defines the multiplicative constant $\alpha$}
\defSub{a50} {Defines the $a_{50}$ parameter of the inverse-logistic preference function}
\defSub{ato95} {Defines the $a_{to95}$ parameter of the inverse-logistic preference function}
\par\textbf{\commandlabsubarg{preference\_function}{type}{exponential}}\par
\defSub{layer} {Defines the layer which supplies the preference function independent variable}
\defSub{alpha} {Defines the multiplicative constant $\alpha$}
\defSub{lambda} {Defines the $\lambda$ parameter of the exponential preference function}
\par\textbf{\commandlabsubarg{preference\_function}{type}{threshold}}\par
\defSub{layer} {Defines the layer which supplies the preference function independent variable}
\defSub{categories} {Define the categories are used to calculate the abundance}
\defSub{selectivities} {Define the selectivities applied to each category}
\defSub{alpha} {Defines the multiplicative constant $\alpha$}
\defSub{n} {Defines the $N$ parameter of the threshold preference function}
\defSub{lambda} {Defines the $\lambda$ parameter of the threshold preference function}
\par\textbf{\commandlabsubarg{preference\_function}{type}{threshold\_biomass}}\par
\defSub{layer} {Defines the layer which supplies the preference function independent variable}
\defSub{categories} {Define the categories are used to calculate the biomass}
\defSub{selectivities} {Define the selectivities applied to each category}
\defSub{size\_at\_age}{Define the age-weight relationships for each of the categories that the biomass is calculated from}
\defSub{alpha} {Defines the multiplicative constant $\alpha$}
\defSub{biomass} {Defines the $B$ biomass parameter of the threshold biomass preference function}
\defSub{lambda} {Defines the $\lambda$ parameter of the threshold biomass preference function}
\par\defComLab{layer} {Define a layer function with label}\par
\defSub{type} {Define the type of layer}
\par\textbf{\commandlabsubarg{layer}{type}{numeric}}\par
\defSub{data} {Define the values of the layer}
\defSub{rescale} {Rescale values of the layer}
\par\textbf{\commandlabsubarg{layer}{type}{categorical}}\par
\defSub{data} {Define the values of the layer}
\par\textbf{\commandlabsubarg{layer}{type}{distance}}\par
\par\textbf{\commandlabsubarg{layer}{type}{abundance}}\par
\defSub{categories} {Define the categories are used to calculate the abundance}
\defSub{selectivities} {Define the selectivities applied to each category}
\par\textbf{\commandlabsubarg{layer}{type}{biomass}}\par
\defSub{categories} {Define the categories are used to calculate the biomass}
\defSub{selectivities} {Define the selectivities applied to each category}
\defSub{size\_at\_age}{Define the age-weight relationships for each of the categories that the biomass is calculated from}
\par\textbf{\commandlabsubarg{layer}{type}{abundance\_density}}\par
\defSub{categories} {Define the categories are used to calculate the abundance}
\defSub{selectivities} {Define the selectivities applied to each category}
\par\textbf{\commandlabsubarg{layer}{type}{biomass\_density}}\par
\defSub{categories} {Define the categories are used to calculate the biomass}
\defSub{selectivities} {Define the selectivities applied to each category}
\defSub{size\_at\_age}{Define the age-weight relationships for each of the categories that the biomass is calculated from}
\par\textbf{\commandlabsubarg{layer}{type}{meta}}\par
\defSub{years} {Define the years}
\defSub{layers} {Define the layer labels for each of the years}
\defSub{initialisation\_layer} {Define the layer label to use during the initialisation}
\par\defComLab{derived\_quantity} {Define a derived quantity with label}\par
\defSub{type} {Define the type of derived quantity}
\par\textbf{\commandlabsubarg{derived\_quantity}{type}{abundance}}\par
\defSub{categories} {Define the categories are used to calculate the derived quantity}
\defSub{selectivity} {Define the selectivities}
\defSub{time\_step} {Define the time step at the end of which, the derived quantity is calculated}
\defSub{layer} {Name of the layer}
\par\textbf{\commandlabsubarg{derived\_quantity}{type}{biomass}}\par
\defSub{categories} {Define the categories are used to calculate the derived quantity}
\defSub{selectivities} {Define the selectivities}
\defSub{time\_step} {Define the time step at the end of which, the derived quantity is calculated}
\defSub{layer} {Name of the layer}
\par\defComLab{size\_at\_age} {Define a size-at-age relationship with label}\par
\defSub{type} {Define the type of relationship}
\par\textbf{\commandlabsubarg{size\_at\_age}{type}{von\_Bertalanffy}}\par
\defSub{Linf} {Define the $L_\infty$ parameter of the von Bertalanffy relationship}
\defSub{k} {Define the $k$ parameter of the von Bertalanffy relationship}
\defSub{t0} {Define the $t_0$ parameter of the von Bertalanffy relationship}
\defSub{distribution} {Define the distribution of sizes-at-age around the mean}
\defSub{by\_length} {Specifies if the linear interpolation of c.v.s is a linear function of mean size or of age}
\defSub{cv} {Define the c.v. of the distribution of sizes-at-age around the mean}
\defSub{growth\_proportions} {Define the proportion of the year for each time step for evaluating size}
\defSub{size\_weight} {Define the label of the associated size-weight relationship}
\par\textbf{\commandlabsubarg{size\_at\_age}{type}{Schnute}}\par
\defSub{y1} {Define the $y_1$ parameter of the Schnute relationship}
\defSub{y2} {Define the $y_2$ parameter of the Schnute relationship}
\defSub{tau1} {Define the $\tau_1$ parameter of the Schnute relationship}
\defSub{tau2} {Define the $\tau_2$ parameter of the Schnute relationship}
\defSub{a} {Define the $a$ parameter of the Schnute relationship}
\defSub{b} {Define the $b$ parameter of the Schnute relationship}
\defSub{by\_length} {Specifies if the linear interpolation of c.v.s is a linear function of mean size or of age}
\defSub{cv} {Define the c.v. of the distribution of sizes-at-age around the mean}
\defSub{growth\_proportions} {Define the proportion of the year for each time step for evaluating size}
\defSub{size\_weight} {Define the label of the associated size-weight relationship}
\par\defComLab{size\_weight} {Define a size-weight relationship with label}\par
\defSub{Type} {Define the type of relationship}
\par\textbf{\commandlabsubarg{size\_weight}{type}{none}}\par
\par\textbf{\commandlabsubarg{size\_weight}{type}{basic}}\par
\defSub{a} {Define the $a$ parameter of the basic relationship}
\defSub{b} {Define the $b$ parameter of the basic relationship}
\par\defComLab{selectivity} {Define a selectivity function with label}\par
\defSub{type} {Define the type of selectivity function}
\par\textbf{\commandlabsubarg{selectivity}{type}{constant}}\par
\defSub{alpha} {Defines the $\alpha$ parameter of the selectivity function}
\defSub{c} {Defines the $C$ parameter of the selectivity function}
\par\textbf{\commandlabsubarg{selectivity}{type}{knife\_edge}}\par
\defSub{alpha} {Defines the $\alpha$ parameter of the selectivity function}
\defSub{e} {Defines the $E$ parameter of the selectivity function}
\par\textbf{\commandlabsubarg{selectivity}{type}{all\_values}}\par
\defSub{alpha} {Defines the $\alpha$ parameter of the selectivity function}
\defSub{v} {Defines the $V$ parameters (one for each age class) of the selectivity function}
\par\textbf{\commandlabsubarg{selectivity}{type}{all\_values\_bounded}}\par
\defSub{alpha} {Defines the $\alpha$ parameter of the selectivity function}
\defSub{l} {Defines the $L$ parameter of the selectivity function}
\defSub{h} {Defines the $H$ parameter of the selectivity function}
\defSub{v} {Defines the $V$ parameters (one for each age class from $L$ to $H$) of the selectivity function}
\par\textbf{\commandlabsubarg{selectivity}{type}{increasing}}\par
\defSub{alpha} {Defines the $\alpha$ parameter of the selectivity function}
\defSub{l} {Defines the $L$ parameter of the selectivity function}
\defSub{h} {Defines the $H$ parameter of the selectivity function}
\defSub{v} {Defines the $V$ parameters (one for each age class from $L$ to $H$) of the selectivity function}
\par\textbf{\commandlabsubarg{selectivity}{type}{logistic}}\par
\defSub{alpha} {Defines the $\alpha$ parameter of the selectivity function}
\defSub{a50} {Defines the $a_{50}$ parameter of the selectivity function}
\defSub{ato95} {Defines the $a_{to95}$ parameter of the selectivity function}
\par\textbf{\commandlabsubarg{selectivity}{type}{logistic\_producing}}\par
\defSub{alpha} {Defines the $\alpha$ parameter of the selectivity function}
\defSub{l} {Defines the $L$ parameter of the selectivity function}
\defSub{h} {Defines the $H$ parameter of the selectivity function}
\defSub{a50} {Defines the $a_{50}$ parameter of the selectivity function}
\defSub{ato95} {Defines the $a_{to95}$ parameter of the selectivity function}
\par\textbf{\commandlabsubarg{selectivity}{type}{double\_normal}}\par
\defSub{alpha} {Defines the $\alpha$ parameter of the selectivity function}
\defSub{mu} {Defines the $\mu$ parameter of the selectivity function}
\defSub{sigma\_l} {Defines the $\sigma_L$ parameter of the selectivity function}
\defSub{sigma\_r} {Defines the $\sigma_R$ parameter of the selectivity function}
\par\textbf{\commandlabsubarg{selectivity}{type}{double\_exponential}}\par
\defSub{alpha} {Defines the $\alpha$ parameter of the selectivity function}
\defSub{x1} {Defines the $x_1$ parameter of the selectivity function}
\defSub{x2} {Defines the $x_2$ parameter of the selectivity function}
\defSub{x0} {Defines the $x_0$ parameter of the selectivity function}
\defSub{y0} {Defines the $y_0$ parameter of the selectivity function}
\defSub{y1} {Defines the $y_1$ parameter of the selectivity function}
\defSub{y2} {Defines the $y_2$ parameter of the selectivity function}
\par\defComLab{joint\_selectivity} {Define a joint selectivity}\par
\defSub{selectivities} {Define the labels of the selectivities to be defined as 'joint'}
\subsection{Estimation command and subcommand syntax}\par
\defCom{Estimation}\par\par
\defSub{minimiser} {The label of the minimiser to use, if doing a point estimate}
\defSub{MCMC} {The label of the MCMC to use, of doing an MCMC}
\defSub{profile} {The labels of the profiles to use, if doing a profile}
\defSub{random\_seed}{Defines the random number generator seed}
\par\defComLab{Minimiser}{Define the an minimiser estimator with label}\par\par
\defSub{type} {Define the type of minimiser}
\par\textbf{\commandlabsubarg{minimiser}{type}{numerical\_differences}}\par
\defSub{max\_iterations} {Define the maximum number of iterations for the minimiser}
\defSub{max\_evaluations} {Define the maximum number of evaluations for the minimiser}
\defSub{step\_size} {Define the step-size for the minimiser}
\defSub{gradient\_tolerance} {Define the gradient tolerance for the minimiser}
\par\textbf{\commandlabsubarg{minimiser}{type}{DE\_solver}}\par
\defSub{max\_iterations} {Define the maximum number of iterations for the minimiser}
\defSub{max\_evaluations} {Define the maximum number of evaluations for the minimiser}
\defSub{step\_size} {Define the step-size for the minimiser}
\defSub{gradient\_tolerance} {Define the gradient tolerance for the minimiser}
\par\defComLab{MCMC}{Define the MCMC estimation arguments}\par\par
\defSub{type} {Define the method of MCMC}
\par\textbf{\commandsubarg{MCMC}{type}{Metropolis\_Hastings}}\par
\defSub{start} {Covariance multiplier for the starting point of the Markov chain}
\defSub{length} {Length of the Markov chain}
\defSub{keep} {Spacing between recorded values in the chain}
\defSub{max\_correlation} {Maximum absolute correlation in the covariance matrix of the proposal distribution}
\defSub{correlation\_adjustment\_method} {Method for adjusting small variances in the covariance proposal matrix}
\defSub{correlation\_adjustment\_diff} {Minimum non-zero variance times the range of the bounds in the covariance matrix of the proposal distribution}
\defSub{step\_size} {Initial step-size (as a multiplier of the approximate covariance matrix)}
\defSub{proposal\_distribution} {The shape of the proposal distribution (either \textit{t} or normal)}
\defSub{df} {Degrees of freedom of the multivariate t proposal distribution}
\par\defComLab{profile}{Define the profile parameters and arguments}\par\par
\defSub{parameter} {Name of the parameter to be profiled}
\defSub{n} {Number of values at which to profile the parameter}
\defSub{lower\_bound} {lower bound on parameter}
\defSub{upper\_bound} {Upper bound on parameter}
\par\defComArg{estimate}{parameter\_name}{Estimate an estimable parameter}\par\par
\defSub{same}{Names of the other parameters which are constrained to have the same value}
\defSub{estimation\_phase}{Phase at which this parameter should be estimated, in point estimation}
\defSub{lower\_bounds}{Lower bounds on this parameter}
\defSub{upper\_bound}{Upper bound on this parameter}
\defSub{MCMC\_fixed}{Should this parameter be fixed during MCMC?}
\defSub{prior}{Defines the label for the prior for this parameter}
\par\defComLab{prior}{Define the prior label}\par\par
\defSub{type} {Define the type of prior}
\par\textbf{\commandlabsubarg{prior}{type}{uniform}}\par
\par\textbf{\commandlabsubarg{prior}{type}{uniform\_log}}\par
\par\textbf{\commandlabsubarg{prior}{type}{normal}}\par
\defSub{mu}{Defines the mean $\mu$ of the normal prior}
\defSub{cv}{Defines the c.v. $c$ of the normal prior}
\par\textbf{\commandlabsubarg{prior}{type}{normal\_by\_sd}}\par
\defSub{mu}{Defines the mean $\mu$ of the normal by standard deviation prior}
\defSub{sd}{Defines the standard deviation $\sigma$ of the normal by standard deviation prior}
\par\textbf{\commandlabsubarg{prior}{type}{lognormal}}\par
\defSub{mu}{Defines the mean $\mu$ of the lognormal prior}
\defSub{cv}{Defines the c.v. $c$ of the lognormal prior}
\par\textbf{\commandlabsubarg{prior}{type}{beta}}\par
\defSub{A}{The lower value of the range parameter $A$ of the Beta prior}
\defSub{B}{The upper value of the range parameter $B$ of the Beta prior}
\defSub{mu}{Defines the mean $\mu$ of the Beta prior}
\defSub{sd}{Defines the standard deviation $\sigma$ of the Beta prior}
\par\defComLab{ageing\_error}{Define ageing error with \argument{label}}\par\par
\defSub{type} {The type of ageing error}
\par\textbf{\commandlabsubarg{ageing\_error}{type}{none}}\par
\par\textbf{\commandlabsubarg{ageing\_error}{type}{normal}}\par
\defSub{c} {Parameter of the normal ageing error model}
\par\textbf{\commandlabsubarg{ageing\_error}{type}{off\_by\_one}}\par
\defSub{k} {The $k$ parameter of the off-by-one ageing error model}
\defSub{p1} {The $p_1$ parameter of the off-by-one ageing error model}
\defSub{p2} {The $p_2$ parameter of the off-by-one ageing error model}
\par\defComLab{catchability}{Define a catchability constant with \argument{label}}\par\par
\defSub{q} {Value of the q parameter}
\subsection{Observation command and subcommand syntax}\par
\defComLab{observation}{Define an observation}\par\par
\defSub{type} {Define the type of observation}
\par\textbf{\commandlabsubarg{observation}{type}{event\_mortality\_at\_age}}\par
\defSub{year} {Define the year that the observation applies to}
\defSub{process\_label} {Define the label of the event mortality process}
\defSub{proportion\_time\_step} {Define the interpolated proportion of the time-step passes that the observation applies to}
\defSub{min\_age} {Define the minimum age for the observation}
\defSub{max\_age} {Define the maximum age for the observation}
\defSub{age\_plus\_group} {Define is the the maximum age for the observation is a plus group}
\defSub{layer} {Name of the categorical layer used to group the spacial cells for the observation}
\defSub{obs [label]}{Define the following data as observations for the categorical layer with value \texttt{[label]}}
\defSub{tolerance}{Define the tolerance on the sum-to-one error check in \SPM}
\defSub{error\_value [label]}{Define the following data as error values (e.g., $N$ for multinomial likelihoods, c.v. for lognormal likelihoods, etc.) for the categorical layer with value \texttt{[label]}}
\defSub{distribution}{Define the likelihood distribution}
\defSub{process\_error}{Define the process error term}
\defSub{simulate}{Defines if this observation should be simulated when doing simulations}
\par\textbf{\commandlabsubarg{observation}{type}{proportions\_at\_age}}\par
\defSub{year} {Define the year that the observation applies to}
\defSub{time\_step} {Define the time-step that the observation applies to}
\defSub{proportion\_time\_step} {Define the interpolated proportion of the time-step passes that the observation applies to}
\defSub{categories} {Define the categories}
\defSub{selectivities} {Define the selectivities applied to each category}
\defSub{min\_age} {Define the minimum age for the observation}
\defSub{max\_age} {Define the maximum age for the observation}
\defSub{age\_plus\_group} {Define is the the maximum age for the observation is a plus group}
\defSub{layer} {Name of the categorical layer used to group the spacial cells for the observation}
\defSub{obs [label]}{Define the following data as observations for the categorical layer with value \texttt{[label]}}
\defSub{tolerance}{Define the tolerance on the sum-to-one error check in \SPM}
\defSub{error\_value [label]}{Define the following data as error values (e.g., $N$ for multinomial likelihoods, c.v. for lognormal likelihoods, etc.) for the categorical layer with value \texttt{[label]}}
\defSub{distribution}{Define the likelihood distribution}
\defSub{process\_error}{Define the process error term}
\defSub{simulate}{Defines if this observation should be simulated when doing simulations}
\par\textbf{\commandlabsubarg{observation}{type}{proportions\_by\_category}}\par
\defSub{year} {Define the year that the observation applies to}
\defSub{time\_step} {Define the time-step that the observation applies to}
\defSub{proportion\_time\_step} {Define the interpolated proportion of the time-step passes that the observation applies to}
\defSub{categories} {Define the categories}
\defSub{categories2} {Define the categories}
\defSub{selectivities} {Define the selectivities applied to each category}
\defSub{selectivities2} {Define the selectivities applied to each category}
\defSub{min\_age} {Define the minimum age for the observation}
\defSub{max\_age} {Define the maximum age for the observation}
\defSub{age\_plus\_group} {Define is the the maximum age for the observation is a plus group}
\defSub{layer} {Name of the categorical layer used to group the spacial cells for the observation}
\defSub{obs [label]}{Define the following data as observations for the categorical layer with value \argument{[label]}}
\defSub{error\_value [label]}{Define the following data as error values (e.g., $N$ for multinomial likelihoods, c.v. for lognormal likelihoods, etc.) for the categorical layer with value \texttt{[label]}}
\defSub{distribution}{Define the likelihood distribution}
\defSub{process\_error}{Define the process error term}
\defSub{simulate}{Defines if this observation should be simulated when doing simulations}
\par\textbf{\commandlabsubarg{observation}{type}{abundance}}\par
\defSub{year} {Define the year that the observation applies to}
\defSub{time\_step} {Define the time-step that the observation applies to}
\defSub{proportion\_time\_step} {Define the interpolated proportion of the time-step passes that the observation applies to}
\defSub{catchability} {Define the catchability constant label for the observation}
\defSub{categories} {Define the categories into which recruitment occurs}
\defSub{selectivities} {Define the selectivities applied to each category}
\defSub{layer} {Name of the categorical layer used to group the spacial cells for the observation}
\defSub{obs [label]}{Define the following data as observations for the categorical layer with value \argument{[label]}}
\defSub{error\_value [label]}{Define the following data as error values (e.g., $N$ for multinomial likelihoods, c.v. for lognormal likelihoods, etc.) for the categorical layer with value \texttt{[label]}}
\defSub{distribution}{Define the likelihood distribution}
\defSub{process\_error}{Define the process error term}
\defSub{simulate}{Defines if this observation should be simulated when doing simulations}
\par\textbf{\commandlabsubarg{observation}{type}{biomass}}\par
\defSub{year} {Define the year that the observation applies to}
\defSub{time\_step} {Define the time-step that the observation applies to}
\defSub{proportion\_time\_step} {Define the interpolated proportion of the time-step passes that the observation applies to}
\defSub{catchability} {Define the catchability constant label for the observation}
\defSub{categories} {Define the categories into which recruitment occurs}
\defSub{selectivities} {Define the selectivities applied to each category}
\defSub{layer} {Name of the categorical layer used to group the spacial cells for the observation}
\defSub{obs [label]}{Define the following data as observations for the categorical layer with value \argument{[label]}}
\defSub{error\_value [label]}{Define the following data as error values (e.g., $N$ for multinomial likelihoods, c.v. for lognormal likelihoods, etc.) for the categorical layer with value \texttt{[label]}}
\defSub{distribution}{Define the likelihood distribution}
\defSub{process\_error}{Define the process error term}
\defSub{simulate}{Defines if this observation should be simulated when doing simulations}
\par\defComLab{likelihood}{Define an likelihood}\par\par
\defSub{type} {Define the type of likelihood}
\par\textbf{\commandlabsubarg{likelihood}{type}{binomial}}\par
\defSub{year} {Define the year that the observation applies to}
\defSub{delta}{Define the delta robustifying constant for the distribution}
\par\textbf{\commandlabsubarg{likelihood}{type}{Multinomial}}\par
\defSub{year} {Define the year that the observation applies to}
\defSub{delta}{Define the delta robustifying constant for the distribution}
\par\textbf{\commandlabsubarg{likelihood}{type}{lognormal}}\par
\defSub{year} {Define the year that the observation applies to}
\defSub{delta}{Define the delta robustifying constant for the distribution}
\subsection{Reporting command and subcommand syntax}\par
\defComLab{report}{Define an output report}\par\par
\defSub{type} {Define the type of report}
\par\textbf{\commandlabsubarg{report}{type}{partition}}\par
\defSub{year} {Define the year that the partition report applies to}
\defSub{time\_step} {Define the time-step that the partition report applies to}
\defSub{file\_name} {Define the name of the output file where the report is written}
\defSub{overwrite} {Specify if any previous file with the same name as the output file should be overwritten or appended to}
\par\textbf{\commandlabsubarg{report}{type}{initialisation}}\par
\defSub{phase} {Define the phase of initialisation that the partition report applies to}
\defSub{file\_name} {Define the name of the output file where the report is written}
\defSub{overwrite} {Specify if any previous file with the same name as the output file should be overwritten or appended to}
\par\textbf{\commandlabsubarg{report}{type}{derived\_quantity}}\par
\defSub{derived\_quantity} {Define the label of the derived quantity to print}
\defSub{file\_name} {Define the name of the output file where the report is written}
\defSub{overwrite} {Specify if any previous file with the same name as the output file should be overwritten or appended to}
\par\textbf{\commandlabsubarg{report}{type}{estimate\_values}}\par
\defSub{file\_name} {Define the name of the output file where the report is written}
\defSub{overwrite} {Specify if any previous file with the same name as the output file should be overwritten or appended to}
\par\textbf{\commandlabsubarg{report}{type}{estimate\_summary}}\par
\defSub{file\_name} {Define the name of the output file where the report is written}
\defSub{overwrite} {Specify if any previous file with the same name as the output file should be overwritten or appended to}
\par\textbf{\commandlabsubarg{report}{type}{objective\_function}}\par
\defSub{file\_name} {Define the name of the output file where the report is written}
\defSub{overwrite} {Specify if any previous file with the same name as the output file should be overwritten or appended to}
\par\textbf{\commandlabsubarg{report}{type}{observation}}\par
\defSub{observation} {Define the label of the observation to print}
\defSub{file\_name} {Define the name of the output file where the report is written}
\defSub{overwrite} {Specify if any previous file with the same name as the output file should be overwritten or appended to}
\par\textbf{\commandlabsubarg{report}{type}{layer}}\par
\defSub{layer} {Define the label of the layer to print}
\defSub{year} {Define the year for the printing of the layer}
\defSub{time\_step} {Define the time-step for the printing of the layer}
\defSub{file\_name} {Define the name of the output file where the report is written}
\defSub{overwrite} {Specify if any previous file with the same name as the output file should be overwritten or appended to}
\par\textbf{\commandlabsubarg{report}{type}{selectivity}}\par
\defSub{selectivity} {Define the label of the selectivity to print}
\defSub{year} {Define the year for the printing of the selectivity}
\defSub{time\_step} {Define the time-step for the printing of the selectivity}
\defSub{file\_name} {Define the name of the output file where the report is written}
\defSub{overwrite} {Specify if any previous file with the same name as the output file should be overwritten or appended to}
\par\textbf{\commandlabsubarg{report}{type}{random\_number\_seed}}\par
\defSub{file\_name} {Define the name of the output file where the report is written}
\defSub{overwrite} {Specify if any previous file with the same name as the output file should be overwritten or appended to}
\subsection{Other commands and subcommands}\par
\defComArg{include}{file}{Include an external file}\par\par
