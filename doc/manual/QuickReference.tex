\section{Quick reference\label{sec:quick-reference}}
\subsection*{General commands and subcommands}
\defComArg{include\_file}{file\_name}{Include an external file}
\subsection*{Population command and subcommand syntax}
\defCom{Structure}{Define the spatial and population model structure, annual cycle and model years}\\ \\
\defSub{CellShape}{The shape of the cells in the spatial grid}
\defSub{CellLength}{The length (distance) of one side of a cell}
\defSub{Nrows}{The number of rows $n_{rows}$ in the spatial structure}
\defSub{Ncols}{The number of columns $n_{cols}$ in the spatial structure}
\defSub{Layer}{The label for the base layer}
\defSub{Categories} {Labels of the categories (rows) of the population component of the partition}
\defSub{MinAge}{Minimum age of the population}
\defSub{MaxAge}{Maximum age of the population}
\defSub{PlusGroup}{Define the largest age/size as a plus group}
\defSub{InitialisationPhases}{Define the labels of the phases of the initialisation}
\defSub{InitialYear}{Define the first year of the model, immediately following initialisation}
\defSub{CurrentYear}{Define the current year of the model}
\defSub{FinalYear}{Define the final year of the model in projections}
\defSub{TimeSteps} {Define the \command{TimeStep} labels (in order that they are applied) to form the annual cycle}
\\ \defComLab{InitialisationPhase}{Define the processes and years of the initialisation phase with label}\\ \\
\defSub{Years} {Define the number of years to run}
\defSub{Processes} {Define the processes (in order of occurrence) to run in each year of the initialisation}
\\ \defComLab{TimeStep} {Define a time step with label}\\ \\
\defSub{Process} {Define the process labels, in the order that they are applied, for the time step}
\\ \defComLab{AgeingProcess} {Define an ageing process with label}\\ \\
\defSub{Categories} {Define the categories that ageing is applied to}
\\ \defComLab{ConstantRecruitmentProcess} {Define a constant recruitment process}\\ \\
\defSub{R0} {Define the total amount of recruitment at equilibrium abundance levels}
\defSub{Categories} {Define the categories into which recruitment occurs}
\defSub{Proportions} {Define the proportion of recruitment that occurs into each category}
\defSub{Ages} {Define the ages within each category that receive recruitment}
\defSub{Layer} {Name of the layer used to determine where recruitment occurs}
\\ \defComLab{BHRecruitmentProcess}{Define a Beverton-Holt recruitment process}\\ \\
\defSub{R0} {Define the total amount of recruitment at equilibrium abundance levels}
\defSub{Categories} {Define the categories into which recruitment occurs}
\defSub{Proportions} {Define the proportion of recruitment that occurs into each category}
\defSub{Ages} {Define the age within each category that receive recruitment}
\defSub{Steepness} {Define the Beverton-Holt stock recruitment relationship steepness ($h$) parameter}
\defSub{SigmaR} {Define the recruitment variability $\sigma_R$ in the stock-recruitment relationship}
\defSub{Rho} {Define the autocorrelation $\rho$ in the recruitment variability in the stock-recruitment relationship}
\defSub{SSB} {Define the label of the \command{DerivedParameter} that defines the SSB}
\defSub{YCS-Values} {YCS values}
\defSub{YCS-Years} {Years for YCS values}
\defSub{Layer} {Name of the layer used to determine where recruitment occurs}
\\ \defComLab{MortalityRateProcess} {Define a mortality rate process with label}\\ \\
\defSub{Categories} {Define the categories that ageing is applied to}
\\ \defComLab{MortalityEventProcess} {Define a mortality event process with label}\\ \\
\defSub{Categories} {Define the categories that ageing is applied to}
\\ \defComLab{CategoryTransitionProcess} {Define a category transition process with label}\\ \\
\defSub{Categories} {Define the categories that ageing is applied to}
\\ \defComLab{CategoryMoveProcess} {Define a category shift process with label}\\ \\
\defSub{Categories} {Define the categories that ageing is applied to}
\\ \defComLab{PreferenceMovementProcess} {Define a preference function movement process with label}\\ \\
\defSub{Categories} {Define the categories that ageing is applied to}
\\ \defComLab{PreferenceFunction} {Define a preference function with label}\\ \\
\\ \defComLab{Layer} {Define a layer with label}\\ \\
\defSub{Type} {Define the type of layer}
\defSub{Type} {Define the type of layer}
\defSub{Row$x$} {Define the values of the layer for row $x$}
\defSub{Year} {Define the years for \argument{meta-layer}}
\defSub{Value} {Define the layer labels for each of the years for \argument{meta-layer}}
\\ \defComLab{DerivedQuantity} {Define a derived quantity with label}\\ \\
\defSub{Categories} {Define the categories are used to calculate the derived quantity}
\defSub{Selectivity} {Define the selectivities}
\defSub{TimeStep} {Define the time step at the end of which, the derived quantity is calculated}
\\ \defComLab{ConstantSelectivity} {Define a constant selectivity with label}\\ \\
\\ \defComLab{VaryingSelectivity} {Define an annually varying selectivity with label}\\ \\
\subsection*{Estimation command and subcommand syntax}
\subsection*{Output command and subcommand syntax}
