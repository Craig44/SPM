\section{Reporting command and subcommand syntax\label{sec:report-syntax}}

\subsection{Reports}

The report types available are,

\begin{description}
  \item Print the partition
  \item Print a derived parameter
  \item Print the free parameter values
  \item Print the objective function values
  \item Print an observation values, fits, and residuals
  \item Print a layer
  \item Print a selectivity's values
\end{description}

Each type of report requires a set of subcommands and arguments specific to that report.

\defComLab{report}{Define an output report}

\defSub{type} {Define the type of report}
\defType{String}
\defDefault{None}
\defValue{A valid type of report}

\subsubsection[Print the partition]{\commandlabsubarg{report}{type}{partition}}

\defSub{year} {Define the year that the partition report applies to}
\defType{Integer}
\defDefault{None}
\defValue{A positive integer between \commandsub{model}{initial\_year} and \commandsub{model}{current\_year}}

\defSub{time\_step} {Define the time-step that the partition report applies to}
\defType{Integer}
\defDefault{None}
\defValue{A valid time-step}

\defSub{file\_name} {Define the name of the output file where the report is written}
\defType{String}
\defDefault{None}
\defValue{A valid file name. If not supplied, then output is directed to the standard out}

\subsubsection[Print a derived parameter]{\commandlabsubarg{report}{type}{derived\_parameter}}

\defSub{derived\_parameter} {Define the label of the derived parameter to print}
\defType{String}
\defDefault{None}
\defValue{A valid label from \command{Derived\_parameter}}

\defSub{file\_name} {Define the name of the output file where the report is written}
\defType{String}
\defDefault{None}
\defValue{A valid file name. If not supplied, then output is directed to the standard out}

\subsubsection[Unformatted printing of the free parameters]{\commandlabsubarg{report}{type}{Parameter\_vector}\label{sec:InputFileFormat}}

Prints the free parameters in a format suitable for use with \texttt{spm -i}.

\defSub{file\_name} {Define the name of the output file where the report is written}
\defType{String}
\defDefault{None}
\defValue{A valid file name. If not supplied, then output is directed to the standard out}

\subsubsection[Print the free parameters]{\commandlabsubarg{report}{type}{Parameters}}

\defSub{file\_name} {Define the name of the output file where the report is written}
\defType{String}
\defDefault{None}
\defValue{A valid file name. If not supplied, then output is directed to the standard out}

\subsubsection[Print the Objective function values]{\commandlabsubarg{report}{type}{Objective}}

\defSub{file\_name} {Define the name of the output file where the report is written}
\defType{String}
\defDefault{None}
\defValue{A valid file name. If not supplied, then output is directed to the standard out}

\subsubsection[Print an observation, fits, and residuals]{\commandlabsubarg{report}{type}{observation}}

\defSub{observation} {Define the label of the observation to print}
\defType{String}
\defDefault{None}
\defValue{A valid label from \command{Observation}}

\defSub{obs} {Print the observations values?}
\defType{Logical}
\defDefault{True}
\defValue{If true, then the observations observed values are printed}

\defSub{fits} {Print the observations expected values (fits)?}
\defType{Logical}
\defDefault{True}
\defValue{If true, then the observations expected values are printed}

\defSub{residuals} {Print the observations residual values?}
\defType{Logical}
\defDefault{True}
\defValue{If true, then the observations residuals are printed}

\defSub{error} {Print the observations error values?}
\defType{Logical}
\defDefault{True}
\defValue{If true, then the observations error values are printed (e.g., c.v's,. Ns, etc.,)}

\defSub{likelihood} {Print the observations likelihood values?}
\defType{Logical}
\defDefault{False}
\defValue{If true, then the individual contribution of each observations to the objective function is printed}

\defSub{file\_name} {Define the name of the output file where the report is written}
\defType{String}
\defDefault{None}
\defValue{A valid file name. If not supplied, then output is directed to the standard out}

\subsubsection[Print a layer]{\commandlabsubarg{report}{type}{layer}}

\defSub{layer} {Define the label of the layer to print}
\defType{String}
\defDefault{None}
\defValue{A valid label from \command{Layer}}

\defSub{year} {Define the year for the printing of the layer}
\defType{Integer}
\defDefault{None}
\defValue{A positive integer between \commandsub{model}{initial\_year} and \commandsub{model}{current\_year}}

\defSub{time\_step} {Define the time-step for the printing of the layer}
\defType{Integer}
\defDefault{None}
\defValue{A valid time-step}

\defSub{file\_name} {Define the name of the output file where the report is written}
\defType{String}
\defDefault{None}
\defValue{A valid file name. If not supplied, then output is directed to the standard out}

\subsubsection[Print a selectivity]{\commandlabsubarg{report}{type}{selectivity}}

\defSub{selectivity} {Define the label of the selectivity to print}
\defType{String}
\defDefault{None}
\defValue{A valid label from \command{Selectivity}}

\defSub{year} {Define the year for the printing of the selectivity}
\defType{Integer}
\defDefault{None}
\defValue{A positive integer between \commandsub{model}{initial\_year} and \commandsub{model}{current\_year}}

\defSub{time\_step} {Define the time-step for the printing of the selectivity}
\defType{Integer}
\defDefault{None}
\defValue{A valid time-step}

\defSub{file\_name} {Define the name of the output file where the report is written}
\defType{String}
\defDefault{None}
\defValue{A valid file name. If not supplied, then output is directed to the standard out}
