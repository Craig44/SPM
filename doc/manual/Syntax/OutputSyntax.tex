\section{Reporting command and subcommand syntax\label{sec:output-syntax}}

\subsection{Reports}

The report types available are,

\begin{description}
  \item Partition printouts
\end{description}

Each type of report requires a set of subcommands and arguments specific to that report.

\defComLab{report}{Define an output report}

\defSub{type} {Define the type of report}
\defType{String}
\defDefault{None}
\defValue{A valid type of report}

\subsubsection[Printing the partition]{\commandlabsubarg{report}{type}{partition}}

\defSub{year} {Define the year that the partition report applies to}
\defType{Integer}
\defDefault{None}
\defValue{A positive integer between \commandsub{model}{initial\_year} and \commandsub{model}{current\_year}}

\defSub{time\_step} {Define the time-step that the partition report applies to}
\defType{Integer}
\defDefault{None}
\defValue{A valid time-step}

\defSub{file\_name} {Define the name of the output file where the report is written}
\defType{String}
\defDefault{None}
\defValue{A valid file name}

\subsubsection[Printing the free parameters]{\commandlabsubarg{report}{type}{Parameters}\label{sec:InputFileFormat}}

\defSub{file\_name} {Define the name of the output file where the report is written}
\defType{String}
\defDefault{None}
\defValue{A valid file name}

