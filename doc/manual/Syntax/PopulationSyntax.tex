\section{Population section command and subcommand syntax\label{sec:population-syntax}}

\subsection{Model structure}

\defCom{ModelStructure}{Define the spatial and population model structure, annual cycle and model years}

\defSub{CellShape}{The shape of the cells in the model}
\defType{String}
\defDefault{Square}
\defValue{Either \argument{Square} or \argument{Hexagon}}

\defSub{CellDistance}{The distance, $\lambda$, of one side of a cell}
\defType{Constant}
\defDefault{1}
\defValue{A positive real number , i.e., $\lambda > 0$}

\defSub{Nrows}{The number of rows $n_{rows}$ in the spatial structure}
\defType{integer}
\defDefault{None}
\defValue{A positive integer, $n_{rows} > 0$}

\defSub{Ncols}{The number of columns $n_{cols}$ in the spatial structure}
\defType{integer}
\defDefault{None}
\defValue{A positive integer, $n_{cols} > 0$}

\defSub{Layer}{The label for the base layer}
\defType{string}
\defDefault{None}
\defValue{Must be a label of a layer defined by \command{Layer}}

\defSub{Categories} {Labels of the categories (rows) of the population component of the partition}
\defType{Vector of strings, of length $1\ldots n_{categories}$}
\defDefault{None}
\defValue{Names of categories must be unique}

\defSub{MinAge}{Minimum age of the population}
\defType{Integer}
\defDefault{None}
\defValue{A non-negative integer, ${Age}_{min}\geq 0$ and ${Age}_{min}\leq {Age}_{max}$}

\defSub{MaxAge}{Maximum age of the population}
\defType{Integer}
\defDefault{None}
\defValue{A non-negative integer, ${Age}_{max}\geq 0$ and ${Age}_{min}\geq {Age}_{min}$}

\defSub{PlusGroup}{Define the largest age/size as a plus group}
\defType{switch}
\defDefault{True}
\defValue{Defines  the largest age/size as a plus group}

\defSub{InitialisationPhases}{Define the labels of the phases of the initialisation}
\defType{Vector of strings, of length of the number of initialisation phases}
\defDefault{None}
\defValue{A valid label defined by \command{InitialisationPhases}}

\defSub{InitialYear}{Define the first year of the model, immediately following initialisation}
\defType{Integer}
\defDefault{None}
\defValue{Defines the first year of the model, $\geq 1$, e.g. 1990}

\defSub{CurrentYear}{Define the current year of the model}
\defType{Integer}
\defDefault{None}
\defValue{Defines the current year of the model, i.e., the model is run from \commandsub{ModelStructure}{.FirstYear}\ to \commandsub{ModelStructure}{.CurrentYear}}

\defSub{FinalYear}{Define the final year of the model in projections}
\defType{Integer}
\defDefault{None}
\defValue{Defines the final year of the model for use in projections, i.e., the model is run from \commandsub{ModelStructure}{.FirstYear} to \commandsub{ModelStructure}{.CurrentYear}, then projected to \commandsub{ModelStructure}{.FinalYear}}

\defSub{TimeSteps} {Define the \command{TimeStep} labels (in order that they are applied) to form the annual cycle}
\defType{String vector}
\defDefault{None}
\defValue{Defines the labels of the time steps that are run in each year}

\defComLab{InitialisationPhase}{Define the processes and years of the initialisation phase with label}

\defSub{Years} {Define the number of years to run}
\defType{Integer}
\defDefault{None}
\defValue{A non-negative integer}

\defSub{Processes} {Define the processes (in order of occurrence) to run in each year of the initialisation}
\defType{String vector}
\defDefault{None}
\defValue{A valid process label}

\defComLab{TimeStep} {Define a time step with label}

\defSub{Process} {Define the process labels, in the order that they are applied, for the time step}
\defType{String vector}
\defDefault{None}
\defValue{Defines the labels of the processes for that time step}

\subsection{Processes}

There are a number of processes that are defined by \SPM, including 

\begin{itemize}
	\item Ageing process
	\item Constant recruitment process
	\item Beverton-Holt stock-recruit relationship recruitment process
	\item Mortality rate process
	\item Mortality event process
	\item Category transition process
	\item Category shift process
	\item Preference movement process
\end{itemize}

Each of the different types of process require different subcommands and arguments.

\defComLab{AgeingProcess} {Define an ageing process with label}

\defSub{Categories} {Define the categories that ageing is applied to}
\defType{String vector}
\defDefault{None}
\defValue{A valid category from \commandsub{ModelStructure}{.Categories}}

\defComLab{ConstantRecruitmentProcess} {Define a constant recruitment process}

\defSub{R0} {Define the total amount of recruitment at equilibrium abundance levels}
\defType{Estimable}
\defDefault{None}
\defValue{Total amount (in numbers) of recruitment applied across all categories at equilibrium abundances}

\defSub{Categories} {Define the categories into which recruitment occurs}
\defType{String vector}
\defDefault{None}
\defValue{Must be a valid category from \commandsub{ModelStructure}{.Categories}}

\defSub{Proportions} {Define the proportion of recruitment that occurs into each category}
\defType{Estimable vector of length \commandsub{ConstantRecruitment}{[label].categories}}
\defDefault{None}
\defValue{Proportion of the annual recruitment that is applied to each category}

\defSub{Ages} {Define the ages within each category that receive recruitment}
\defType{Integer}
\defDefault{None}
\defValue{The age classes that receive recruitment}

\defSub{Layer} {Name of the layer used to determine where recruitment occurs}
\defType{String}
\defDefault{None}
\defValue{A valid layer as defined by \command{Layer}}

\defSub{LayerMethod} {Method by which the layer values determine where recruitment occurs}
\defType{String}
\defDefault{None}
\defValue{Must be one of uniform or uniform-by-area. If \argument{Uniform}, then defined as uniformly distributed over cells in the layer where the values of the layer lie between \subcommand{LayerMin}\ and \subcommand{LayerMax}\ values inclusively. If \argument{uniform-by-area}, then defined as uniformly distributed in proportion to the cell area (from the base layer) where the values of the layer lie between  \subcommand{LayerMin}\ and  \subcommand{LayerMax}\ values inclusively}

\defSub{LayerMin} {Minimum value of the layer where recruitment occurs}
\defType{Constant}
\defDefault{None}
\defValue{...}

\defSub{LayerMax} {Maximum value of the layer where recruitment occurs}
\defType{Constant}
\defDefault{None}
\defValue{...}

\defComLab{BHRecruitmentProcess}{Define a Beverton-Holt recruitment process}

\defSub{R0} {Define the total amount of recruitment at equilibrium abundance levels}
\defType{Estimable}
\defDefault{None}
\defValue{Total amount (in numbers) of recruitment applied across all categories at equilibrium abundances}

\defSub{Categories} {Define the categories into which recruitment occurs}
\defType{String vector}
\defDefault{None}
\defValue{Must be a valid category from \commandsub{ModelStructure}{.Categories}}

\defSub{Proportions} {Define the proportion of recruitment that occurs into each category}
\defType{Estimable vector of length \commandsub{Process}{[label].Categories}}
\defDefault{None}
\defValue{Proportion of the annual recruitment that is applied to each category}

\defSub{Ages} {Define the age within each category that receive recruitment}
\defType{Integer}
\defDefault{None}
\defValue{The age classes that receive recruitment}

\defSub{Steepness} {Define the Beverton-Holt stock recruitment relationship steepness ($h$) parameter}
\defType{Estimable}
\defDefault{None}
\defValue{Steepness value between 0.2 and 1.0}

\defSub{SigmaR} {Define the recruitment variability $\sigma_R$ in the stock-recruitment relationship}
\defType{Estimable}
\defDefault{None}
\defValue{...}

\defSub{Rho} {Define the autocorrelation $\rho$ in the recruitment variability in the stock-recruitment relationship}
\defType{Estimable}
\defDefault{None}
\defValue{...}

\defSub{SSB} {Define the label of the \command{DerivedParameter} that defines the SSB}
\defType{String}
\defDefault{None}
\defValue{Must be a valid \command{DerivedParameter} label}

\defSub{YCS-Values} {YCS values}
\defType{Estimable vector}
\defDefault{None}
\defValue{Must be vector of length equal to \commandsub{BHRecruitmentProcess}{[label].YCS-Years}}
\defNote{Special values can be used here, i.e., mean, all}

\defSub{YCS-Years} {Years for YCS values}
\defType{Integer vector}
\defDefault{None}
\defValue{Must be vector of that specifies the years of \commandsub{BHRecruitmentProcess}{[label].YCS-Values}}
\defNote{Special year ranges (YYYY-YYYY) can be used}

\defSub{Layer} {Name of the layer used to determine where recruitment occurs}
\defType{String}
\defDefault{None}
\defValue{A valid layer as defined by \command{Layer}}

\defSub{LayerMethod} {Method by which the layer values determine where recruitment occurs}
\defType{String}
\defDefault{None}
\defValue{Must be one of uniform or uniform-by-area. If \argument{Uniform}, then defined as uniformly distributed over cells in the layer where the values of the layer lie between \subcommand{LayerMin}\ and \subcommand{LayerMax}\ values inclusively. If \argument{uniform-by-area}, then defined as uniformly distributed in proportion to the cell area (from the base layer) where the values of the layer lie between  \subcommand{LayerMin}\ and  \subcommand{LayerMax}\ values inclusively}

\defSub{LayerMin} {Minimum value of the layer where recruitment occurs}
\defType{Constant}
\defDefault{None}
\defValue{...}

\defSub{LayerMax} {Maximum value of the layer where recruitment occurs}
\defType{Constant}
\defDefault{None}
\defValue{...}

\defComLab{MortalityRateProcess} {Define a mortality rate process with label}

\defSub{Categories} {Define the categories that ageing is applied to}
\defType{String vector}
\defDefault{None}
\defValue{A valid category from \commandsub{ModelStructure}{.Categories}}

\defComLab{MortalityEventProcess} {Define a mortality event process with label}

\defSub{Categories} {Define the categories that ageing is applied to}
\defType{String vector}
\defDefault{None}
\defValue{A valid category from \commandsub{ModelStructure}{.Categories}}

\defComLab{CategoryTransitionProcess} {Define a category transition process with label}

\defSub{Categories} {Define the categories that ageing is applied to}
\defType{String vector}
\defDefault{None}
\defValue{A valid category from \commandsub{ModelStructure}{.Categories}}

\defComLab{CategoryMoveProcess} {Define a category shift process with label}

\defSub{Categories} {Define the categories that ageing is applied to}
\defType{String vector}
\defDefault{None}
\defValue{A valid category from \commandsub{ModelStructure}{.Categories}}

\defComLab{PreferenceMovementProcess} {Define a preference function movement process with label}

\defSub{Categories} {Define the categories that ageing is applied to}
\defType{String vector}
\defDefault{None}
\defValue{A valid category from \commandsub{ModelStructure}{.Categories}}

\subsection{Preference functions}

\defComLab{PreferenceFunction} {Define a preference function with label}

\subsection{Layers}

\defComLab{Layer} {Define a layer with label}

\defComLab{MetaLayer} {Define a meta-layer with label}

\subsection{Derived quantities}

\defComLab{DerivedQuantity} {Define a derived quantity with label}

\subsection{Growth}

\subsection{Selectivities}

\defComLab{ConstantSelectivity} {Define a constant selectivity with label}

\defComLab{VaryingSelectivity} {Define an annually varying selectivity with label}


