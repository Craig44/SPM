\section{Population section command and subcommand syntax\label{sec:population-syntax}}

\subsection{Spatial structure}

\defCom{spatial\_structure}{Define the spatial partition structure}

\defSub{cell\_shape}{The shape of the cells in the model}
\defType{String}
\defDefault{Square}
\defEffect{Defines the shape of cells in the spatial grid, either \subcommand{Square}{}, \subcommand{hexagonal}{}, or \subcommand{Octagonal}{}}

\defSub{nrows}{The number of rows $n_{rows}$ in the spatial structure}
\defType{integer}
\defDefault{None}
\defEffect{Defines the number of rows $n_{rows}$ for the spatial structure}

\defSub{ncols}{The number of columns $n_{cols}$ in the spatial structure}
\defType{integer}
\defDefault{None}
\defEffect{Defines the number of columns $n_{cols}$ for the spatial structure}

\defSub{layer}{The label for the base layer}
\defType{string}
\defDefault{None}
\defEffect{Defines the label for the base layer. Must be one of the layers defined by  \command{layer}{}}

\subsection{Population structure}

\defCom{population\_structure} {Define the population partition structure}

\defSub{type} {Define the population partition as either age, size, or stage structured}
\defType{string}
\defDefault{age}
\defEffect{Defines the nature of the population partition columns (i.e., age, size, or stage structured)}
\defCondition{Size and stage population partitions are not yet implemented}

\defSub{categories} {Labels of the categories (rows) of the population partition}
\defType{Vector of strings, of length $1\ldots n_{categories}$}
\defDefault{None}
\defEffect{Defines the labels of the categories of the population partition}
\defCondition{Must be unique}

\defSub{min\_age}{Minimum age of the population}
\defType{Integer}
\defDefault{None}
\defEffect{Defines the minimum age of the population partition}
\defCondition{Must be $geq 0$}

\defSub{max\_age}{Maximum age of the population}
\defType{Integer}
\defDefault{None}
\defEffect{Defines the maximum age of the population partition}
\defCondition{Must be $geq 0$ and $geq min_{age}$}

\defSub{plus\_group}{Define the largest age/size as a plus group}
\defType{switch}
\defDefault{True}
\defEffect{Defines  the largest age/size as a plus group}

\subsection{Initialization}

\defCom{initialization}{Define the initialization method and phases of the initialization}

\defSub{initialization\_phases}{Defines the labels of the phases of the initialization}
\defType{Vector of strings, of length of the number of initialization phases}
\defDefault{None}
\defEffect{Defines the phases, labels of the phases and the order of the phases to run for the initialization}

\defSub{initial\_year}{Define the first year of the model, immediately following initialization}
\defType{Integer}
\defDefault{None}
\defEffect{Defines the first year of the model}

\defSub{current\_year}{Define the current year of the model}
\defType{Integer}
\defDefault{None}
\defEffect{Defines the current year of the model, i.e., the model is run from \command{initialization}{first\_year} to \command{initialization}{current\_year}}

\defSub{final\_year}{Define the final year of the model in projections}
\defType{Integer}
\defDefault{None}
\defEffect{Defines the final year of the model for use in projections, i.e., the model is run from \command{initialization}{first\_year} to \command{initialization}{current\_year}, then projected to \command{initialization}{final\_year}}

\defComLab{initialization\_phase}{Define the processes and years of the initialization phase with label}

\defSub{method} {Define the method of initialization}
\defType{String}
\defDefault{Iterative}
\defEffect{Defines the method of initialization in that phase. Currently only iterative is defined}

\defSub{years} {Define the number of years to run}
\defType{Integer}
\defDefault{None}
\defEffect{Defines the number of years in this phase of the initialization over which the model runs}

\defSub{processes} {Define the processes (in order of occurrence) to run in each year of the initialization}
\defType{String vector}
\defDefault{None}
\defEffect{Defines the labels of the processes that are run in each year during that phase of initialization}

\subsection{Annual cycle and time steps}

\defComLab{annual\_cycle}{Define the labels of the time steps that run in each year of the model}

\defSub{time\_steps} {Define the \subcommand{time\_step} labels (in order that they are applied) to form the annual cycle}
\defType{String vector}
\defDefault{None}
\defEffect{Defines the labels of the time steps that are run in each year}

\defComLab{time\_steps} {Define a time step with label}

\defSub{process} {Define the process labels, in the order that they are applied, for the time step}
\defType{String vector}
\defDefault{None}
\defEffect{Defines the labels of the processes for that time step}

\subsection{Processes}

\defComLab{process} {Define a process with label}

\defSub{type} {Define the type of process}
\defType{String}
\defDefault{None}
\defEffect{Defines the type of process applied by \command{process}[\subcommand{label}]. Must be a valid process type. There are a number of processes that are defined by \SPM, including 
\begin{description}
\item ageing
\item Beverton-Holt\_recruitment
\item constant\_recruitment
\item mortality\_rate
\item event\_mortality
\item category\_transition
\item category\_transition\_rate
\item directed\_movement
\end{description}
Each of the different types of process requires a different set of subcommands.}

If \subcommand{type=ageing},

\defSub{categories} {Define the categories that ageing is applied to}
\defType{String vector}
\defDefault{None}
\defEffect{Ages the named categories}
 
If \subcommand{type=Beverton-Holt\_recruitment},

\defSub{categories} {Define the categories into which recruitment occurs}
\defType{String vector}
\defDefault{None}
\defEffect{Categories that have recruitment}

\defSub{proportions} {Define the proportion of recruitment that occurs into each category}
\defType{Estimable vector of length \command{process}\subcommand{[label].categories}}
\defDefault{None}
\defEffect{Proportion of the annual recruitment that is applied to each category}

\defSub{R0} {Define the total amount of recruitment at equilibrium abundance levels}
\defType{Estimable}
\defDefault{None}
\defEffect{Total amount (in numbers) of recruitment applied across all categories at equilibrium abundances}

\defSub{ages} {Define the ages within each category that receive recruitment}
\defType{Integer}
\defDefault{None}
\defEffect{The age classes that receive recruitment}

\defSub{steepness} {Define the Beverton-Holt stock recruitment relationship steepness ($h$) parameter}
\defType{Estimable}
\defDefault{None}
\defEffect{The steepness parameter in the Beverton-Holt stock-recruit relationship}

\defSub{sigma\_r} {Define the recruitment variability in the stock-recruitment relationship applied in projections}
\defType{Estimable}
\defDefault{None}
\defEffect{...}

\defSub{rho} {Define the autocorrelation in the recruitment variability in the stock-recruitment relationship applied in projections}
\defType{Estimable}
\defDefault{None}
\defEffect{...}

\defSub{layer} {Name of the layer used to determine where recruitment occurs}
\defType{String}
\defDefault{None}
\defEffect{...}

\defSub{layer\_method} {Method by which the layer values determine where recruitment occurs}
\defType{String}
\defDefault{None}
\defEffect{Must be one of uniform or uniform-by-area. If Uniform, then defined as uniformly distributed over cells in the layer where the values of the layer lie between layer\_min and layer\_max values inclusively. If uniform-by-area, then defined as uniformly distributed in proprotion to the cell area (from the base layer) where the values of the layer lie between layer\_min and layer\_max values inclusively}

\defSub{layer\_max} {Maximum value of the layer where recruitment occurs}
\defType{Constant}
\defDefault{None}
\defEffect{...}

\defSub{layer\_min} {Minimum value of the layer where recruitment occurs}
\defType{Constant}
\defDefault{None}
\defEffect{...}

If \subcommand{type=constant\_recruitment},

\defSub{categories} {Define the categories into which recruitment occurs}
\defType{String vector}
\defDefault{None}
\defEffect{Categories that have recruitment}

\defSub{proportions} {Define the proportion of recruitment that occurs into each category}
\defType{Estimable vector of length \command{process}\subcommand{[label].categories}}
\defDefault{None}
\defEffect{Proportion of the annual recruitment that is applied to each category}

\defSub{R0} {Define the total amount of recruitment at equilibrium abundance levels}
\defType{Estimable}
\defDefault{None}
\defEffect{Total amount (in numbers) of recruitment applied across all categories at equilibrium abundances}

\defSub{ages} {Define the ages within each category that receive recruitment}
\defType{Integer}
\defDefault{None}
\defEffect{The age classes that receive recruitment}

\defSub{layer} {Name of the layer used to determine where recruitment occurs}
\defType{String}
\defDefault{None}
\defEffect{...}

\defSub{layer\_method} {Method by which the layer values determine where recruitment occurs}
\defType{String}
\defDefault{None}
\defEffect{Must be one of uniform or uniform-by-area. If Uniform, then defined as uniformly distributed over cells in the layer where the values of the layer lie between layer\_min and layer\_max values inclusively. If uniform-by-area, then defined as uniformly distributed in proprotion to the cell area (from the base layer) where the values of the layer lie between layer\_min and layer\_max values inclusively}

\defSub{layer\_max} {Maximum value of the layer where recruitment occurs}
\defType{Constant}
\defDefault{None}
\defEffect{...}

\defSub{layer\_min} {Minimum value of the layer where recruitment occurs}
\defType{Constant}
\defDefault{None}
\defEffect{...}

\defSub{categories} {Define the categories into which recruitment occurs}
\defType{String vector}
\defDefault{None}
\defEffect{Categories that have recruitment}

\subsection{Layers}

\subsection{Derived quantities}

\subsection{Growth}

\subsection{Selectivities}

\subsection{Annually varying parameters}
