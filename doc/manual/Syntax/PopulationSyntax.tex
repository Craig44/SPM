\subsection{Population commands and subcommands\label{sec:population-syntax}}

\subsubsection{Spatial structure}

\defCom{spatial\_structure}{Defines the spatial partition structure}

\defSub{nrows}{The number of rows $n_{rows}$ in the spatial structure}
\defType{integer}
\defDefault{None}
\defEffect{Defines the number of rows $n_{rows}$ for the spatial structure}

\defSub{ncols}{The number of columns $n_{cols}$ in the spatial structure}
\defType{integer}
\defDefault{None}
\defEffect{Defines the number of columns $n_{cols}$ for the spatial structure}

\defSub{layer}{The label for the base layer}
\defType{string}
\defDefault{None}
\defEffect{Defines the label for the base layer. Must be one of the layers defined by  \command{layer}{}}

\subsubsection{Population structure}

\defCom{population\_structure} {Define the population partition structure}

\defSub{type} {Defines the population partition as either age, size, or stage structured}
\defType{string}
\defDefault{age}
\defEffect{Defines the nature of te population partition columns (i.e., age, size, or stage structured)}
\defCondition{Size and stage population partitions are not yet implemented}

\defSub{categories} {Labels of the categories (rows) of the population partition}
\defType{Vector of strings, of length $1\ldots n_{categories}$}
\defDefault{None}
\defEffect{Defines the labels of the categories of the population partition}
\defCondition{Must be unique}

\defSub{min\_age}{Minimum age of the population}
\defType{Integer}
\defDefault{None}
\defEffect{Defines the minimum age of the population partition}
\defCondition{Must be $geq 0$}

\defSub{max\_age}{Maximum age of the population}
\defType{Integer}
\defDefault{None}
\defEffect{Defines the maximum age of the population partition}
\defCondition{Must be $geq 0$ and $geq min_{age}$}

\defSub{plus\_group}{Define the largest age/size as a plus group}
\defType{switch}
\defDefault{True}
\defEffect{Defines  the largest age/size as a plus group}

\subsubsection{Initialisation}

\defCom{initialisation}{Define the initialisation method and phases of the initialisation}

\defSub{method} {Defines the method of initialisation}
\defType{String}
\defDefault{Iterative}
\defEffect{Defines the that iniialisation is determined iteratively}

\defSub{initialisation\_phases}{Defines the labels of the phases of the initialisation}
\defType{Vector of strings, of length of the number of initialisation phases}
\defDefault{None}
\defEffect{Defines the phases, labels of the phases and the order of the phases to run for the initialisation}

\defSub{first\_year}{Define the first year of the model, immediately following  initialisation}
\defType{Integer}
\defDefault{None}
\defEffect{Defines the first year of the model}

\defSub{current\_year}{Define the current year of the model}
\defType{Integer}
\defDefault{None}
\defEffect{Defines the current year of the model, i.e., the model is run from \command{initialisation}{first\_year} to \command{initialisation}{current\_year}}

\defSub{final\_year} {Define the final year of the model in projections}
\defType{Integer}
\defDefault{None}
\defEffect{Defines the final year of the model for use in projections, i.e., the model is run from \command{initialisation}{first\_year} to \command{initialisation}{current\_year}, then projected to \command{initialisation}{final\_year}}

\defComLab{initialisation\_phase}{Define the initialisation phases with label \texttt{label}}


\subsubsection{Annual cycle}

\subsubsection{Time steps}

\subsubsection{Population processes}

\subsubsection{Movement processes}

\subsubsection{Layers}

\subsubsection{Derived quantities}

\subsubsection{Growth}

\subsubsection{Selectivities}


