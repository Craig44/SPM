\section{Running \SPM\label{sec:running-spm}\index{Running \SPM}}

\SPM\ is run from the console window (i.e., the DOS command line) on \I{Microsoft Windows} or from a terminal window on \I{Linux}. \SPM\ gets its information from input data files, the key one of which is the \config\index{Input configuration file}. 

The \config\ is compulsory and defines the model structure, processes, observations, parameters (both the fixed parameters and the parameters to be estimated)\index{Estimable parameters}\index{Estimated parameters}, and the reports (outputs) requested. The following sections  describe how to construct the \SPM\ configuration file. By convention, the name of the \config\ ends with the suffix \texttt{.spm}, however, any file name is acceptable.

Other input files can, in some circumstances, be supplied to define the starting point for an estimation, define the parameters for a projection, or to simulate observations.

Simple command line arguments\index{Command line arguments} are used to determine the actions or \emph{tasks}\index{Tasks} of \SPM, i.e., to run a model with a set of parameter values, estimate parameter values (either point estimates or MCMC), project quantities into the future, simulate observations, etc,. Hence, the \emph{command line arguments} define the \emph{task}. For example, \texttt{-r} is the \emph{run}, \texttt{-e} is the \emph{estimation}, and \texttt{-m} is the \emph{MCMC} task. The \emph{command line arguments} are described in Section \ref{sec:command-line-arguments}.

\subsection{\I{Using \SPM}}

To use \SPM, open a console (i.e., the command prompt) window (Microsoft Windows) or a terminal window (Linux). Navigate to a directory of your choice, where your \config s are located. Then type \texttt{spm} with any arguments (see Section \ref{sec:command-line-arguments} for the the list of possible arguments). SPM will print output to the screen and return you to the command prompt when it completes its task. Note that the \SPM\ executable (binary) must be either in the directory where you run it or somewhere in your \texttt{PATH}. Note that an automated installer is available for \SPM\ on Microsoft Windows. If you use the installer, then it will give you the option of modifying your \texttt{PATH} for you (as well a a number of other options to make using the program a little easier). Otherwise, see your operating system documentation for help on identifying or modifying your \texttt{PATH}.

\subsection{The \config\label{sec:config-files}}\index{Input configuration file}

The \config\ is made up of four broad sections; the description of the population structure and parameters (the population section), the estimation methods and variables (the estimation section), the observations and their associated likelihoods (the observation section), and the outputs and reports that \SPM\ will return (the report section). The \config\ is made up of a number of commands (many with subcommands) which specify various options for each of these components.

The command and subcommand definitions in the \config\ can be extensive, and can result in a \config\ that is long and difficult to navigate. To aid readability and flexibility, we can use the \config\ command \command{include} \texttt{\emph{file}}. The command causes an external file, \argument{\emph{file}}, to be read and processed, exactly as if its contents had been inserted in the main \config\ at that point\index{Including external files}. The file name must be a complete file name with extension, but can use either a relative or absolute path as part of its name. Note that included files can also contain \command{include} commands --- but be careful that you do not set up a recursive state. See Section \ref{sec:general-syntax} for more detail.

\subsection{\I{Redirecting standard output}\label{sec:redirecting-stdout}}

\SPM\ uses the standard out\index{Standard out} to display run-time information. \I{Standard error} is not used by \SPM, but may be used by the operating system to report an error with \SPM. We suggest redirecting both the standard out and standard error into files\index{Redirecting standard out}\index{Redirecting standard error}. With the bash shell (on Linux systems), you can do this using the command structure,

\begin{verbatim} (spm [arguments] > out) >& err &\end{verbatim}

It may also be useful to redirect the standard input, especially is you're using \SPM\ inside a batch job software, i.e. 

\begin{verbatim} (spm [arguments] > out < /dev/null) >& err &\end{verbatim}

On Microsoft Windows systems, you can redirect to standard output using,

\begin{verbatim} spm [arguments] > out\end{verbatim}

And, on some Microsoft Windows systems (e.g., the Professional version of WindowsXP), you can redirect to both standard output and standard error, using the syntax, 

\begin{verbatim} spm [arguments] > out 2> err\end{verbatim}

Note that \SPM\ outputs a few lines of header information to the output. The header\index{Output header information} consists of the program name and version, the arguments passed to \SPM\ from the command line, the date and time that the program was called (derived from the system time), the user name, and the machine name (including the operating system and the process identification number). These can be used to track outputs as well as identifying the version of \SPM\ used to run the model.

\subsection{\I{Command line arguments}\label{sec:command-line-arguments}}

The call to \SPM\ is of the following form.: 

\texttt{spm [-c \emph{config\_file}] [\emph{task}] [\emph{options}]}

\begin{description}
  \item [\texttt{-c \emph{config\_file}}] Define the \config\ for \SPM. If omitted, then \SPM\ looks for a file named \texttt{config.spm}.
\end{description}

and where \emph{task} is one of;
\begin{description}
\item [\texttt{-h}]  Display help (this page).
\item [\texttt{-l}] Display the reference for the software license (CPLv1.0).
\item [\texttt{-v}] Display the \SPM\ version number.

\item [\texttt{-r}] \emph{Run} the model once using the parameter values in the \config, or optionally, with the values from the file denoted with the command line argument \texttt{-i \emph{file}}.

\item [\texttt{-e}] Do a point \emph{estimate} using the values in the \config\ as the starting point for the parameters to be estimated, or optionally, with the start values from the file denoted with the command line argument \texttt{-i \emph{file}}.

\item [\texttt{-p}] Do a likelihood \emph{profile} using the parameter values in the \config\ as the starting point, or optionally, with the start values from the file denoted with the command line argument \texttt{-i \emph{file}}.

\item [\texttt{-m}] Do an \emph{MCMC} estimate using the values in the \config\ as the starting point for the parameters to be estimated, or optionally, with the start values from the file denoted with the command line argument \texttt{-i \emph{file}}. \NYI.

\item [\texttt{-f}] Project the model \emph{forward} in time using the parameter values in the \config\ as the starting point for the estimation, or optionally, with the start values from the file denoted with the command line argument \texttt{-i \emph{file}}. \NYI.

\item [\texttt{-s}] \emph{Simulate} observations using values in the \config\ as the parameter values, or optionally, with the values for the parameters denoted as estimated from the file with the command line argument \texttt{-i \emph{file}}.

\end{description}

In addition, the following are optional arguments\index{Optional command line arguments} [\emph{options}],

\begin{description}
\item [\texttt{-i \emph{file}}] \emph{Input} one or more sets of estimated parameter values from \texttt{\emph{file}}. See Section \ref{sec:InputFileFormat} for details about the format of \texttt{\emph{file}}.

\item [\texttt{-t \emph{number}}] Number of \emph{threads} to run (i.e., number of processors available for use). \NYI.

\item [\texttt{-q}] Run \emph{quietly}, i.e., suppress verbose printing of \SPM.

\item [\texttt{-g \emph{seed}}]  Seed the random number \emph{generator} with \texttt{\emph{seed}}, a positive (long) integer value. Note, if \texttt{-g} is not specified, then \SPM\ looks the command \commandsub{estimation}{random\_seed} for a random number seed , and if not defined, then automatically generates a random number seed based on the computer clock time.
\end{description}

\subsection{Constructing an \SPM\ \config \label{constructing-spm-config}}\index{Input configuration file syntax}

The model definition, parameters, observations, and reports are specified in an \config. The  population section is described in Section \ref{sec:population-section} and the population commands in Section \ref{sec:population-syntax}. Similarly, the estimation section is described in Section \ref{sec:estimation-section} and its commands in Section \ref{sec:estimation-syntax}, and in Section \ref{sec:report-section} and Section \ref{sec:report-syntax} for the report and report commands. 

\subsubsection{Commands}\index{Commands}

\SPM\ has a range of commands that define the model structure, processes, observations, and how tasks are carried out. There are three types of commands, 

\begin{enumerate}
\item Commands that have an argument and do not have subcommands (for example, \command{include}\ \argument{\emph{file}})
\item Commands that have a label and subcommands (for example \command{process})
\item Commands that do not have either a label or argument, but have subcommands (for example \command{model})
\end{enumerate}

Commands that have a label must have a unique label, i.e., the label cannot be used on more than one command of that type. The labels must start with a letter or underscore, can contain letters, underscores, or numbers, but must not contain white-space or a full-point ('.').

\subsubsection{Subcommands}\index{Commands ! Subcommands}

Subcommands in \SPM\ are for defining options and parameter values for commands. They always take an argument which is one of a specific \emph{type}. The types acceptable for each subcommand are defined in Section \ref{sec:syntax}, and are summarised below. 

Like commands (\command{command}), subcommands and their arguments are not order specific --- except that that all subcommands of a given command must appear before the next \command{command} block. \SPM\ may report an error if they are not supplied in this way, however, in some circumstances a different order may result in a valid, but unintended set of actions, leading to possible errors in your expected results.  

The arguments for a subcommand are either\index{Subcommand argument type},

\begin{description}
\item \textbf{switch} true/false 
\item \textbf{integer} an integer number
\item \textbf{integer vector} a vector of integer numbers
\item \textbf{constant} a real number (i.e., double)
\item \textbf{constant vector} a vector of real numbers (i.e., vector of doubles)
\item \textbf{estimable} a real number that can be estimated (i.e., estimable double)
\item \textbf{estimable vector} a vector of real numbers that can be estimated (i.e., vector of estimable doubles)
\item \textbf{string} a categorical (string) value
\item \textbf{string vector} a vector of categorical values
\end{description}

Switches are parameters which are either true or false. Enter \emph{true} as \argument{true} or \argument{t}, and \emph{false} as \argument{false} or \argument{f}. 

Integers must be entered as integers (i.e., if \subcommand{year}\ is an integer then use 2008, not 2008.0)

Arguments of type integer vector, constant vector, estimable vector, or categorical vector contain one or more entries on a row, separated by white space (tabs or spaces). 

\emph{Estimable} parameters are those parameters that \SPM\ can estimate, if requested. If a particular parameter is not being estimated in a particular model run, then it acts as a constant.  Within \SPM\, only estimable parameters can be estimated. And, you have to tell \SPM\ those that are to be estimated in any particular model. Estimable parameters that are being estimated within a particular model run are called the \emph{estimated parameters}.

\subsubsection{The command-block format}\index{Command block format}

Each command-block either consists of a single command (starting with the symbol \@) and, for most commands, a label or an argument. Each command is then followed by its subcommands and their arguments, e.g., 

\begin{description}
\item \command{command}, or 
\item \command{command} \subcommand{argument}, or
\item \command{command} \subcommand{\emph{label}}
\end{description}

and then
\begin{description}
\item \subcommand{subcommand} \subcommand{argument}
\item \subcommand{subcommand} \subcommand{argument}
\item etc,.
\end{description}

Blank lines are ignored, as is extra white space (i.e., tabs and spaces) between arguments. But don't put extra white space before a \command{} character (which must also be the first character on the line), and make sure the file ends with a carriage return. Commands and subcommands consist of letters and/or underscores, must not contain a spaces or full-point ('.').

There is no need to mark the end of a command block. This is automatically recognized by either the end of the file, section, or the start of the next command block (which is marked by the \command{} on the first character of a line). Note, however, that the \command{include} is the only exception to this rule. See Section \ref{sec:general-syntax})\index{Command ! Include files} for details of the use of \command{include}. 

In general, commands, sub-commands, and arguments in the parameter files are case insensitive. But note, however, that if you are on a Linux system then external calls to files are case sensitive (i.e., when using \command{include} \subcommand{\emph{file}}, the argument \subcommand{\emph{file}} will be case sensitive). 

\subsubsection{Commenting out lines}\index{Comments}\index{Commenting \config}

Text that follows a \commentline\ on a line are considered to be comments and are ignored. If you want to remove a group of commands or subcommands using \commentline, then comment out all lines in the block, not just the first line. 

Alternatively, you can comment out an entire block or section by placing curly brackets around the text that you want to comment out. Put in a \commentstart\ as the first character on the line to start the comment block, then end it with \commentend. All lines (including line breaks) between \commentstart\ and \commentend\ inclusive are ignored. (These should ideally be the first character on a line. But if not, then the entire line will be treated as part of the comment block.)

\subsubsection{Determining parameter names\label{sec:parameter-names}\index{Determining parameter names}\index{Parameter names}}

When SPM processes a \config, it translates each command and each subcommand into a parameter with a unique name. For commands, this parameter name is simply the command name. For subcommands, the parameter name format is either, 

\begin{description}
\item \texttt{command[label].subcommand} if the command has a label, or
\item \texttt{command.subcommand} if the command has no label, or
\item \texttt{command[label].subcommand[i]} if the command has a label and the subcommand arguments are a vector, and we are accessing the  \emph{i}th element of that vector. 
\item \texttt{command[label].subcommand[i-j]} if the command has a label, and the subcommand arguments are a vector, and we are accessing the elements from $i$ to $j$ (inclusive) of that vector. \NYI.
\end{description} 

The unique parameter name is used to reference the parameter when estimating, applying a penalty, or applying a profile. For example, the parameter name of subcommand \subcommand{r0} of the command \command{process} with the label \argument{MyRecruitment} is,

\texttt{process[MyRecruitment].r0}

\subsection{\SPM\ exit status values\index{Exit status value}}

When \SPM\ completes its task successfully or errors out gracefully, it returns a single exit status value ($0$) to the operating system. The operating system will return $-1$ if \SPM\ terminates unexpectedly. To determine if \SPM\ has completed its task successfully, check the standard output for error and information messages.
