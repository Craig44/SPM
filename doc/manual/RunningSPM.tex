\section{Running \SPM\label{sec:running-spm}}

\SPM\ gets its information from input data files, the key one of which is the \config\index{Input configuration file}. The \config\ is compulsory and defines the model structure, processes, observations, free and fixed parameters\index{Free parameters}\index{Fixed parameters}, and the outputs requested. Section \ref{sec:getting-started} describes how to construct the \SPM\ configuration file. By convention, the name of the \config\ ends with the suffix \texttt{.spm}, however, any file name is acceptable.

Other input files can, in some circumstances, be supplied to define the starting point for an estimation, define the parameters for a projection, or to simulate observations.  

Simple command line arguments\index{Command line arguments} are used to determine the actions or \emph{tasks}\index{Tasks} of \SPM, i.e., to run a model with a set of parameter values, estimate parameter values (either point estimates or MCMC), project quantities into the future, simulate observations, etc. Hence, the \emph{command line arguments} define the \emph{task}. For example, \texttt{-r} is \emph{run}, \texttt{-e} is \emph{estimation}, and \texttt{-s} is the \emph{simulation} task. The \emph{command line arguments} are described in Section \ref{sec:command-line-arguments}.

\subsection{The \config\label{sec:config-files}}\index{Input configuration file}

The \config\ comprises of four parts (preamble, population, estimation, and output). All of these, except the preamble, are compulsory. \SPM\ will error out if one of the compulsory sections is missing.

\begin{enumerate}
\item \textbf{preamble}: 
the preamble occurs at the start of the \config, before the three sections defined below. \SPM\ ignores any text within this section, and hence the preamble can be used to record comments or descriptions of the \config. 

\item \textbf{population}: 
the population section is defined in the \config\ by the text \texttt{[population]}. This must be the only text on that line. All commands and subcommands that follow the \texttt{[population]} header, until the \texttt{[estimation]} header is reached, are considered to be commands and subcommands the define the population structure of the model. The population section is described in Section \ref{sec:population-section}.

\item \textbf{estimation}:
the estimation section is defined in the \config\ by the text \texttt{[estimation]}. This must be the only text on that line. All commands and subcommands that follow the \texttt{[estimation]} header, until the \texttt{[output]} header is reached, are considered to be commands and subcommands the define the estimation structure of the model.  The estimation section is described in Section \ref{sec:estimation-section}.

\item \textbf{output}:
the output section is defined in the \config\ by the text \texttt{[output]}. This must be the only text on that line. All commands and subcommands that follow the \texttt{[output]} header, until the end of the file is reached, are considered to be commands and subcommands the define the outputs of the model.  The output section is described in Section \ref{sec:output-section}.

\end{enumerate}

The command and subcommand definitions in the \config\ can be extensive, and can result in a \config\ that is long and difficult to navigate. To aid readability and flexibility, we can use the \config\ command \command{IncludeFile} ``\emph{filename}''. The command causes an external file, \emph{filename}, to be read and processed, exactly as if its contents had been inserted in the main \config\ at that point. The file name must be a complete file name with extension, but can have either a relative or absolute path as part of its name. Note that included files can also contain \command{IncludeFile} commands \textemdash\ be careful that you do not set up a recursive state. See Section \ref{sec:general-syntax} for more detail.

\subsection{Redirecting standard output\label{sec:redirecting-stdout}}

\SPM\ uses the standard out to display run-time information. Standard error is not used by \SPM, but may be used by the operating system to report an error with \SPM. We suggest redirecting both the standard out and standard error into files. With the bash shell (on Linux systems), you can do this using the command structure,

\begin{verbatim} (SPM [arguments] > out) >& err &\end{verbatim}

It may also be useful to redirect the standard input, especially is you're using \SPM\ inside a batch job software, i.e. 

\begin{verbatim} (SPM [arguments] > out < /dev/null) >& err &\end{verbatim}

On Microsoft Windows systems, you can redirect to standard output using,

\begin{verbatim} SPM [arguments] > out\end{verbatim}

And, on some recent Microsoft Windows systems (e.g., Professional versions of WindowsNT, Windows2000, and WindowsXP), you can redirect to both standard output and standard error, using the syntax, 

\begin{verbatim} SPM [arguments] > out 2> err\end{verbatim}

Note that \SPM\ outputs a few lines of header information to the output. The header consists of the program name and version, the arguments passed to \SPM\ from the command line, the date and time that the program was called (derived from the system time), the user name, and the machine name (including the operating system and the process identification number). These can be used to track outputs as well as identifying the version of \SPM\ used to run the model.

\subsection{Command line arguments\label{sec:command-line-arguments}}

The call to \SPM\ is of the following form.: 

\texttt{SPM [\emph{task}] [\emph{config\_file}] [\emph{options}]}

where \emph{task} is one of;
\begin{description}
\item \texttt{-h}:{\hspace{0.5cm}  Display help (this page).}

\item \texttt{-l}:{\hspace{0.5cm} Display the reference for the software license (CPLv1.0).}

\item \texttt{-v}:{\hspace{0.5cm} Display the \SPM\ version number.}

\item \texttt{-r \emph{config\_file}}:{\hspace{0.5cm} \emph{Run} the model once using the parameter values in the \config\ denoted by \emph{\texttt{config\_file}}, or optionally, with the values in the file denoted with the command line argument \texttt{-i \emph{file}}.}

\item \texttt{-e \emph{config\_file}}:{\hspace{0.5cm} Do a point \emph{estimate} of the free parameters using the parameter values in the \config\ denoted by \emph{\texttt{config\_file}} as the starting point for the estimation, or optionally, with the start values in the file denoted with the command line argument \texttt{-i \emph{file}}.}

\item \texttt{-p \emph{config\_file}}:{\hspace{0.5cm} Do a likelihood \emph{profile} using the parameter values in the \config\ denoted by \emph{\texttt{config\_file}} as the starting point, or optionally, with the start values in the file denoted with the command line argument \texttt{-i \emph{file}}.}

\item \texttt{-m \emph{config\_file}}:{\hspace{0.5cm} Do an \emph{MCMC} estimate of the free parameters using the parameter values in the \config\ denoted by \emph{\texttt{config\_file}} as the starting point for the estimation, or optionally, with the start values in the file denoted with the command line argument \texttt{-i \emph{file}}.}

\item \texttt{-f \emph{config\_file}}:{\hspace{0.5cm} Project the model \emph{forward} in time using the parameter values in the \config\ denoted by \emph{\texttt{config\_file}} as the starting point for the estimation, or optionally, with the start values in the file denoted with the command line argument \texttt{-i \emph{file}}.}

\item \texttt{-s \emph{config\_file}}:{\hspace{0.5cm} \emph{Simulate} observations of the free parameters using the parameter values in the \config\ denoted by \emph{\texttt{config\_file}} as the starting point for the estimation, or optionally, with the start values in the file denoted with the command line argument \texttt{-i \emph{file}}.}

\end{description}

In addition, the following are optional arguments [\emph{options}],

\begin{description}
\item \texttt{-i \emph{file}}:{\hspace*{0.5cm} \emph{Input} one or more sets of free parameter values from \texttt{\emph{file}}. See Section \ref{sec:InputFileFormat} for details about the format of \texttt{\emph{file}}.}

\item \texttt{-t \emph{number}}:{\hspace*{0.5cm} Number of \emph{threads} to run (i.e., number of processors available for use).}

\item \texttt{-q}:{\hspace*{0.5cm} Run \emph{quietly}, i.e., suppress verbose printing of \SPM.}

\item \texttt{-g \emph{seed}}:{\hspace*{0.5cm} Seed the random number \emph{generator} with \texttt{\emph{seed}}, a positive (long) integer value. Note, if \texttt{-g} is not specified, then \SPM\ first looks to the \texttt{[estimation]} section for a random number seed \command{RandomSeed}, and if not defined, then automatically generates a random number seed based on the computer clock time.}
\end{description}

\subsection{Constructing an \SPM\ \config \label{constructing-spm-config}}\index{Input configuration file syntax}

The model definition, parameters, observations, and output are specified in a \config. The definition of the population section is described in Section \ref{sec:population-section} and the commands and subcommands in Section \ref{sec:population-syntax}. Similarly, the definition of the estimation section is described in Section \ref{sec:estimation-section} and the commands and subcommands in Section \ref{sec:estimation-syntax}, and in Section \ref{sec:output-section} and Section \ref{sec:output-syntax} for the output. 

In general, the \config\ uses a command-block format to describe commands, subcommands, and their arguments and values. The \config\ consists of any number of command-blocks in any order, but these must be within the \texttt{[population]}, \texttt{[estimation]}, and  \texttt{[output]} sections for the population, estimation, and output commands respectively. 

\subsubsection{The command-block format}\index{Command block format}

Each command-block either consists of a single command (starting with the symbol \@) and, for most commands, a label or an argument. Each command is then followed by its subcommands and their arguments, e.g., 

\begin{description}
\item \command{command}, or 
\item \command{command} \subcommand{argument}, or
\item \command{command} \subcommand{\emph{label}}
\end{description}

and then
\begin{description}
\item \subcommand{subcommand} \subcommand{argument}
\item \subcommand{subcommand} \subcommand{argument}
\item etc.
\end{description}

Blank lines are ignored, as is extra white space (i.e., tabs and spaces) between arguments. But don't put extra white space before a \command{} character (which must also be the first character on the line), and make sure the file ends with a carriage return. Commands and subcommands consist of letters and/or underscores, must not contain a spaces or full-point (`.').

There is no need to mark the end of a command block. This is automatically recognized by either the end of the file, section, or the start of the next command block (which is marked by the \command{} on the first character of a line). The only exception is \command{IncludeFile} (see Section \ref{sec:general-syntax})\index{Command ! IncludeFile}. 

In general, commands, sub-commands, and arguments in the parameter files are case insensitive. But note, however, that if you are on a Linux system then external calls to files are case sensitive (i.e., when using \command{IncludeFile} \subcommand{\emph{file}}, the argument \subcommand{\emph{file}} will be case sensitive). 

\subsubsection{Commenting out lines}\index{Comments}

Comments beginning with \commentline\ are ignored. If you want to remove a group of commands or subcommands using \commentline, then you can comment out all lines in the block, not just the first line. 

Alternatively, you can comment out an entire block or section by placing curly brackets around the text that you want to comment out. Put in a \commentstart\ as the first character on the line to start the comment block, then end it with \commentend. All lines (including line breaks) between \commentstart\ and \commentend\ inclusive are ignored. (These should ideally be the first character on a line. But if not, then the entire line will be treated as part of the comment block.)

\subsubsection{Commands}\index{Commands}

\SPM\ has a range of commands that define the model structure, processes, observations, how tasks are carried out. There are three types of commands, 
\begin{enumerate}
\item Commands that have an argument and do not have subcommands (for example, \command{IncludeFile})
\item Commands that have a label and subcommands (for example \command{ConstantRecruitment})
\item Commands that do not have either a label or argument, but have subcommands (for example \command{Structure})
\end{enumerate}

Commands that have a label must have a unique label, i.e., the label cannot be used on more that one command. The labels must start with a letter or underscores, can contain letters, underscores, or numbers, but must not contain white-space or full-point (`.').

\subsubsection{Subcommands}\index{Commands ! Subcommands}

Subcommands in \SPM\ are for defining options and parameter values for commands. They always take an argument which can be one or more of a range of types (the specific types acceptable for each subcommand are defined in Section \ref{sec:syntax}, but the general types are described below). 

Unlike commands (\command{command}), subcommands can be order specific. In other words, the order in which they appear within a command-block is important, and can affect the way in which they are interpreted. Usually, \SPM\ will report an error if they are not supplied in the order that is expected, however, in some circumstances \SPM\ will not \textemdash leading to possible errors in your expected results.  

Typically, the arguments for a subcommand are either\index{Subcommand argument type},

\begin{description}
\item \textbf{switch} true/false 
\item \textbf{integer} an integer number
\item \textbf{integer vector} a vector of integer numbers
\item \textbf{constant} a real number 
\item \textbf{constant vector} a vector of real numbers
\item \textbf{estimable} a real number that can be estimated
\item \textbf{estimable vector} a vector of real numbers that can be estimated
\item \textbf{string} a categorical value
\item \textbf{string vector} a vector of categorical values
\end{description}

Switches are parameters which are either true or false. Enter \emph{true} as \texttt{true} or \texttt{t}, and \emph{false} as \texttt{false} or \texttt{f}. 

Integers must be entered as integers (i.e., if \subcommand{year}\ is an integer then use 2008, not 2008.0)

Arguments of type integer vector, constant vector, estimable vector, or categorical vector contain one or more entries on a row, separated by white space (tabs or spaces). 

Estimable parameters are those parameters that \SPM\ can estimate, if requested. If a particular parameter is not being estimated in a particular model run, then it acts as a constant.  Within \SPM\, only estimable parameters can be estimated. However, you have to tell \SPM\ those that are to be estimated in any particular model. Estimable parameters that are being estimated within a particular model run are called the \emph{free parameters}.

\subsubsection{Determining parameter names}

When SPM processes a \config, it translates each command and each subcommand into a parameter with a unique name. For commands, this parameter name is simply the command name. For subcommands, the parameter name format is either, 

\begin{description}
\item \command{command}\subcommand{[label]}.\subcommand{subcommand} if the command has a label, or
\item \command{command}.\subcommand{subcommand} if the command has no label, or
\item \command{command}\subcommand{[label]}.\subcommand{subcommand}\subcommand{[i]} if the command has a label, and we are accessing the \emph{i}th element of that vector of values of the subcommand.
\end{description} 

The unique parameter name is then used to reference the parameter when estimating, applying a penalty, or a profile. For example, the parameter name of subcommand \subcommand{R0} of the command \command{population\_process} with the label \subcommand{recruitment} is,

\texttt{@population\_process[recruitment].R0}

\subsection{\SPM\ exit status values}

When \SPM\ completes its task successfully or errors out gracefully, it returns a single exit status value (0) to the operating system. Otherwise the operating system will return -1. To determine if \SPM\ completed its task successfully, then check the standard output for error and information messages.
