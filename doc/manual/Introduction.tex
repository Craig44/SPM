\section{Introduction\label{sec:Introduction}}

\SPM\ (\SPMName) is a generalised spatially explicit age-structured population dynamics and movement model. \SPM\ can model population dynamics and movement parameters for an age-structured population using a range of observations, including tagging, relative abundance, and age frequency data. \SPM\ implements an age-structured population within an arbitrary shaped spatial structure, which can have user defined categories (e.g., immature, mature, male, female, etc.), and age range. 

This manual describes how to use \SPM, including how to run \SPM, how to set up an \config. Further, we describe the population dynamics and estimation methods, and describe how to specify and interpret output. 

\subsection{Version\label{sec:version}}

This document (last modified \DocVer) describes \SPM\ \VER. The \SPM\ version number is suffixed with a date/time (\texttt{yyyy-mm-dd}) and revision number, giving the revision control system UTC date and revision number for the most recent modification of the source files. User manual updates will usually be issued for each minor version or date release of \SPM, and can be obtained, on request, from the authors.\index{Version number}

\subsection{Citing \SPM}

A suitable reference for \SPM\ and this document is:

\ManualRef\index{Citation}\index{Citing \SPM}

\subsection{\I{Software license}\index{Common Public License}}

This program and the accompanying materials are made available under the terms of the \href{http://www.opensource.org/licenses/cpl1.0.php}{Common Public License v1.0} which accompanies this software (see Section \ref{sec:Common-Public-License}).

Copyright \copyright 2008-\SourceControlYearDoc, \href{http://www.niwa.co.nz}{\Organisation} and the \href{http://www.mpi.govt.nz}{New Zealand Ministry for Primary Industries}. All rights reserved.

\subsection{\I{System requirements}}

\SPM\ is available for most IBM compatible machines running 64-bit only \I{Linux} and \I{Microsoft Windows} operating systems.

Several of \SPM s tasks are highly computer intensive and a fast processor is recommended. We recommend a minimum of 10 megabytes of free RAM (although, depending on the scope of the problem, you may need much more). Some of \SPM s tasks can be multi-threaded, and hence multi-core machines may perform some tasks considerably quicker than single core processors. The program itself requires only a few megabytes of hard-disk space but output files can consume large amounts of disk space. Depending on number and type of user output requests, the output could range from a few hundred kilobytes to several hundred megabytes. However, we note that, depending on the model implemented, some of \SPM s tasks can take a considerable amount of time.

\subsection{\I{Necessary files}}

For both 64-bit Linux and Microsoft Windows, only the executable file \texttt{spm} or \texttt{spm.exe} is required to run \SPM . No other software is required. We do not currently create a version for 32-bit operating systems. 

\SPM\ offers little in the way of  post-processing of the output, and a package available that allows tabulation and graphing of model outputs is recommended. We suggest software such as \href{http://www.microsoft.com}{Microsoft Excel}, \href{http://www.insightful.com}{S-Plus}, or \href{http://www.r-project.org}{\R}\ (R Development Core Team 2007). To assist in the post processing of \SPM output, we provide the \texttt{spm} \R\ package for importing the \SPM\ output into R (see Section \ref{sec:post-processing}).

\subsection{Getting help\index{Getting help}\index{User assistance}\index{Notifying errors}}

\SPM\ is distributed as unsupported software, however we would appreciate being notified of any problems or errors in \SPM. See Section \ref{sec:reporting-errors} for how to report errors to the \authors. Further information about \SPM\ can be obtained by contacting the \authors.

\subsection{Technical details\index{Technical specifications}}

\SPM\ was compiled on Linux using \href{http://gcc.gnu.org}{\texttt{gcc}}, the C/C++ compiler developed by the \href{http://gcc.gnu.org}{GNU Project}. The 64-bit Linux \index{Linux} version was compiled using \texttt{gcc} version 4.1.2 20070115 (prerelease) (\href{http://www.opensuse.org/}{SUSE Linux}). Note that \SPM\ is not supported for Linux kernel versions prior to 2.6. The \href{http://www.microsoft.com}{Microsoft Windows}\index{Microsoft Windows} version was compiled using \href{http://www.mingw.org}{Mingw32}\index{Mingw} \href{http://gcc.gnu.org}{\texttt{gcc}} 4.6.1. The \href{http://www.microsoft.com}{Microsoft Windows} installer was built using the \href{http://nsis.sourceforge.net/Main_Page}{Nullsoft Scriptable Install System}.

\SPM\ uses two minimisers --- the first is closely based on the main algorithm of \cite{779}, and which which uses finite difference gradients\index{Finite differences minimiser}, and the second is an implementation of the differential evolution solver\index{Differential evolution minimiser} \citep{1442}, and based on code by \href{mailto:<godwin@pushcorp.com>}{Lester E. Godwin} of \href{http://www.pushcorp.com}{PushCorp, Inc.} The random number generator\index{Random number generator} used by \SPM\ uses an implementation of the Mersenne twister random number generator \citep{796}. This, the command line functionality, matrix operations, and a number of other functions use the \href{http://www.boost.org/}{BOOST} C++ library (Version 1.38.0)\index{BOOST C++ library}.

Note that the output from \SPM\ may differ slightly on the different platforms due to different precision arithmetic or other platform dependent implementation issues. The source code\index{\SPM\ source code} for \SPM\ is available either as a part of the installation, or on request from the \authors.


