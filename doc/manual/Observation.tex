\section{The observation section\label{sec:observation-section}}

The objective function is based on the goodness-of-fit of the model to your observations. In the current release of \SPM, most observations are different kinds of time series, i.e., data which were recorded for one or more years, in the same format each year. Examples of time series data types include relative abundance indices, commercial catch length frequencies, survey numbers-at-age, etc,. 

Generally, time series must relate to a specified time step and a specified area, and one or more years in which they were recorded.

\subsection{Types of observations}

Five types

\begin{description}
  \item Observations of a mortality event proportions of individuals by age class
  \item Observations of proportions of individuals by age class
  \item Observations of proportions of individuals between categories within each age class
  \item Relative and absolute abundance observations
  \item Relative and absolute biomass observations
\end{description}

\subsubsection{Event mortality proportions-at-age}

\subsubsection{Proportions-at-age}

In the case study, proportions-at-age data from the commercial catch, by SSRU and year (Figure 8), were fitted to the modelled proportions-at-age composition using a multinomial likelihood, i.e., 

\begin{equation}
  -\log \left(L \right) =  -\log \left(N! \right) + \sum\limits_i \log \left( \left(NO_i \right)! \right) - NO_i \log \left(E_i \right)
\end{equation}

where $O_i$ are the observed proportions at age $i$, $E_i$ are the model expected proportions at age $i$, and $N$ is the effective sample size. 

\subsubsection{Proportions-by-category}

The likelihood for these observations was implemented as a binomial likelihood. Here, 

\begin{equation}
  \begin{split}
    -\log \left(L \right)= -\sum\limits_i & \log \left(N_i! \right) - \log \left(\left(N_i \left(1 - O_i \right) \right)! \right) - \log \left(\left(N_i O_i \right)! \right) + N_i O_i \log \left(E_i \right) \\
    &+ N_i \left(1 - O_i \right)\log \left(1 - E_i \right)
  \end{split}
\end{equation}

where $O_i$ are the observed proportions by category $i$, $E_i$ are the expected proportions by category $i$, and $N$ is the effective sample size. 

\subsubsection{Abundance}

The lognormal likelihood with observations $O_i$, c.v. $c_i$, and expected values $qE_i$, and was,

\begin{equation}
 - \log \left( L \right) = \sum\nolimits_{i = 1}^n \left( {\log \left( {\sigma _i } \right) + 0.5\left( {\frac{{\log \left( {{{O_i } \mathord{\left/  {\vphantom {{O_i } {qE_i }}} \right. \kern-\nulldelimiterspace} {qE_i }}} \right)}} {{\sigma _i }} + 0.5\sigma _i } \right)^2 } \right) 
\end{equation}

where 

\begin{equation}
  \sigma_i  = \sqrt{\log \left(1+c_i^2 \right)}
\end{equation}

.

For the case study, c.v.s were all assumed to be 0.2.


\subsection{The objective function}

\subsubsection{The objective function}

In Bayesian estimation, the objective function is a negative log-posterior,

\[
Objective(p)= - \sum\limits_i {\log \left[ {L\left( {{\bf{p}}|O_i } \right)} \right]}  - \log \left[ {\pi \left( {\bf{p}} \right)} \right]
\]

where $\pi$ is the joint prior density of the parameters $p$.

Under either estimation method, penalties can be added to the objective function (see Section XXX). You will usually want to use penalties to ensure that the exploitation rate constraints on your fisheries are not breached (otherwise there is nothing to prevent the model from having abundances so low that the recorded catches could not have been taken). A penalty to force the YCS to average to 1 (i.e., to have mean 1) may also be necessary.

\subsubsection{Likelihoods}

\subsubsection{Priors}

\subsubsection{Penalties}

\subsection{Simulating observations}

\subsection{Pseudo-observations}