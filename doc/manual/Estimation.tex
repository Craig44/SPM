\section{The estimation section\label{sec:estimation-section}}

\subsection{Role of the estimation section\label{sec:role-of-the-estimation-section}}

The tasks carried out by the estimation section are: 

1. Get the point estimate, i.e., the maximum likelihood estimate (MLE) or maximum posterior density estimate (MPD) (see Section 6.3).

2. Profile selected parameters, i.e., find, for each of a series of values of a parameter, allowing the other free parameters to vary, the minimum value of the objective function (Section 6.4). This is called either a likelihood or posterior profile. 

3. For Bayesian estimation only, generate an MCMC sample from the posterior distribution (Section 6.5).

4. For maximum likelihood or Bayesian estimation, calculate the approximate covariance matrix of the parameters as the inverse of the minimizer\textquoteright{}s approximation to the Hessian, and the corresponding correlation matrix (Section 6.3).
