\section{Population command and subcommand syntax\label{sec:population-syntax}}

\subsection{\I{Model structure}}

\defCom{model}{Define the spatial structure, population structure, annual cycle, and model years}

\defSub{nrows}{The number of rows $n_{rows}$ in the spatial structure}
\defType{Integer}
\defDefault{No default}
\defValue{A positive integer, $n_{rows} > 0$}

\defSub{ncols}{The number of columns $n_{cols}$ in the spatial structure}
\defType{Integer}
\defDefault{No default}
\defValue{A positive integer, $n_{cols} > 0$}

\defSub{layer}{The label for the base layer}
\defType{String}
\defDefault{No default}
\defValue{Must be a label of a \argument{numeric} layer defined by \command{layer}}

\defSub{cell\_length}{The length (distance) of one side of a cell}
\defType{Constant}
\defDefault{1}
\defValue{A positive real number}

\defSub{categories} {Labels of the categories (rows) of the population component of the partition}
\defType{Vector of strings, of length $1\ldots n_{categories}$}
\defDefault{No default}
\defValue{Names of categories must be unique}

\defSub{min\_age}{Minimum age of the population}
\defType{Integer}
\defDefault{No default}
\defValue{A non-negative integer, ${age}_{min}\geq 0$ and ${age}_{min}\leq {age}_{max}$}

\defSub{max\_age}{Maximum age of the population}
\defType{Integer}
\defDefault{No default}
\defValue{A non-negative integer, ${age}_{max}\geq 0$ and ${age}_{min}\geq {age}_{min}$}

\defSub{age\_plus\_group}{Define the largest age as a plus group}
\defType{Switch}
\defDefault{True}
\defValue{Defines  the largest age as a plus group}

\defSub{age\_size}{Define the label of the associated age-size relationship for each category}
\defType{Vector of strings, of length $n_{categories}$}
\defDefault{No default}
\defValue{Must be labels of command \command{age\_size}}

\defSub{initialisation\_phases}{Define the labels of the phases of the initialisation}
\defType{Vector of strings, of length of the number of initialisation phases}
\defDefault{No default}
\defValue{A valid label defined by \command{initialisation\_phase}}

\defSub{initial\_year}{Define the first year of the model, immediately following initialisation}
\defType{Integer}
\defDefault{No default}
\defValue{Defines the first year of the model, $\geq 1$, e.g. 1990}

\defSub{current\_year}{Define the current year of the model}
\defType{Integer}
\defDefault{No default}
\defValue{Defines the current year of the model, i.e., the model is run from \commandsub{model}{first\_year}\ to \commandsub{model}{current\_year}}

%\defSub{final\_year}{Define the final year of the model in projections}
%\defType{Integer}
%\defDefault{No default}
%\defValue{Defines the final year of the model for use in projections, i.e., the model is run from \commandsub{model}{first\_year} to %\commandsub{model}{current\_year}, then projected to \commandsub{model}{final\_year}}

\defSub{time\_steps} {Define the \command{time\_step} labels (in order that they are applied) to form the annual cycle}
\defType{String vector}
\defDefault{No default}
\defValue{A valid time\_step label, from one of \command{time\_step}}

\subsection{\I{Initialisation}}

%The methods for initialisation available are,
%\begin{itemize}
%	\item Iterative
%\end{itemize}
%
%Each type of initialisation requires a set of subcommands and arguments specific to that type.
%
\defComLab{initialisation\_phase}{Define the processes and years of the initialisation phase with label}
%
%\defSub{type} {Define the type of initialisation}
%\defType{String}
%\defDefault{No default}
%\defValue{A valid type of initialisation}
%
%\subsubsection[Iterative initialisation]{\commandlabsubarg{initialisation\_phase}{type}{iterative}}

\defSub{years} {Define the number of years to run}
\defType{Integer}
\defDefault{No default}
\defValue{A non-negative integer}

\defSub{time\_steps} {Define the \command{time\_step} labels (in order that they are applied) in this initialisation phase}
\defType{String vector}
\defDefault{No default}
\defValue{A valid time\_step label, from one of \command{time\_step}}

\defSub{lambda} {Define the absolute proportional difference for assessing convergence between annual iterations during the initialisation}
\defType{Constant}
\defDefault{Zero}
\defValue{A number between $0$ and $1$}
\defNote{If $\lambda$ is zero, then \SPM\ will calculate the value of the absolute proportional difference between annual iterations, but not terminate early}

\defSub{lambda\_years} {Define the years to test for convergence during the initialisation}
\defType{Constant vector}
\defDefault{No default}
\defValue{A vector of valid years between $1$ and \commandlabsub{initialisation\_phase}{years}}
\defNote{If $\lambda$ is defined, but \commandlabsub{initialisation\_phase}{lambda\_years} is not, \SPM\ will evaluate $\lambda$ for the last year in \commandlabsub{initialisation\_phase}{years}}

\subsection{\I{Time-steps}}

\defComLab{time\_step} {Define a time-step with label}

\defSub{processes} {Define the process labels, in the order that they are applied, for the time-step}
\defType{String vector}
\defDefault{No default}
\defValue{Defines the labels of the processes for that time-step}

\subsection{\I{Processes}}

The population processes available are,

\begin{itemize}
	\item Constant recruitment process
  \item Beverton-Holt stock-recruit relationship recruitment process
  \item Local Beverton-Holt stock-recruit relationship recruitment process
	\item Ageing process
	\item Constant relationship mortality rate process
	\item Annually varying relationship mortality rate process
	\item Mortality event (as a number) process
	\item Mortality event (as a biomass) process
	\item Holling mortality rate
	\item Prey-suitability predation process
	\item Category transition process
	\item Category transition rate process
	\item Category transition by age process
\end{itemize}

The movement processes available are,

\begin{itemize}
	\item Migration movement
	\item Adjacent cell movement
	\item Preference movement
\end{itemize}

Each type of process requires a set of subcommands and arguments specific to that process.

\defComLab{process} {Define a process with label}

\defSub{type} {Define the type of process}
\defType{String}
\defDefault{No default}
\defValue{A valid type of process}

\subsubsection[Constant recruitment process]{\commandlabsubarg{process}{type}{constant\_recruitment}}

\defSub{r0} {Define the total amount of recruitment at equilibrium abundance levels}
\defType{Estimable}
\defDefault{No default}
\defValue{Total amount (in numbers) of recruitment applied across all categories at equilibrium abundances}

\defSub{categories} {Define the categories into which recruitment occurs}
\defType{String vector}
\defDefault{No default}
\defValue{Valid categories from \commandsub{model}{categories}}

\defSub{proportions} {Define the proportion of recruitment that occurs into each category}
\defType{Estimable vector, of length \subcommand{categories}}
\defDefault{No default}
\defValue{Proportion of the annual recruitment that is applied to each category}

\defSub{age} {Define the age that receives recruitment}
\defType{Integer}
\defDefault{The minimum age of the population}
\defValue{The age class that receives recruitment}

\defSub{layer} {Name of the layer used to determine where recruitment occurs}
\defType{String}
\defDefault{No default}
\defValue{A valid layer as defined by \command{layer}. If a numeric layer, then recruitment is in proportion to the layer values. Note that the layer values must be non-negative}

\subsubsection[Beverton-Holt recruitment process]{\commandlabsubarg{process}{type}{bh\_recruitment}}

\defSub{r0} {Define the total amount of recruitment at equilibrium abundance levels}
\defType{Estimable}
\defDefault{No default}
\defValue{Total amount (in numbers) of recruitment applied across all categories at equilibrium abundances}

\defSub{categories} {Define the categories into which recruitment occurs}
\defType{String vector}
\defDefault{No default}
\defValue{Valid categories from \commandsub{model}{categories}}

\defSub{proportions} {Define the proportion of recruitment that occurs into each category}
\defType{Estimable vector, of length \commandlabsub{process}{categories}}
\defDefault{No default}
\defValue{Proportion of the annual recruitment that is applied to each category}

\defSub{age} {Define the age that receives recruitment}
\defType{Integer}
\defDefault{The minimum age of the population}
\defValue{The age class that receives recruitment}

\defSub{steepness} {Define the Beverton-Holt stock recruitment relationship steepness ($h$) parameter}
\defType{Estimable}
\defDefault{1.0}
\defValue{Steepness value between 0.2 and 1.0}

%\defSub{sigma\_r} {Define the recruitment variability $\sigma_R$ in the stock-recruitment relationship for projections}
%\defType{Estimable}
%\defDefault{1.0}

%\defSub{rho} {Define the autocorrelation $\rho$ in the recruitment variability in the stock-recruitment relationship for projections}
%\defType{Estimable}
%\defDefault{0.0}

\defSub{b0} {Define the \command{initialisation\_phase} label for the value of the derived quantity  to use as the value of the spawning stock biomass ($B_0$)}
\defType{String}
\defDefault{No default}
\defValue{Must be a valid \command{initialisation\_phase} label}

\defSub{ssb} {Define the label of the \command{derived\_quantity} that defines the spawning stock biomass (SSB)}
\defType{String}
\defDefault{No default}
\defValue{Must be a valid \command{derived\_quantity} label}

\defSub{ssb\_offset} {Define the offset (in years) for the year of the derived quantity that is to be applied as the SSB in the stock-recruit relationship}
\defType{Integer}
\defDefault{No default}
\defValue{Must be a value $\ge 0$}

\defSub{ycs\_values} {YCS values}
\defType{Estimable vector}
\defDefault{No default}
\defValue{Must be vector of length \commandsub{model}{initial} to \commandsub{model}{current}}

\defSub{standardise\_ycs\_years} {Years for which the year class strength values are defined to have mean 1.0}
\defType{Integer vector or integer range}
\defDefault{No default}
\defValue{The expanded vector must have values of years between \commandsub{model}{initial} and \commandsub{model}{current}}

\defSub{layer} {Name of the layer used to determine where recruitment occurs}
\defType{String}
\defDefault{No layer}
\defValue{A valid layer as defined by \command{layer}. If a numeric layer, then the total recruitment $R_0$ in each cell is scaled to be proportional to the value of the layer in that cell.}

\subsubsection[Local Beverton-Holt recruitment process]{\commandlabsubarg{process}{type}{local\_bh\_recruitment}}

\defSub{r0} {Define a multiplier of \subcommand{r0\_layer} for calculating the amount of recruitment in each cell at equilibrium abundance levels}
\defType{Estimable}
\defDefault{No default}
\defValue{Multiplier of \subcommand{r0\_layer} to calculate the amount in each cell (in numbers) of recruitment applied across all categories at equilibrium abundances}

\defSub{categories} {Define the categories into which recruitment occurs}
\defType{String vector}
\defDefault{No default}
\defValue{Valid categories from \commandsub{model}{categories}}

\defSub{proportions} {Define the proportion of recruitment that occurs into each category}
\defType{Estimable vector, of length \subcommand{categories}}
\defDefault{No default}
\defValue{Proportion of the annual recruitment that is applied to each category}

\defSub{age} {Define the age that receives recruitment}
\defType{Integer}
\defDefault{No default}
\defValue{The age class that receives recruitment}

\defSub{steepness} {Define the Beverton-Holt stock recruitment relationship steepness ($h$) parameter}
\defType{Estimable}
\defDefault{1.0}
\defValue{Steepness value between 0.2 and 1.0}

%\defSub{sigma\_r} {Define the recruitment variability $\sigma_R$ in the stock-recruitment relationship for projections}
%\defType{Estimable}
%\defDefault{1.0}

%\defSub{rho} {Define the autocorrelation $\rho$ in the recruitment variability in the stock-recruitment relationship for projections}
%\defType{Estimable}
%\defDefault{0.0}

\defSub{b0} {Define the \command{initialisation\_phase} label for the value of the derived quantity  to use as the value of the spawning stock biomass ($B_0$) in each cell}
\defType{String}
\defDefault{No default}
\defValue{Must be a valid \command{initialisation\_phase} label}

\defSub{ssb} {Define the label of the \command{derived\_quantity\_by\_cell} that defines the spawning stock biomass (SSB) in each cell}
\defType{String}
\defDefault{No default}
\defValue{Must be a valid \command{derived\_quantity\_by\_cell} label}

\defSub{ssb\_offset} {Define the offset (in years) for the year of the derived quantity by cell that is to be applied as the SSB in the stock-recruit relationship}
\defType{Integer}
\defDefault{No default}
\defValue{Must be a value $\ge 0$}

\defSub{ycs\_values} {YCS values}
\defType{Estimable vector}
\defDefault{No default}
\defValue{Must be vector of length \commandsub{model}{initial} to \commandsub{model}{current}}

\defSub{standardise\_ycs\_years} {Years for which the year class strength values are defined to have mean 1.0}
\defType{Integer vector or integer range}
\defDefault{No default}
\defValue{The expanded vector must have values of years between \commandsub{model}{initial} and \commandsub{model}{current}}

\defSub{layer} {Define the label of the layer that defines the distribution of recruitment (as a multiplier of $R_0$ at equilibrium abundances in each cell}
\defType{String}
\defDefault{No layer}
\defValue{A valid numeric layer as defined by \command{layer}. The amount of recruitment in each cell is the product of $R_0$ and the layer value in each cell.}

\subsubsection[Ageing process]{\commandlabsubarg{process}{type}{ageing}}

\defSub{categories} {Define the categories that ageing is applied to}
\defType{String vector}
\defDefault{No default}
\defValue{Valid categories from \commandsub{model}{categories}}

\subsubsection[Constant mortality rate process]{\commandlabsubarg{process}{type}{constant\_mortality\_rate}}

\defSub{m} {Define the constant mortality rate to be applied}
\defType{Estimable vector, of length \subcommand{categories}}
\defDefault{No default}
\defValue{A vector of positive real numbers}

\defSub{categories} {Define the categories that mortality is applied to}
\defType{String vector}
\defDefault{No default}
\defValue{Valid categories from \commandsub{model}{categories}}

\defSub{selectivities} {Define the selectivities applied to each category}
\defType{String vector, of length \subcommand{categories}}
\defDefault{No default}
\defValue{Valid selectivity labels defined by \command{selectivity}}

\defSub{layer} {Name of the layer}
\defType{String}
\defDefault{No layer}
\defValue{A valid layer as defined by \command{layer}. If a numeric layer, then mortality applied is the mortality rate multiplied by the value of the layer. Note that the layer values must be non-negative}

\subsubsection[Constant exploitation rate process]{\commandlabsubarg{process}{type}{constant\_exploitation\_rate}}

\defSub{u} {Define the constant exploitation rate to be applied}
\defType{Estimable vector, of length \subcommand{categories}}
\defDefault{No default}
\defValue{A vector of positive real numbers}

\defSub{u\_max} {Define the maximum constant exploitation rate that could be applied}
\defType{Constant}
\defDefault{$0.99$}
\defValue{A positive real number between $0.0$ And $1.0$}

\defSub{categories} {Define the categories that mortality is applied to}
\defType{String vector}
\defDefault{No default}
\defValue{Valid categories from \commandsub{model}{categories}}

\defSub{selectivities} {Define the selectivities applied to each category}
\defType{String vector, of length \subcommand{categories}}
\defDefault{No default}
\defValue{Valid selectivity labels defined by \command{selectivity}}

\defSub{layer} {Name of the layer}
\defType{String}
\defDefault{No layer}
\defValue{A valid layer as defined by \command{layer}. If a numeric layer, then mortality applied is the exploitation rate multiplied by the value of the layer. Note that the layer values must be non-negative. Note that the result of the calculation will be set to 0 if less than zero, and set to $u_{max}$ if greater than $u_{max}$}

\subsubsection[Annual mortality rate process]{\commandlabsubarg{process}{type}{annual\_mortality\_rate}}

\defSub{years} {Define the years when the mortality rates are applied}
\defType{Integer vector or integer range}
\defDefault{No default}
\defValue{Valid model years}

\defSub{m} {Define the mortality rate to be applied for each year}
\defType{Estimable vector, of length \subcommand{years} once expanded}
\defDefault{No default}
\defValue{A vector of positive real numbers}

\defSub{categories} {Define the categories that mortality is applied to}
\defType{String vector}
\defDefault{No default}
\defValue{A vector of valid categories from \commandsub{model}{categories}}

\defSub{selectivities} {Define the selectivities applied to each category}
\defType{String vector of length \subcommand{categories}}
\defDefault{No default}
\defValue{A vector of valid selectivity labels defined by \command{selectivity}}

\defSub{layer} {Name of the multiplicative layer to be applied to $M$}
\defType{String}
\defDefault{No layer}
\defValue{A valid numeric layer as defined by \command{layer}. Note that the layer values must be non-negative}

\subsubsection[Layer varying exploitation rate process]{\commandlabsubarg{process}{type}{layer\_varying\_exploitation\_rate}}

\defSub{u} {Define the constant exploitation rate to be applied}
\defType{Estimable vector, of length \subcommand{categories}}
\defDefault{No default}
\defValue{A vector of positive real numbers}

\defSub{u\_max} {Define the maximum constant exploitation rate that could be applied}
\defType{Constant}
\defDefault{$0.99$}
\defValue{A positive real number between $0.0$ And $1.0$}

\defSub{categories} {Define the categories that mortality is applied to}
\defType{String vector}
\defDefault{No default}
\defValue{Valid categories from \commandsub{model}{categories}}

\defSub{selectivities} {Define the selectivities applied to each category}
\defType{String vector, of length \subcommand{categories}}
\defDefault{No default}
\defValue{Valid selectivity labels defined by \command{selectivity}}

\defSub{years} {Define the years when the exploitation rates are applied}
\defType{Integer vector or integer range}
\defDefault{No default}
\defValue{Valid model years}

\defSub{layers} {Names of the layers}
\defType{String vector, of length \subcommand{years} once expanded}
\defDefault{No layer}
\defValue{A vector of valid layers as defined by \command{layer}. Note that the layer values must be non-negative. Note that the result of the calculation will be set to 0 if less than zero, and set to $u_{max}$ if greater than $u_{max}$}

\subsubsection[Event mortality process]{\commandlabsubarg{process}{type}{event\_mortality}}

\defSub{categories} {Define the categories that the event mortality is applied to}
\defType{String vector}
\defDefault{No default}
\defValue{Valid categories from \commandsub{model}{categories}}

\defSub{years} {Define the years where the mortality even is applied}
\defType{Integer vector or integer range}
\defDefault{No default}
\defValue{Valid model years}

\defSub{layers} {Define the layers that specify the event mortality (as the abundance) in each year}
\defType{String vector, of length \subcommand{years} once expanded}
\defDefault{No default}
\defValue{Valid layers as defined by \command{layer}. Note that the layer values must be non-negative}

\defSub{u\_max}{Define the maximum exploitation rate}
\defType{Estimable}
\defDefault{0.99}
\defValue{Must be $> 0$ and $< 1$}

\defSub{selectivities} {Define the selectivities applied to each category}
\defType{String vector, of length \subcommand{categories}}
\defDefault{No default}
\defValue{Valid selectivity labels defined by \command{selectivity}}

\defSub{penalty} {Define the event mortality penalty label}
\defType{String}
\defDefault{No default}
\defValue{Valid penalty label defined by \command{penalty}}

\subsubsection[Biomass event mortality process]{\commandlabsubarg{process}{type}{biomass\_event\_mortality}}

\defSub{categories}{Define the categories that the event mortality is applied to}
\defType{String vector}
\defDefault{No default}
\defValue{Valid categories from \commandsub{model}{categories}}

\defSub{years}{Define the years where the mortality event is applied}
\defType{Integer vector or integer range}
\defDefault{No default}
\defValue{Valid years for the model}

\defSub{layers}{Define the layers that specify the event mortality (as a biomass) in each year}
\defType{String vector, of length \subcommand{years} once expanded}
\defDefault{No default}
\defValue{Valid layers defined by \command{layer}. Note that the layer values must be non-negative}

\defSub{u\_max} {Define the maximum exploitation rate}
\defType{Constant}
\defDefault{0.99}
\defValue{Must be $> 0$ and $< 1$}

\defSub{selectivities}{Define the selectivities applied to each category}
\defType{String vector, of length \subcommand{categories}}
\defDefault{No default}
\defValue{Valid selectivity labels defined by \command{selectivity}}

\defSub{penalty} {Define the event mortality penalty label}
\defType{String}
\defDefault{No default}
\defValue{Valid penalty label defined by \command{penalty}}

\subsubsection[Holling mortality rate]{\commandlabsubarg{process}{type}{Holling\_mortality\_rate}}

\defSub{is\_abundance}{Is the mortality applied as a biomass or as abundance}
\defType{Switch}
\defDefault{False}
\defValue{Either True or False}

\defSub{a}{Define the $a$ parameter of the Holling function}
\defType{Constant}
\defDefault{No default}
\defValue{A positive real number}

\defSub{b}{Define the $b$ parameter of the Holling function}
\defType{Constant}
\defDefault{No default}
\defValue{A positive real number}

\defSub{x}{Define the type of Holling function or Michaelis-Menton function}
\defType{Constant}
\defDefault{default $2$}
\defValue{A positive real number, use $2$ for Holling type II or $3$ for Holling Type III, or other positive real value for the generalised Michaelis-Menton function}

\defSub{categories}{Define the categories that the Holling mortality rate is applied to}
\defType{String vector}
\defDefault{No default}
\defValue{Valid categories from \commandsub{model}{categories}}

\defSub{selectivities}{Define the selectivities applied to each category}
\defType{String vector, of length \subcommand{categories}}
\defDefault{No default}
\defValue{Valid selectivity labels defined by \command{selectivity}}

\defSub{predator\_categories}{Define the categories of the predator}
\defType{String vector}
\defDefault{No default}
\defValue{Valid categories from \commandsub{model}{categories}}

\defSub{predator\_selectivities}{Define the selectivities applied to each predator category}
\defType{String vector, of length \subcommand{predator\_categories}}
\defDefault{No default}
\defValue{Valid selectivity labels defined by \command{selectivity}}

\defSub{u\_max} {Define the maximum exploitation rate}
\defType{Constant}
\defDefault{0.99}
\defValue{Must be $> 0$ and $< 1$}

\defSub{penalty} {Define the event mortality penalty label}
\defType{String}
\defDefault{No default}
\defValue{Valid penalty label defined by \command{penalty}}

\subsubsection[Prey-suitability predation process]{\commandlabsubarg{process}{type}{Prey-suitability\_predation}}

\defSub{is\_abundance}{Is the mortality applied as a biomass or as abundance}
\defType{Switch}
\defDefault{False}
\defValue{Either True or False}

\defSub{consumption\_rate}{Define the total predator consumption rate}
\defType{Numeric}
\defDefault{No default}
\defValue{A positive real number}

\defSub{consumption\_rate\_layer}{Define the label of the layer that defines the predator consumption rate in each cell}
\defType{String}
\defDefault{None}
\defValue{A valid numeric layer as defined by \command{layer}. The consumption rate in each cell is the product of the total consumption rate and the layer value in each cell. Note that the layer values must be non-negative.}

%\defSub{prey}{Define the vector of labels for all the prey groups used}
%\defType{String vector}
%\defDefault{No default}
%\defValue{Names must be unique}
%\defNote{These are the labels which links to the relative electivities for groups of categories}

\defSub{prey\_categories}{Define the prey categories that the predation mortality is applied to}
\defType{String vector}
\defDefault{No default}
\defValue{Valid categories from \commandsub{model}{categories}, grouping categories into $n$ prey groups using the $+$ symbol}

\defSub{prey\_selectivities}{Define the selectivities applied to each prey category}
\defType{String vector, of length \subcommand{prey\_categories}}
\defDefault{No default}
\defValue{Valid selectivity labels defined by \command{selectivity}}

\defSub{electivities}{Define the electivities applied to prey groups $1$ \ldots $n$}
\defType{Constant vector, of length $n$, the number of groups of prey in $\subcommand{prey\_categories}$}
\defDefault{No default}
\defValue{A vector of positive real numbers, must have values that sum to $1$}

%\defSub{prey\_groups}{Assign each prey category to a specific prey group}
%\defType{String vector, of length \subcommand{categories}}
%\defDefault{No default}
%\defValue{Valid labels defined by \subcommand{prey}}

\defSub{predator\_categories}{Define the categories of the predator}
\defType{String vector}
\defDefault{No default}
\defValue{Valid categories from \commandsub{model}{categories}}

\defSub{predator\_selectivities}{Define the selectivities applied to each predator category}
\defType{String vector, of length \subcommand{predator\_categories}}
\defDefault{No default}
\defValue{Valid selectivity labels defined by \command{selectivity}}

\defSub{u\_max} {Define the maximum exploitation rate}
\defType{Constant}
\defDefault{0.99}
\defValue{Must be $> 0$ and $< 1$}

\defSub{penalty} {Define the process penalty label}
\defType{String}
\defDefault{No default}
\defValue{Valid penalty label defined by \command{penalty}}

\subsubsection[Category state by age process]{\commandlabsubarg{process}{type}{category\_state\_by\_age}}

\defSub{category} {Define the category that is the object of the process}
\defType{String vector}
\defDefault{No default}
\defValue{A valid list of categories from \commandsub{model}{categories}}

\defSub{layer} {Name of the categorical layer used to group the spatial cells for the process}
\defType{String}
\defDefault{No default}
\defValue{A valid layer as defined by \command{layer}. Must be a layer of \texttt{type=categorical}}

\defSub{min\_age} {Define the minimum age for the process }
\defType{Integer}
\defDefault{No default}
\defValue{A valid age in the range \commandsub{model}{min\_age} and \commandsub{model}{max\_age}}

\defSub{max\_age} {Define the maximum age for the process}
\defType{Integer}
\defDefault{No default}
\defValue{A valid age in the range \commandsub{model}{min\_age} and \commandsub{model}{max\_age}}

\defSub{N [label]}{Define the following data as the number of individuals to move in each age class}
\defType{Estimable vectors}
\defDefault{No default}
\defValue{The label is valid value from the associated category transition by age layer. It is followed by a vector of values giving the numbers in each age class. This subcommand is repeated for each unique value of label}

\subsubsection[Category transition process]{\commandlabsubarg{process}{type}{category\_transition}}

\defSub{from} {Define the categories that are the source of the transition process}
\defType{String vector}
\defDefault{No default}
\defValue{A valid list of categories from \commandsub{model}{categories}}

\defSub{selectivities} {Define the selectivities applied to the source categories}
\defType{String vector, of length \subcommand{from}}
\defDefault{No default}
\defValue{A valid list of selectivity labels defined by \command{selectivity}}

\defSub{to} {Define the categories that are the sink of the transition process}
\defType{String vector}
\defDefault{No default}
\defValue{A valid list of categories from \commandsub{model}{categories}}

\defSub{years} {Define the years where the category transition is applied}
\defType{Integer vector or integer range}
\defDefault{No default}
\defValue{Valid model years}

\defSub{layers} {Define the layers that specify the transitions (as N for each cell) in each year}
\defType{String vector, of length \subcommand{years} once expanded}
\defDefault{No default}
\defValue{Valid layers defined by \command{layer}. Note that the layer values must be non-negative}

\defSub{u\_max} {Define the maximum proportion of individuals that can be moved}
\defType{Constant}
\defDefault{0.99}
\defValue{Must be $> 0$ and $\le 1$}

\defSub{penalty} {Define the penalty to encourage models parameter values away from those which result in not enough individuals to move}
\defType{String}
\defDefault{No default}
\defValue{Valid penalty label defined by \command{penalty}}

\subsubsection[Category transition rate process]{\commandlabsubarg{process}{type}{category\_transition\_rate}}

\defSub{from} {Define the category that is the source of the transition process}
\defType{String}
\defDefault{No default}
\defValue{A valid category from \commandsub{model}{categories}}

\defSub{selectivities} {Define the selectivities applied to the source categories}
\defType{String vector, of length \subcommand{from}}
\defDefault{No default}
\defValue{A valid list of selectivity labels defined by \command{selectivity}}

\defSub{to} {Define the category that is the sink of the transition process}
\defType{String}
\defDefault{No default}
\defValue{A valid category from \commandsub{model}{categories}}

\defSub{proportions} {Define the proportion of individuals to move}
\defType{Estimable}
\defDefault{No default}
\defValue{A value $\ge 0$ and $\le 1$}

\defSub{layer} {Name of the layer}
\defType{String}
\defDefault{No default}
\defValue{A valid layer as defined by \command{layer}. If a numeric layer, then rate applied to each cell is multiplied by the value of the layer. Note that the layer values must be non-negative}

\subsubsection[Category transition by age process]{\commandlabsubarg{process}{type}{category\_transition\_by\_age}}

\defSub{from} {Define the categories that are the source of the transition process}
\defType{String vector}
\defDefault{No default}
\defValue{A valid list of categories from \commandsub{model}{categories}}

\defSub{to} {Define the categories that are the sink of the transition process}
\defType{String vector}
\defDefault{No default}
\defValue{A valid list of categories from \commandsub{model}{categories}}

\defSub{year} {Define the year when the category transition is applied}
\defType{Integer}
\defDefault{No default}
\defValue{A positive integer between \commandsub{model}{initial\_year} and \commandsub{model}{current\_year}}

\defSub{layer} {Name of the categorical layer used to group the spatial cells for the process}
\defType{String}
\defDefault{No default}
\defValue{A valid layer as defined by \command{layer}. Must be a layer of \texttt{type=categorical}}

\defSub{min\_age} {Define the minimum age for the process }
\defType{Integer}
\defDefault{No default}
\defValue{A valid age in the range \commandsub{model}{min\_age} and \commandsub{model}{max\_age}}

\defSub{max\_age} {Define the maximum age for the process}
\defType{Integer}
\defDefault{No default}
\defValue{A valid age in the range \commandsub{model}{min\_age} and \commandsub{model}{max\_age}}

\defSub{N [label]}{Define the following data as the number of individuals to move in each age class}
\defType{Estimable vectors}
\defDefault{No default}
\defValue{The label is valid value from the associated category transition by age layer. It is followed by a vector of values giving the numbers in each age class. This subcommand is repeated for each unique value of label}

\defSub{u\_max} {Define the maximum proportion of individuals that can be moved}
\defType{Constant}
\defDefault{0.99}
\defValue{Must be $> 0$ and $\le 1$}

\defSub{penalty} {Define the penalty to encourage models parameter values away from those which result in not enough individuals to move}
\defType{String}
\defDefault{No default}
\defValue{Valid penalty label defined by \command{penalty}}

\subsubsection[Migration movement]{\commandlabsubarg{process}{type}{migration}}

\defSub{categories} {Define the categories that the migration movement event is applied to}
\defType{String vector}
\defDefault{No default}
\defValue{Valid categories from \commandsub{model}{categories}}

\defSub{selectivities} {Define the selectivities applied to each category}
\defType{String vector, of length \subcommand{categories}}
\defDefault{No default}
\defValue{Valid selectivity labels defined by \command{selectivity}}

\defSub{proportion} {Define the constant multiplier for the proportion of individuals that migrate}
\defType{Estimable}
\defDefault{1.0}
\defValue{A real number between 0 and 1, inclusive}

\defSub{source\_layer} {Define the label of a layer that defines the source cells of the migration movement event}
\defType{String}
\defDefault{No default}
\defValue{A valid layer defined by \command{layer}}

\defSub{sink\_layer} {Define the label of a layer that defines the sink cells of the migration movement event}
\defType{String}
\defDefault{No default}
\defValue{A valid layer defined by \command{layer}}

\subsubsection[Adjacent cell movement]{\commandlabsubarg{process}{type}{adjacent\_cell}}

\defSub{categories} {Define the categories that the adjacent cell movement event is applied to}
\defType{String vector}
\defDefault{No default}
\defValue{Valid categories from \commandsub{model}{categories}}

\defSub{selectivities} {Define the selectivities applied to each category}
\defType{String vector, of length \subcommand{categories}}
\defDefault{No default}
\defValue{Valid selectivity labels defined by \command{selectivity}}

\defSub{layer} {Define the label of a gradient layer that defines the relative strength of movement to adjacent cells}
\defType{String}
\defDefault{Default 1 in every cell, equivalent to uniform diffusion}
\defValue{A valid layer defined by \command{layer}}

\defSub{proportion} {Define the constant multiplier for the proportion that moves from each cell to the neighbouring cell}
\defType{Estimable}
\defDefault{1.0}
\defValue{A real number between 0 and 1, inclusive}

\subsubsection[Preference movement]{\commandlabsubarg{process}{type}{preference}}

\defSub{categories} {Define the categories that the preference function movement is applied to}
\defType{String vector}
\defDefault{No default}
\defValue{Valid categories from \commandsub{model}{categories}}

\defSub{proportion} {Define the constant multiplier for the proportion that the preference function movement is applied to}
\defType{Estimable}
\defDefault{1.0}
\defValue{A real number between 0 and 1, inclusive}

\defSub{preference\_functions} {Define the labels of the individual  preference functions that make up the total preference function}
\defType{String vector}
\defDefault{No default}
\defValue{Valid preference function labels defined by \command{preference\_function}}

\subsubsection[Multi-threaded preference movement]{\commandlabsubarg{process}{type}{preference\_threaded}}

\defSub{categories} {Define the categories that the preference function movement is applied to}
\defType{String vector}
\defDefault{No default}
\defValue{Valid categories from \commandsub{model}{categories}}

\defSub{proportion} {Define the constant multiplier for the proportion that the preference function movement is applied to}
\defType{Estimable}
\defDefault{1.0}
\defValue{A real number between 0 and 1, inclusive}

\defSub{preference\_functions} {Define the labels of the individual  preference functions that make up the total preference function}
\defType{String vector}
\defDefault{No default}
\defValue{Valid preference function labels defined by \command{preference\_function}}

\subsection{\I{Preference functions}}

The individual preference functions available are,

\begin{itemize}
	\item Constant
	\item Normal
	\item Double normal
	\item Logistic
	\item Inverse logistic
	\item Exponential
	\item Threshold
	\item Categorical
	\item Monotonic categorical
	\item Gaussian copula
	\item Gumbel copula
	\item Frank copula
	\item Independence copula
\end{itemize}

Each type of preference function requires a set of subcommands and arguments specific to that function.

\defComLab{preference\_function} {Define a preference function with label}

\defSub{type} {Define the type of preference function}
\defType{String}
\defDefault{No default}
\defValue{A valid type of preference function}

\subsubsection[Constant]{\commandlabsubarg{preference\_function}{type}{constant}}

\defSub{layer} {Defines the layer which supplies the preference function independent variable}
\defType{String}
\defDefault{No default}
\defValue{A valid layer defined by \command{layer}}

\defSub{alpha} {Defines the multiplicative constant $\alpha$}
\defType{Estimable}
\defDefault{1.0}

\subsubsection[Normal]{\commandlabsubarg{preference\_function}{type}{normal}}

\defSub{layer} {Defines the layer which supplies the preference function independent variable}
\defType{String}
\defDefault{No default}
\defValue{A valid layer defined by \command{layer}}

\defSub{alpha} {Defines the multiplicative constant $\alpha$}
\defType{Estimable}
\defDefault{1.0}

\defSub{mu} {Defines the $\mu$ parameter of the normal preference function}
\defType{Estimable}
\defDefault{No default}

\defSub{sigma} {Defines the $\sigma$ parameter of the normal preference function}
\defType{Estimable}
\defDefault{No default}

\subsubsection[Double-normal]{\commandlabsubarg{preference\_function}{type}{double\_normal}}

\defSub{layer} {Defines the layer which supplies the preference function independent variable}
\defType{String}
\defDefault{No default}
\defValue{A valid layer defined by \command{layer}}

\defSub{alpha} {Defines the multiplicative constant $\alpha$}
\defType{Estimable}
\defDefault{1.0}

\defSub{mu} {Defines the $\mu$ parameter of the double-normal preference function}
\defType{Estimable}
\defDefault{No default}

\defSub{sigma\_l} {Defines the $\sigma_L$ parameter of the double-normal preference function}
\defType{Estimable}
\defDefault{No default}

\defSub{sigma\_r} {Defines the $\sigma_R$ parameter of the double-normal preference function}
\defType{Estimable}
\defDefault{No default}

\subsubsection[Logistic]{\commandlabsubarg{preference\_function}{type}{logistic}}

\defSub{layer} {Defines the layer which supplies the preference function independent variable}
\defType{String}
\defDefault{No default}
\defValue{A valid layer defined by \command{layer}, with strictly positive values only}

\defSub{alpha} {Defines the multiplicative constant $\alpha$}
\defType{Estimable}
\defDefault{1.0}

\defSub{a50} {Defines the $a_{50}$ parameter of the logistic preference function}
\defType{Estimable}
\defDefault{No default}

\defSub{ato95} {Defines the $a_{to95}$ parameter of the logistic preference function}
\defType{Estimable}
\defDefault{No default}

\subsubsection[Inverse-logistic]{\commandlabsubarg{preference\_function}{type}{inverse\_logistic}}

\defSub{layer} {Defines the layer which supplies the preference function independent variable}
\defType{String}
\defDefault{No default}
\defValue{A valid layer defined by \command{layer}, with strictly positive values only}

\defSub{alpha} {Defines the multiplicative constant $\alpha$}
\defType{Estimable}
\defDefault{No default}

\defSub{a50} {Defines the $a_{50}$ parameter of the inverse-logistic preference function}
\defType{Estimable}
\defDefault{1.0}

\defSub{ato95} {Defines the $a_{to95}$ parameter of the inverse-logistic preference function}
\defType{Estimable}
\defDefault{No default}

\subsubsection[Exponential]{\commandlabsubarg{preference\_function}{type}{exponential}}

\defSub{layer} {Defines the layer which supplies the preference function independent variable}
\defType{String}
\defDefault{No default}
\defValue{A valid layer defined by \command{layer} of positive values only}

\defSub{alpha} {Defines the multiplicative constant $\alpha$}
\defType{Estimable}
\defDefault{No default}

\defSub{lambda} {Defines the $\lambda$ parameter of the exponential preference function}
\defType{Estimable}
\defDefault{1.0}

\subsubsection[Threshold]{\commandlabsubarg{preference\_function}{type}{threshold}}

\defSub{layer} {Defines the layer which supplies the preference function independent variable}
\defType{String}
\defDefault{No default}
\defValue{A valid layer defined by \command{layer}}

\defSub{alpha} {Defines the multiplicative constant $\alpha$}
\defType{Estimable}
\defDefault{1.0}

\defSub{n} {Defines the $N$ parameter of the threshold preference function}
\defType{Estimable}
\defDefault{No default}

\defSub{lambda} {Defines the $\lambda$ parameter of the threshold preference function}
\defType{Estimable}
\defDefault{No default}

\subsubsection[Categorical]{\commandlabsubarg{preference\_function}{type}{categorical}}

\defSub{layer} {Defines the layer which supplies the preference function independent variable}
\defType{String}
\defDefault{No default}
\defValue{A valid layer defined by \command{layer}}

\defSub{alpha} {Defines the multiplicative constant $\alpha$}
\defType{Estimable}
\defDefault{1.0}

\defSub{category\_labels} {Defines the unique labels of \argument{layer} in order of their coefficients}
\defType{String vector}
\defDefault{No default}
\defValue{A complete set of unique values of labels in \argument{layer} in order of \argument{category\_values}}

\defSub{category\_values} {Defines the coefficients for each unique label of \argument{layer} in order of their labels}
\defType{String vector}
\defDefault{No default}
\defValue{A complete set of positive values of labels in \argument{layer} in order of \argument{category\_labels}}

\subsubsection[Monotonic categorical]{\commandlabsubarg{preference\_function}{type}{monotonic\_categorical}}

\defSub{layer} {Defines the layer which supplies the preference function independent variable}
\defType{String}
\defDefault{No default}
\defValue{A valid layer defined by \command{layer}}

\defSub{alpha} {Defines the multiplicative constant $\alpha$}
\defType{Estimable}
\defDefault{1.0}

\defSub{category\_labels} {Defines the unique labels of \argument{layer} in order of their coefficients}
\defType{String vector}
\defDefault{No default}
\defValue{A complete set of unique values of labels in \argument{layer} in order of \argument{category\_values}}

\defSub{category\_values} {Defines the coefficients for each unique label of \argument{layer} in order of their labels}
\defType{Numeric vector}
\defDefault{No default}
\defValue{A complete set of positive values of labels in \argument{layer} in order of \argument{category\_labels}}

\subsubsection[Gaussian copula]{\commandlabsubarg{preference\_function}{type}{gaussian\_copula}}

\defSub{rho} {Defines the dependence parameter ($\rho$) for the copula}
\defType{Constant}
\defDefault{No default}
\defValue{Any real number $>=-1$ and $<=1$}

\defSub{layers} {Defines the two layers which supplies the preference function independent variables}
\defType{String}
\defDefault{No default}
\defValue{A valid layer defined by \command{layer}}

\defSub{pdf} {Defines the two PDFs for the copula}
\defType{String}
\defDefault{No default}
\defValue{Valid PDFs defined by \command{pdf}}

\subsubsection[Gumbel copula]{\commandlabsubarg{preference\_function}{type}{gumbel\_copula}}

\defSub{rho} {Defines the dependence parameter ($\rho$) for the copula}
\defType{Constant}
\defDefault{No default}
\defValue{Any real number $>=1$}

\defSub{layers} {Defines the two layers which supplies the preference function independent variables}
\defType{String}
\defDefault{No default}
\defValue{A valid layer defined by \command{layer}}

\defSub{pdf} {Defines the two PDFs for the copula}
\defType{String}
\defDefault{No default}
\defValue{Valid PDFs defined by \command{pdf}}

\subsubsection[Frank copula]{\commandlabsubarg{preference\_function}{type}{frank\_copula}}

\defSub{rho} {Defines the dependence parameter ($\rho$) for the copula}
\defType{Constant}
\defDefault{No default}
\defValue{Any real number $!=0$}

\defSub{layers} {Defines the two layers which supplies the preference function independent variables}
\defType{String}
\defDefault{No default}
\defValue{A valid layer defined by \command{layer}}

\defSub{pdf} {Defines the two PDF for the copula}
\defType{String}
\defDefault{No default}
\defValue{Valid PDFs defined by \command{pdf}}

\subsubsection[Independence copula]{\commandlabsubarg{preference\_function}{type}{independence\_copula}}

\defSub{layers} {Defines the two layers which supplies the preference function independent variables}
\defType{String}
\defDefault{No default}
\defValue{A valid layer defined by \command{layer}}

\defSub{pdf} {Defines the two PDFs for the copula}
\defType{String}
\defDefault{No default}
\defValue{Valid PDFs defined by \command{pdf}}

\subsection{\I{Probability Density Functions}}

The available Probability Density Functions (PDF) types are,

\begin{itemize}
	\item Normal
	\item Lognormal
	\item Exponential
	\item Uniform
\end{itemize}

\defComLab{pdf} {Define a PDF for use as a copula preference function with label}

\defSub{type} {Define the type of PDF}
\defType{String}
\defDefault{No default}
\defValue{A valid type of PDF}

\subsubsection[Normal]{\commandlabsubarg{pdf}{type}{normal}}

\defSub{mu} {Define the mean of the PDF}
\defType{Constant}
\defDefault{No default}

\defSub{sigma} {Define the variance of the PDF}
\defType{Constant}
\defDefault{No default}
\defValue{A real number $>0$}

\subsubsection[Lognormal]{\commandlabsubarg{pdf}{type}{lognormal}}

\defSub{mu} {Define the mean of the PDF}
\defType{Constant}
\defDefault{No default}

\defSub{sigma} {Define the variance of the PDF}
\defType{Constant}
\defDefault{No default}
\defValue{A real number $>0$}

\subsubsection[Exponential]{\commandlabsubarg{pdf}{type}{exponential}}

\defSub{lambda} {Define the mean of the PDF}
\defType{Constant}
\defDefault{No default}
\defValue{A real number $>0$}

\subsubsection[Uniform]{\commandlabsubarg{pdf}{type}{uniform}}

\defSub{a} {Define the minimum of the PDF}
\defType{Constant}
\defDefault{No default}
\defValue{A real number}

\defSub{b} {Define the maximum of the PDF}
\defType{Constant}
\defDefault{No default}
\defValue{A real number $>a$}

\subsection{\I{Layers}}

The available layer types  are,

\begin{itemize}
	\item Numeric
	\item Categorical
	\item Distance
	\item Haversine
	\item Abundance
	\item Biomass
	\item Abundance density
	\item Biomass density
	\item Numeric meta-layer
	\item Categorical meta-layer
	\item Derived quantity layer
	\item Derived quantity by cell layer
\end{itemize}

\defComLab{layer} {Define a layer function with label}

\defSub{type} {Define the type of layer}
\defType{String}
\defDefault{No default}
\defValue{A valid type of layer}

\subsubsection[Numeric]{\commandlabsubarg{layer}{type}{numeric}}

\defSub{data} {Define the values of the layer}
\defType{Constant vector, with total length \commandsub{model}{ncols} $\times$ \commandsub{model}{nrows}}
\defDefault{No default}
\defValue{A vector of values of length equal to the number of elements defined for the spatial structure}

\defSub{rescale} {If defined, then the values of the layer are rescaled to sum to this value}
\defType{Estimable}
\defDefault{No default}
\defNote{If not defined, then the layer is not rescaled}

\subsubsection[Categorical]{\commandlabsubarg{layer}{type}{categorical}}

\defSub{data} {Define the values of the layer}
\defType{Constant vector, with total length \commandsub{model}{ncols} $\times$ \commandsub{model}{nrows}}
\defDefault{No default}
\defValue{A vector of values of length equal to the number of elements defined for the spatial structure}

\subsubsection[Distance]{\commandlabsubarg{layer}{type}{distance}}

There are no other subcommands for \commandlabsubarg{layer}{type}{distance}.

\subsubsection[Haversine]{\commandlabsubarg{layer}{type}{haversine}}

\defSub{latitude} {Define the layer that specifies the latitudes for each cell}
\defType{String}
\defDefault{No default}
\defCondition{The argument must be a numeric layer, with values between -90 and 90.}

\defSub{longitude} {Define the layer that specifies the longitudes for each cell}
\defType{String}
\defDefault{No default}
\defCondition{The argument must be a numeric layer, with values between 0 and 360.}

\subsubsection[Dijkstra]{\commandlabsubarg{layer}{type}{dijkstra}}

There are no other subcommands for \commandlabsubarg{layer}{type}{dijkstra}.

\subsubsection[Haversine Dijkstra]{\commandlabsubarg{layer}{type}{haversine\_dijkstra}}

\defSub{latitude} {Define the layer that specifies the latitudes for each cell}
\defType{String}
\defDefault{No default}
\defCondition{The argument must be a numeric layer, with values between -90 and 90.}

\defSub{longitude} {Define the layer that specifies the longitudes for each cell}
\defType{String}
\defDefault{No default}
\defCondition{The argument must be a numeric layer, with values between 0 and 360.}

\subsubsection[Abundance]{\commandlabsubarg{layer}{type}{abundance}}

\defSub{categories} {Define the categories are used to calculate the abundance}
\defType{String vector}
\defDefault{No default}
\defValue{Valid categories from \commandsub{model}{categories}}

\defSub{selectivities} {Define the selectivities applied to each category}
\defType{String vector, of length \subcommand{categories}}
\defDefault{No default}
\defValue{Valid selectivity labels from \command{selectivity}}

\subsubsection[Biomass]{\commandlabsubarg{layer}{type}{biomass}}

\defSub{categories} {Define the categories are used to calculate the biomass}
\defType{String vector}
\defDefault{No default}
\defValue{Valid categories from \commandsub{model}{categories}}

\defSub{selectivities} {Define the selectivities applied to each category}
\defType{String vector, of length \subcommand{categories}}
\defDefault{No default}
\defValue{Valid selectivity labels from \command{selectivity}}

\subsubsection[Abundance-density]{\commandlabsubarg{layer}{type}{abundance\_density}}

\defSub{categories} {Define the categories are used to calculate the abundance}
\defType{String vector}
\defDefault{No default}
\defValue{Valid categories from \commandsub{model}{categories}}

\defSub{selectivities} {Define the selectivities applied to each category}
\defType{String vector, of length \subcommand{categories}}
\defDefault{No default}
\defValue{Valid selectivity labels from \command{selectivity}}

\subsubsection[Biomass-density]{\commandlabsubarg{layer}{type}{biomass\_density}}

\defSub{categories} {Define the categories are used to calculate the biomass}
\defType{String vector}
\defDefault{No default}
\defValue{Valid categories from \commandsub{model}{categories}}

\defSub{selectivities} {Define the selectivities applied to each category}
\defType{String vector, of length \subcommand{categories}}
\defDefault{No default}
\defValue{Valid selectivity labels from \command{selectivity}}

\subsubsection[Numeric meta-layer]{\commandlabsubarg{layer}{type}{numeric\_meta}}

\defSub{default\_layer} {Define the default layer to use in years or initialisation phases where it is not otherwise defined}
\defType{String}
\defDefault{No default}
\defCondition{The argument must be a numeric layer}

\defSub{years} {Define the years that have a non-default layer}
\defType{Integer vector or integer range}
\defDefault{No default}
\defValue{Must be valid model years}

\defSub{layers} {Define the layers for each of the years}
\defType{String vector, of length \argument{year} once expanded}
\defDefault{No default}
\defCondition{The arguments must be numeric layers}

\defSub{initialisation\_phases} {Define the initialisation phases that have a non-default layer}
\defType{String vector}
\defDefault{No default}
\defCondition{The arguments must be numeric layers}

\defSub{initialisation\_layers} {Define the layers for each of the initialisation phases}
\defType{String vector, of length the number of \argument{initialisation\_phases}}
\defDefault{No default}
\defCondition{The arguments must be numeric layers}

\subsubsection[Categorical meta-layer]{\commandlabsubarg{layer}{type}{categorical\_meta}}

\defSub{default\_layer} {Define the default layer to use in years or initialisation phases where it is not otherwise defined}
\defType{String}
\defDefault{No default}
\defCondition{The argument must be a categorical layer}

\defSub{years} {Define the years that have a non-default layer}
\defType{Integer vector or integer range}
\defDefault{No default}
\defValue{Must be valid model years}

\defSub{layers} {Define the layers for each of the years}
\defType{String vector, of length \argument{year} once expanded}
\defDefault{No default}
\defCondition{The arguments must be categorical layers}

\defSub{initialisation\_phases} {Define the initialisation phases that have a non-default layer}
\defType{String vector}
\defDefault{No default}
\defCondition{The arguments must be categorical layers}

\defSub{initialisation\_layers} {Define the layers for each of the initialisation phases}
\defType{String vector, of length the number of \argument{initialisation\_phases}}
\defDefault{No default}
\defCondition{The arguments must be categorical layers}

\subsubsection[Derived quantity layer]{\commandlabsubarg{layer}{type}{derived\_quantity}}

\defSub{derived\_quantity} {Define the label of the \command{derived\_quantity} that is used as the source for the layer}
\defType{String}
\defDefault{No default}
\defValue{Must be a valid \command{derived\_quantity} label}

\defSub{year\_offset} {Define the offset (in years) for the year of the derived quantity that is to be applied}
\defType{Integer}
\defDefault{No default}
\defValue{Must be a value $\ge 0$}

\subsubsection[Derived quantity by cell layer]{\commandlabsubarg{layer}{type}{derived\_quantity\_by\_cell}}

\defSub{derived\_quantity\_by\_cell} {Define the label of the \command{derived\_quantity\_by\_cell} that is used as the source for the layer}
\defType{String}
\defDefault{No default}
\defValue{Must be a valid \command{derived\_quantity\_by\_cell} label}

\defSub{year\_offset} {Define the offset (in years) for the year of the derived quantity by cell that is to be applied}
\defType{Integer}
\defDefault{No default}
\defValue{Must be a value $\ge 0$}

\subsection{\I{Derived quantities by cell}}

The individual types of derived quantity by cell available are,

\begin{itemize}
	\item Abundance
	\item Biomass
\end{itemize}

\defComLab{derived\_quantity\_by\_cell} {Define a derived quantity by cell with label}

\defSub{type} {Define the type of derived quantity by cell}
\defType{String}
\defDefault{No default}
\defValue{A valid type of derived quantity by cell, either \texttt{abundance} or \texttt{biomass}}

\subsubsection[Abundance]{\commandlabsubarg{derived\_quantity\_by\_cell}{type}{abundance}}

\defSub{categories} {Define the categories are used to calculate the derived quantity by cell}
\defType{String vector}
\defDefault{No default}
\defValue{Valid categories from \commandsub{model}{categories}}

\defSub{selectivities} {Define the selectivities}
\defType{String vector, of length \subcommand{categories}}
\defDefault{No default}
\defValue{Valid selectivity labels from \command{selectivity}}

\defSub{initialisation\_time\_steps} {Define the time-steps during the initialisation phases at the end of which the derived quantity by cell is calculated}
\defType{String}
\defDefault{No default}
\defValue{A valid time-step label from \command{time\_step}}

\defSub{time\_step} {Define the time-step at the end of which the derived quantity by cell is calculated}
\defType{String}
\defDefault{No default}
\defValue{A valid time-step label from \command{time\_step}}

\defSub{layer} {Define the layer to be used in the calculations}
\defType{String}
\defDefault{None}
\defValue{A valid numeric layer as defined by \command{layer}. Note, the final values of the derived quantity by cell is the sum of the each cell is multiplied by the value of this layer.}

\subsubsection[Biomass]{\commandlabsubarg{derived\_quantity\_by\_cell}{type}{biomass}}

\defSub{categories} {Define the categories are used to calculate the derived quantity by cell}
\defType{String vector}
\defDefault{No default}
\defValue{Valid categories from \commandsub{model}{categories}}

\defSub{selectivities} {Define the selectivities}
\defType{String vector, of length \subcommand{categories}}
\defDefault{No default}
\defValue{Valid selectivity labels from \command{selectivity}}

\defSub{initialisation\_time\_steps} {Define the time-steps during the initialisation phases at the end of which the derived quantity by cell is calculated}
\defType{String}
\defDefault{No default}
\defValue{A valid time-step label from \command{time\_step}}

\defSub{time\_step} {Define the time-step at the end of which the derived quantity by cell is calculated}
\defType{String}
\defDefault{No default}
\defValue{A valid time-step label from \command{time\_step}}

\defSub{layer} {Define the layer to be used in the calculations}
\defType{String}
\defDefault{None}
\defValue{A valid numeric layer as defined by \command{layer}. Note, the final values of the derived quantity by cell is the sum of the each cell is multiplied by the value of this layer.}

\subsection{\I{Derived quantities}}

The individual types of derived quantities available are,

\begin{itemize}
	\item Abundance
	\item Biomass
\end{itemize}

\defComLab{derived\_quantity} {Define a derived quantity with label}

\defSub{type} {Define the type of derived quantity}
\defType{String}
\defDefault{No default}
\defValue{A valid type of derived quantity, either \texttt{abundance} or \texttt{biomass}}

\subsubsection[Abundance]{\commandlabsubarg{derived\_quantity}{type}{abundance}}

\defSub{categories} {Define the categories are used to calculate the derived quantity}
\defType{String vector}
\defDefault{No default}
\defValue{Valid categories from \commandsub{model}{categories}}

\defSub{selectivities} {Define the selectivities}
\defType{String vector, of length \subcommand{categories}}
\defDefault{No default}
\defValue{Valid selectivity labels from \command{selectivity}}

\defSub{initialisation\_time\_steps} {Define the time-steps during the initialisation phases at the end of which the derived quantity is calculated}
\defType{String}
\defDefault{No default}
\defValue{A valid time-step label from \command{time\_step}}

\defSub{time\_step} {Define the time-step at the end of which the derived quantity is calculated}
\defType{String}
\defDefault{No default}
\defValue{A valid time-step label from \command{time\_step}}

\defSub{layer} {Define the layer to be used in the calculations}
\defType{String}
\defDefault{None}
\defValue{A valid numeric layer as defined by \command{layer}. Note, the final value of the derived quantity is the sum of the each cell is multiplied by the value of this layer.}

\subsubsection[Biomass]{\commandlabsubarg{derived\_quantity}{type}{biomass}}

\defSub{categories} {Define the categories are used to calculate the derived quantity}
\defType{String vector}
\defDefault{No default}
\defValue{Valid categories from \commandsub{model}{categories}}

\defSub{selectivities} {Define the selectivities}
\defType{String vector, of length \subcommand{categories}}
\defDefault{No default}
\defValue{Valid selectivity labels from \command{selectivity}}

\defSub{initialisation\_time\_steps} {Define the time-steps during the initialisation phases at the end of which the derived quantity is calculated}
\defType{String}
\defDefault{No default}
\defValue{A valid time-step label from \command{time\_step}}

\defSub{time\_step} {Define the time-step at the end of which the derived quantity is calculated}
\defType{String}
\defDefault{No default}
\defValue{A valid time-step label from \command{time\_step}}

\defSub{layer} {Define the layer to be used in the calculations}
\defType{String}
\defDefault{None}
\defValue{A valid numeric layer as defined by \command{layer}. Note, the final value of the derived quantity is the sum of the each cell is multiplied by the value of this layer.}

\subsection{\I{Age-size relationship}}

The individual types of size-at-age relationships available are,

\begin{itemize}
	\item none
	\item von Bertalanffy
	\item Schnute
\end{itemize}

\defComLab{age\_size} {Define an age-size relationship with label}

\defSub{type} {Define the type of size-at-age relationship}
\defType{String}
\defDefault{No default}
\defValue{A valid type of size-at-age relationship}

\subsubsection[None]{\commandlabsubarg{age\_size}{type}{none}}

There are no other subcommands for \commandlabsubarg{age\_size}{type}{none}.

\subsubsection[von Bertalanffy]{\commandlabsubarg{age\_size}{type}{von\_bertalanffy}}

\defSub{linf} {Define the $L_\infty$ parameter of the von Bertalanffy relationship}
\defType{Estimable}
\defDefault{No default}
\defValue{A positive real number}
\defCondition{Only define if using the von Bertalanffy relationship}

\defSub{k} {Define the $k$ parameter of the von Bertalanffy relationship}
\defType{Estimable}
\defDefault{No default}
\defValue{A positive real number}
\defCondition{Only define if using the von Bertalanffy relationship}

\defSub{t0} {Define the $t_0$ parameter of the von Bertalanffy relationship}
 \defType{Estimable}
\defDefault{No default}
\defValue{A real number}
\defCondition{Only define if using the von Bertalanffy relationship}

\defSub{distribution} {Define the distribution of sizes-at-age around the mean}
\defType{String}
\defDefault{Normal}
\defValue{Either normal or lognormal}
\defCondition{Only define if using the von Bertalanffy relationship}

\defSub{by\_length} {Specifies if the linear interpolation of c.v.s is a linear function of mean size or of age}
\defType{Logical}
\defDefault{True}
\defValue{If true, the c.v. is a function of length, else a function of age}

\defSub{cv} {Define the c.v. of the distribution of sizes-at-age around the mean}
\defType{Estimable}
\defDefault{No default}
\defValue{A positive real number}
%
%\defSub{growth\_proportions} {Define the proportion of the year for each time-step for evaluating size}
%\defType{Constant vector}
%\defDefault{No default}
%\defValue{A vector of values, $\le1$ of length equal to the number of time-steps}

\defSub{size\_weight} {Define the label of the associated size-weight relationship}
\defType{String}
\defDefault{No default}
\defValue{A valid label from \command{size\_weight}}

\subsubsection[Schnute]{\commandlabsubarg{age\_size}{type}{schnute}}

\defSub{y1} {Define the $y_1$ parameter of the Schnute relationship}
\defType{Estimable}
\defDefault{No default}
\defValue{A positive real number}
\defCondition{Only define if using the Schnute relationship}

\defSub{y2} {Define the $y_2$ parameter of the Schnute relationship}
\defType{Estimable}
\defDefault{No default}
\defValue{A positive real number}
\defCondition{Only define if using the Schnute relationship}

\defSub{tau1} {Define the $\tau_1$ parameter of the Schnute relationship}
\defType{Estimable}
\defDefault{No default}
\defValue{A real number}
\defCondition{Only define if using the Schnute relationship}

\defSub{tau2} {Define the $\tau_2$ parameter of the Schnute relationship}
\defType{String}
\defDefault{Normal}
\defValue{A positive real number}
\defCondition{Only define if using the Schnute relationship}

\defSub{a} {Define the $a$ parameter of the Schnute relationship}
\defType{String}
\defDefault{Normal}
\defValue{A positive real number}
\defCondition{Only define if using the Schnute relationship}

\defSub{b} {Define the $b$ parameter of the Schnute relationship}
\defType{String}
\defDefault{Normal}
\defValue{Either normal or lognormal}
\defCondition{Only define if using the Schnute relationship}

\defSub{distribution} {Define the distribution of sizes-at-age around the mean}
\defType{String}
\defDefault{Normal}
\defValue{Either normal or lognormal}
\defCondition{Only define if using the von Bertalanffy relationship}

\defSub{by\_length} {Specifies if the linear interpolation of c.v.s is a linear function of mean size or of age}
\defType{Logical}
\defDefault{True}
\defValue{If true, the c.v. is a function of length, else a function of age}

\defSub{cv} {Define the c.v. of the distribution of sizes-at-age around the mean}
\defType{Estimable}
\defDefault{No default}
\defValue{A positive real number}
%
%\defSub{growth\_proportions} {Define the proportion of the year for each time-step for evaluating size}
%\defType{Constant vector}
%\defDefault{No default}
%\defValue{A vector of values, $\le1$ of length equal to the number of time-steps}
%
\defSub{size\_weight} {Define the label of the associated size-weight relationship}
\defType{String}
\defDefault{No default}
\defValue{A valid label from \command{size\_weight}}

\subsection{\I{Size-weight}}

The individual types of size-weight relationship available are,

\begin{itemize}
	\item None
	\item Basic
\end{itemize}

\defComLab{size\_weight} {Define a size-weight relationship with label}

\defSub{type} {Define the type of relationship}
\defType{String}
\defDefault{No default}
\defValue{A valid type of size-weight relationship}

\subsubsection[None]{\commandlabsubarg{size\_weight}{type}{none}}

There are no other subcommands for \commandlabsubarg{size\_weight}{type}{none}.

\subsubsection[Basic]{\commandlabsubarg{size\_weight}{type}{basic}}

\defSub{a} {Define the $a$ parameter of the basic size-weight relationship}
\defType{Estimable}
\defDefault{No default}
\defValue{A positive real number}

\defSub{b} {Define the $b$ parameter of the basic size-weight relationship}
\defType{Estimable}
\defDefault{No default}
\defValue{A positive real number}

\subsection{\I{Selectivities}}

The individual selectivity functions available are,

\begin{itemize}
	\item Constant
	\item Knife edge
	\item All values
	\item All values bounded
	\item Increasing
	\item Logistic
	\item Inverse Logistic
	\item Logistic producing
	\item Double normal
	\item Double exponential
	\item Cubic spline
\end{itemize}

Each type of selectivity function requires a set of subcommands and arguments specific to that function.

\defComLab{selectivity} {Define a selectivity function with label}

\defSub{type} {Define the type of selectivity function}
\defType{String}
\defDefault{No default}
\defValue{A valid type of selectivity function}

\subsubsection[Constant]{\commandlabsubarg{selectivity}{type}{constant}}

\defSub{c} {Defines the $C$ parameter of the selectivity function}
\defType{Estimable}
\defDefault{No default}
\defValue{A positive real number}

\subsubsection[Knife-edge]{\commandlabsubarg{selectivity}{type}{knife\_edge}}

\defSub{e} {Defines the $E$ parameter of the selectivity function}
\defType{Estimable}
\defDefault{No default}
\defValue{A positive real number}

\subsubsection[All-values]{\commandlabsubarg{selectivity}{type}{all\_values}}

\defSub{v} {Defines the $V$ parameters (one for each age class) of the selectivity function}
\defType{Estimable vector}
\defDefault{No default}
\defValue{A vector of positive real numbers, of length equal to the number of age classes}

\subsubsection[All-values-bounded]{\commandlabsubarg{selectivity}{type}{all\_values\_bounded}}

\defSub{l} {Defines the $L$ parameter of the selectivity function}
\defType{Integer}
\defDefault{No default}
\defValue{A positive real number}

\defSub{h} {Defines the $H$ parameter of the selectivity function}
\defType{Integer}
\defDefault{No default}
\defValue{A positive real number, must be greater than $L$}

\defSub{v} {Defines the $V$ parameters (one for each age class from $L$ to $H$) of the selectivity function}
\defType{Estimable vector}
\defDefault{No default}
\defValue{A vector of positive real numbers, of length equal to the number of age classes from $L$ to $H$}

\subsubsection[Increasing]{\commandlabsubarg{selectivity}{type}{increasing}}

\defSub{alpha} {Defines the $\alpha$ parameter of the selectivity function}
\defType{Estimable}
\defDefault{1.0}
\defValue{A positive real number}

\defSub{l} {Defines the $L$ parameter of the selectivity function}
\defType{Integer}
\defDefault{No default}
\defValue{A positive real number}

\defSub{h} {Defines the $H$ parameter of the selectivity function}
\defType{Integer}
\defDefault{No default}
\defValue{A positive real number, must be greater than $L$}

\defSub{v} {Defines the $V$ parameters (one for each age class from $L$ to $H$) of the selectivity function}
\defType{Estimable vector}
\defDefault{No default}
\defValue{A vector of positive real numbers, of length equal to the number of age classes from $L$ to $H$}

\subsubsection[Logistic]{\commandlabsubarg{selectivity}{type}{logistic}}

\defSub{alpha} {Defines the $\alpha$ parameter of the selectivity function}
\defType{Estimable}
\defDefault{1.0}
\defValue{A positive real number}

\defSub{a50} {Defines the $a_{50}$ parameter of the selectivity function}
\defType{Estimable}
\defDefault{No default}
\defValue{A positive real number}

\defSub{ato95} {Defines the $a_{to95}$ parameter of the selectivity function}
\defType{Estimable}
\defDefault{No default}
\defValue{A positive real number}

\subsubsection[InverseLogistic]{\commandlabsubarg{selectivity}{type}{inverse\_logistic}}

\defSub{alpha} {Defines the $\alpha$ parameter of the selectivity function}
\defType{Estimable}
\defDefault{1.0}
\defValue{A positive real number}

\defSub{a50} {Defines the $a_{50}$ parameter of the selectivity function}
\defType{Estimable}
\defDefault{No default}
\defValue{A positive real number}

\defSub{ato95} {Defines the $a_{to95}$ parameter of the selectivity function}
\defType{Estimable}
\defDefault{No default}
\defValue{A positive real number}

\subsubsection[Logistic producing]{\commandlabsubarg{selectivity}{type}{logistic\_producing}}

\defSub{alpha} {Defines the $\alpha$ parameter of the selectivity function}
\defType{Estimable}
\defDefault{1.0}
\defValue{A positive real number}

\defSub{l} {Defines the $L$ parameter of the selectivity function}
\defType{Integer}
\defDefault{No default}
\defValue{A positive real number}

\defSub{h} {Defines the $H$ parameter of the selectivity function}
\defType{Integer}
\defDefault{No default}
\defValue{A positive real number, must be greater than $L$}

\defSub{a50} {Defines the $a_{50}$ parameter of the selectivity function}
\defType{Estimable}
\defDefault{No default}
\defValue{A positive real number}

\defSub{ato95} {Defines the $a_{to95}$ parameter of the selectivity function}
\defType{Estimable}
\defDefault{No default}
\defValue{A positive real number}

\subsubsection[Double-normal]{\commandlabsubarg{selectivity}{type}{double\_normal}}

\defSub{alpha} {Defines the $\alpha$ parameter of the selectivity function}
\defType{Estimable}
\defDefault{1.0}
\defValue{A positive real number}

\defSub{mu} {Defines the $\mu$ parameter of the selectivity function}
\defType{Estimable}
\defDefault{No default}

\defSub{sigma\_l} {Defines the $\sigma_L$ parameter of the selectivity function}
\defType{Estimable}
\defDefault{No default}

\defSub{sigma\_r} {Defines the $\sigma_R$ parameter of the selectivity function}
\defType{Estimable}
\defDefault{No default}

\subsubsection[Double-exponential]{\commandlabsubarg{selectivity}{type}{double\_exponential}}

\defSub{alpha} {Defines the $\alpha$ parameter of the selectivity function}
\defType{Estimable}
\defDefault{1.0}
\defValue{A positive real number}

\defSub{x1} {Defines the $x_1$ parameter of the selectivity function}
\defType{Integer}
\defDefault{No default}

\defSub{x2} {Defines the $x_2$ parameter of the selectivity function}
\defType{Integer}
\defDefault{No default}

\defSub{x0} {Defines the $x_0$ parameter of the selectivity function}
\defType{Estimable}
\defDefault{No default}

\defSub{y0} {Defines the $y_0$ parameter of the selectivity function}
\defType{Estimable}
\defDefault{No default}

\defSub{y1} {Defines the $y_1$ parameter of the selectivity function}
\defType{Estimable}
\defDefault{No default}

\defSub{y2} {Defines the $y_2$ parameter of the selectivity function}
\defType{Estimable}
\defDefault{No default}

\subsubsection[Spline]{\commandlabsubarg{selectivity}{type}{spline}}

\defSub{alpha} {Defines the $\alpha$ parameter of the selectivity function}
\defType{Estimable}
\defDefault{1.0}
\defValue{A positive real number}

\defSub{knots} {Defines the locations of the knots for the cubic spline function}
\defType{Constant vector}
\defDefault{No default}
\defValue{A vector of real numbers of length equal to the number of values}

\defSub{values} {Defines the values at the knots for the cubic spline function}
\defType{Estimable vector}
\defDefault{No default}
\defValue{A vector of real numbers of length equal to the number of knots}

\defSub{method} {The method for constraining the end values of the spline}
\defType{String}
\defDefault{Natural}
\defValue{Either, \texttt{natural} (sets the 2nd derivatives to $0$ at the boundaries),  \texttt{fixed} (sets the 1st derivatives to zero at the boundaries), or \texttt{parabolic} sets the spline to be a parabola at the boundaries)}

%Not implemented yet. A way to apply joint selectivity would be to arbitrarily fix the first selectivity to one and only estimate the second one (say using a pointer?). If the estimated selectivity is above 1, it gets fixed to 1 and the other one gest estimated. Both get reported.
%
%\defComLab{joint\_selectivity} {Define a joint selectivity}
%
%\defSub{selectivities} {Define the labels of the selectivities to be defined as 'joint'}
%\defType{String vector of length 2}
%\defDefault{No default}
%\defValue{Valid \command{selectivity} labels}
